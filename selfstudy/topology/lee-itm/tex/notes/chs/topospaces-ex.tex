\subsection{Exercises}

\begin{exr}[source=Primary Source Material]
    fill in the following pfs:
    \begin{itemize}
        \item prop 1.2
        \item prop 1.4
        \item prop 2.4
        \item sequence lemma (below defn 3.3)
        \item prop 4.7 (w/o invariance of bdry)
    \end{itemize}
\end{exr}

\begin{exr}[source=Primary Source Material]
    Let $ X $ be an infinite set. Show that each of the following are topologies.
    \begin{itemize}
        \item $ \cl{T}_{1} \ = \ \set{U \subseteq X: \abs{U} < \infty} \cup
            \set{\eset} $, called the \textbf{finite complement (cofinite)
            topology}.
        \item $ \cl{T}_{2} \ = \ \set{U \subseteq X: \abs{U} \leq \abs{\bN}} \cup
            \set{\eset} $, called the \textbf{countable complement (cocountable)
            topology}.
        \item $ \cl{T}_{3} \ = \ \set{U \subseteq X: U = \eset \trm{ or } p \in
            U} $ for some fixed point $ p $, called the \textbf{particular point
            topology}.
        \item $ \cl{T}_{4} \ = \ \set{U \subseteq X: U = X \trm{ or } p \notin
            U} $ for some fixed point $ p $, called the \textbf{excluded point
            topology}.
    \end{itemize}
\end{exr}

\newpage
\begin{exr}[source=Primary Source Material]
    Let $ X $ be a space, $ B \subseteq X $ a subset, and $ \cl{A} $ a collection
    of subsets of $ X $. Prove each of the following.
    \begin{enumerate}
        \item $ \bar{X \setminus B} = X \setminus \sint(B) $.
        \item $ \sint(X\setminus B) = X \setminus \bar{B} $.
        \item $ \bar{\displaystyle\bigcap_{A \in \cl{A}}A} \subseteq
            \displaystyle\bigcap_{A \in \cl{A}} \bar{A} $.
        \item $ \displaystyle\bigcup_{A \in \cl{A}} \bar{A} \subseteq
            \bar{\displaystyle\bigcup_{A \in \cl{A}}A} $.
        \item $ \sint\left(\displaystyle\bigcap_{A \in \cl{A}}A\right) \subseteq
            \displaystyle\bigcap_{A \in \cl{A}} \sint A $.
        \item $ \displaystyle\bigcup_{A \in \cl{A}} \sint A \subseteq
            \sint\left(\displaystyle\bigcup_{A \in \cl{A}}A\right) $.
    \end{enumerate}
    Furthermore, show that when $ \cl{A} $ is finite, equality holds in 4 and 5,
    but not necessarily in 3 or 6.
\end{exr}

\begin{exr}[source=Primary Source Material]
    For each of the following, provide two subsets $ X, Y \subseteq \bR^{2} $ as
    Euclidean spaces, and a map $ f:X\gto Y $ with the indicated property.
    \begin{itemize}
        \item $ f $ is open but not closed nor continuous.
        \item $ f $ is closed but not open nor continuous.
        \item $ f $ is continuous but not open nor closed.
        \item $ f $ is open and closed but not continuous.
        \item $ f $ is continuous and open but not closed.
        \item $ f $ is continuous and closed but not open.
    \end{itemize}
\end{exr}

\begin{exr}[source=Primary Source Material]
    Suppose we have spaces $ D, T, H, A $ with $ D $ discrete, $ T $ trivial,
    $ H $ Hausdorff, and $ A $ arbitrary. Prove each of the following claims:
    \begin{itemize}
        \item Every map from $ D $ to $ A $ is continuous.
        \item Every map from $ A $ to $ T $ is continuous.
        \item If a map from $ T $ to $ H $ is continuous, then it is constant.
    \end{itemize}
\end{exr}

\begin{exr}[source=Primary Source Material]
    Suppose $ f,g: X \gto Y $ are continuous with $ Y $ Hausdorff. Show that the
    set $ S = \set{x: f(x) = g(x)} $ is closed in $ X $. Give a counterexample if
    $ Y $ is not Hausdorff.
\end{exr}

\begin{exr}[source=Primary Source Material]
    Let $ f:X \gto Y $ be continuous, and $ \cl{B} $ a basis of $ X $.
    Show that $ f(\cl{B}) $ is a basis of $ Y $ iff $ f $ is open and surjective.
\end{exr}

\begin{exr}[source=Primary Source Material]
    Let $ X $ be a set, $ \cl{A} $ a collection of subsets. Let $ \cl{T} $ be
    the collection of subsets consisting of $ X, \eset $, and all unions of
    finite intersections of sets in $ \cl{A} $.
    \begin{enumerate}
        \item Show that $ \cl{T} $ is a topology, called the topology
            \textbf{generated} by $ \cl{A} $, with $ \cl{A} $ called the
            \textbf{subbasis} for $ \cl{T} $.
        \item Show that $ \cl{T} $ is the coarsest topology for which all the
            sets in $ \cl{A} $ are open.
        \item Let $ Y $ be any space. Show that a map $ f:Y \gto X $ is
            continuous iff $ f^{-1}(U) $ is open in $ Y $ for every $ U \in
            \cl{A} $.
    \end{enumerate}
\end{exr}

\begin{exr}[source=Primary Source Material]
    Let $ X $ be a totally ordered set. Equip $ X $ with the \textbf{order
    topology}, the topology generated by the subbasis consisting of all sets of
    the following forms for $ a \in X $:
    \begin{equation*}
        (a, \infty) = \set{x: a < x} \qquad (-\infty, a) = \set{x: a > x}
    \end{equation*}
    \begin{enumerate}
        \item Show that $ (a,b) $ is open in $ X $ and $ [a,b] $ is closed in
            $ X $ (with these sets defined the natural way).
        \item Show that $ X $ is Hausdorff.
        \item Given $ a < b $, show $ \bar{(a, b)} \subseteq [a,b] $. Give an
            example showing that equality need not hold.
        \item Show that the order topology on $ \bR $ is the same as the
            Euclidean topology.
    \end{enumerate}
\end{exr}

\begin{exr}[source=Primary Source Material]
    Show that if $ X $ is first countable and $ f:X \gto Y $ is a map such that
    $ p_{n} \sto p $ implies $ f(p_{n}) \sto f(p) $, then $ f $ is continuous.
    (self-addition) Show necessity of first countability.
\end{exr}

\begin{exr}[source=Primary Source Material]
    Recall the topologies defined in Exercise 5.2.
    \begin{itemize}
        \item Show that $ \bR $ with the particular point topology is first
            countable and separable, but not second countable or Lindelof.
        \item Show that $ \bR $ with the excluded point topology is first
            countable and Lindelof, but not second countable or separable.
        \item Show that $ \bR $ with the cofinite topology is separable and
            Lindelof, but not first or second countable.
    \end{itemize}
\end{exr}

\begin{exr}[source=Primary Source Material]
    Let $ X $ be a space and $ \cl{U} $ an open cover.
    Show that if each $ U \in \cl{U} $ has a basis, the union of all such bases
    is a basis of $ X $. Furthermore, show that if $ \cl{U} $ is countable and
    each $ U $ is second countable, then $ X $ is second countable.
\end{exr}

\begin{exr}[source=Primary Source Material]
    Show that 2nd countable, separable, and Lindelof are equivalent for
    metric spaces.
\end{exr}

\begin{exr}[source=Primary Source Material]
    Suppose $ X $ is locally Euclidean of dimension $ n $, and $ f:X \gto Y $
    a surjective local homeomorphism. Show that $ Y $ is locally Euclidean of
    dimension $ n $.
\end{exr}

