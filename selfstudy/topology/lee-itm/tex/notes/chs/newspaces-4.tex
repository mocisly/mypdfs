\subsection{Quotient Spaces}
our last construction is more involved, and makes spaces smaller. this plays an
important role in ``cutting and pasting" arguments used to define many manifolds.

\begin{defn}
    given a space $ X $, set $ Y $, and surj map $ q:X\sto Y $, define a subset
    $ U \subseteq Y $ as open iff $ q^{-1}(U) $ open in $ X $. this is the
    \textbf{quotient topology} induced by $ q $. in this case, $ q $ is called a
    \textbf{quotient map}.
\end{defn}

the most common construction is to quotient by some equivalence relation, with
the natural projection as our quotient map. book gives many examples; we
highlight a few.

\begin{xmp}[source=Primary Source Material]
    we define $\bb{P}^{n}$, the \textbf{real projective space} of dimension $n$,
    as all 1-dim linear subspaces in $\bR^{n+1}$. there is a natural map
    $q:\bR^{n+1}\setminus\set{0}\sto\bb{P}^{n}$ sending vectors to their spans.
    the space is given the qtopo wrt $q$.
\end{xmp}

\newpage
\begin{xmp}[source=Primary Source Material]
    let $ X $ be a space, $ A \subseteq X $ a subset. define $ \sim $ as the
    equivalence relation generated by $ a_{1} \sim a_{2} $ for $ a_{1},a_{2} \in
    A $. then, $ X/A $ is the set of singletons $ [x] $ for
    $ x \in X\setminus A $, and a single class representing all of $ A $. this
    space is said to be obtained by \textbf{collapsing} $ \bm{A} $ \textbf{to a
    point}. for instance, we will see that $ \bar{\bb{B}^{n}}/\bb{S}^{n-1} \simeq
    \bb{S}^{n} $.
\end{xmp}

\begin{xmp}[source=Primary Source Material]
    given a space $ X $, the quotient $ (X\times (0,1))/(X \times \set{0}) $ is
    called the \textbf{cone on} $ \bm{X} $ denoted by $ CX $. as an example, we
    will see that $ C\bb{S}^{n} \simeq \bb{B}^{n} $.
\end{xmp}

\begin{xmp}[source=Primary Source Material]
    given spaces $ X_{1}, \dots, X_{k} $, we define the \textbf{wedge sum} 
    $ X_{1} \vee \cdots \vee X_{k} $ as the quotient obtained from the disjoint
    union, and the relation which collapses the set $ \set{p_{1},\dots,p_{k}} $
    to a point. each $ p_{i} $ is some fixed point in $ X_{i} $, called the
    \textbf{base point for} $ \bm{X_{i}} $.

    more generally, we can do this for an arbitrary family of spaces. the wedge
    sum is sometimes called the \textbf{one-point union}.
\end{xmp}

unlike subspaces and prod spaces, quotient spaces do not behave as nicely. in
particular, none of the key manifold properties (hausdorff, 2nd ctbl, locally
euclidean) are automatically inherited. if we want a quotient space to be a
manifold, we need at \textit{least} locally euclidean and hausdorff. in many
cases, this is sufficient.

\begin{prop}
    sps $ P $ 2nd ctbl and $ M $ a quotient space of $ P $. if $ M $ locally
    euclidean, then it is 2nd ctbl. if its also hausdorff, it is thus a manifold.
\end{prop}

\newpage
\begin{pf}[source=Primary Source Material]
    given the quotient map $ q $, let $ \cl{U} $ be a cover of $ M $ by
    coordinate balls. then the set $ \set{q^{-1}(U):U\in\cl{U}} $ is an open
    cover of $ P $, which has a ctbl subcover. sps $ \cl{U}' \subseteq \cl{U} $
    is ctbl, such that $ q^{-1}(\cl{U}) $ covers $ P $. then $ \cl{U}' $ ctbly
    covers $ M $, and since each is 2nd ctbl, $ M $ is thus 2nd ctbl.
\end{pf}

usually, proving hausdorff requires using the defin. but for \textit{open}
quotient maps, we can use a criterion.

\begin{prop}
    let $ q:X\sto Y $ be an open quotient map. then $ Y $ hausdorff iff the set
    given by:
    \begin{equation*}
        R=\set{(x_{1},x_{2}):q(x_{1})=q(x_{2})}
    \end{equation*}
    is closed in $ X^{2} $.
\end{prop}

\begin{pf}[source=Primary Source Material]
    sps $ Y $ hausdorff. if $ (x_{1},x_{2}) \notin R $, then there exist disjoint
    nbhds $ V_{1}, V_{2} $ of $ q(x_{1})$ and $q(x_{2}) $. then $ q^{-1}(V_{1})
    \times q^{-1}(V_{2}) $ is a nbhd of $ (x_{1},x_{2}) $ disjoint from $ R $.
    thus $ X^{2} \setminus R $ is open, and so $ R $ is closed.

    sps $ R $ closed. given distinct $ y_{1}, y_{2} \in Y $, fix $ x_{1},x_{2}
    \in X $ s.t. $ q(x_{i})=y_{i} $. since $ (x_{1},x_{2}) \notin R $, there
    exists $U_{1}\times U_{2}$ nbhd of $(x_{1},x_{2})$ in $X^{2}$ disjoint from
    $R$. since $q$ open, $q(U_{1})$ and $q(U_{2})$ thus disjoint nbhds of $y_{1},
    y_{2}$ as needed.
\end{pf} \

\begin{crll}
    sps $\sim$ an equiv relation on $X$. if $X\sto X/\sim$ an open map, then
    $X/\sim$ hausdorff iff $\sim$ a closed subset of $X^{2}$.
\end{crll}

