\subsection{Subspaces}
okay i think imma start being semi lazy w this. tbh. lol. ok maybe a bit more
than semi

\begin{prop}[type=Theorem,title=Characteristic Property of the Subspace Topology]
    Let $ X, Y $ be spaces and $ S \subseteq X $ a subspace. A map $ f:Y \gto S $
    is continuous iff the map $ \iota_{S}\circ f:Y\gto X $ is continuous, where
    $ \iota_{S} $ is the inclusion map.
\end{prop}

\begin{pf}[source=Primary Source Material]
    If $ \iota_{S}\circ f $ is continuous, then for any open $ U \subseteq S $,
    there is an open $ V \subseteq X $ such that $ U = S \cap V =
    \iota_{S}^{-1}(V) $. Then:
    \begin{equation*}
        f^{-1}(U) \ = \ f^{-1}(\iota_{S}^{-1}(V)) \ = \
        (\iota_{S}\circ f)^{-1}(V)
    \end{equation*}
    This is open since $ \iota_{S}\circ f $ is continuous, and so $ f $ is cts.

    Now, sps $ f $ cts. For open $ V \subseteq X $, we have:
    \begin{equation*}
        (\iota_{S}\circ f)^{-1}(V) \ = \ f^{-1}(\iota_{S}^{-1}(V)) \ = \
        f^{-1}(S \cap V)
    \end{equation*}
    This is open in $ Y $ since $ S \cap V $ open in $ S $, so
    $ \iota_{S}\circ f $ cts.
\end{pf} \

\begin{crll}
    The inclusion map $ \iota_{S}:S \gto X $ is continuous. Furthermore, we can
    restrict the domain or expand/restrict the codomain of any cts map and the
    result is still cts.
\end{crll}

\begin{prop}
    Hausdorff, first, and second ctbl are inheritable by subspace.
\end{prop}
See Exercise ?? for proof.

\begin{prop}
    A cts inj map that is either open or closed is an embedding.
\end{prop}

\begin{pf}[source=Primary Source Material]
    Sps $ f:X \gto Y $ is cts inj. Then $ f':X \gto f(X) $ is cts bij. If $ f $
    open, then clearly $ f' $ open, and the result follows. arg for closed is
    identical.
\end{pf} \

\begin{lm}[title=Gluing Lemma]
    Let $ X, Y $ be spaces, and $ \set{A_{i}} $ either an arbitrary open cover or
    finite closed cover of $ X $. Sps we have maps $ f_{i}:A_{i}\gto Y $ such
    that $ f_{i}\rvert_{A_{i}\cap A_{j}} = f_{j}\rvert_{A_{i}\cap A_{j}} $. Then
    there exists a unique $ f:X \gto Y $ cts, with $ f\rvert_{A_{i}} = f_{i} $.
\end{lm}

\begin{pf}[source=Primary Source Material]
    existence and uniqueness of $ f $ follows from set theory, continuity from
    open cover follows from local criterion. otw, if $ \set{A_{1},\dots,A_{k}} $
    closed cover, let $ K \subseteq Y $ closed. then $ f^{-1}(K) \cap A_{i}
    = f_{i}^{-1}(K) $, and $ f^{-1}(K) $ is the union over all $ i $. since
    $ f_{i}^{-1}(K) $ closed in $ A_{i} $ by ctsness and $ A_{i} $ closed,
    $ f_{i}^{-1}(K) $ closed in $ X $, and result follows
\end{pf}

When choosing a topology for $ S \subseteq X $, we want the inclusion map to be
cts and we want cts maps whose images are in $ S $ to be cts as maps into $ S $.
The first asks that $ S $ has enough open subsets; the second asks it doesn't
have too many. The subspace topology is the optimal compromise between them.

Natural topologies such as this can usually be characterized in terms of cts maps
wrt them. This is where Theorem 6.1 gets its name. The next theorem makes this
precise.

\newpage
\begin{prop}
    Given $ S \subseteq X $, the subspace topology is the unique topology [up to
    homeomorphism] for which the characteristic property holds.
\end{prop}

\begin{pf}[source=Primary Source Material]
    denote by $ S_{g} $ an arbitrary topology with the given property, and by
    $ S_{s} $ the subspace topology. we show $ I:S_{s}\gto S_{g} $ is cts.

    note the inclusions $ \iota_{g}:S_{g} \ito X $ and $ \iota_{s}:S_{s} \ito X $
    cts. thus, $ \iota_{g}=\iota_{s}\circ I^{-1} $ and $ \iota_{s}=\iota_{g}
    \circ I $ cts, so by the characteristic property $ I $ is a homeo as needed.
\end{pf}

