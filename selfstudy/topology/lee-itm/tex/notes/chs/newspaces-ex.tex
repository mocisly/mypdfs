\subsection{Exercises}

\begin{exr}[source=Primary Source Material]
    fill in the following missing proofs:
    \begin{itemize}
        \item prop 6.2
        \item prop 7.4
        \item thm 8.2
        \item prop 8.3
        \item prop 10.2
        \item prop 10.3
        \item thm 10.5 (uniqueness)
        \item prop 12.2
    \end{itemize}
\end{exr}

\begin{exr}[source=Primary Source Material]
    sps $M$ an $n$-mfld w/ bdry. show $\p M$ is an $(n-1)$-mfld w/o bdry under
    the subsp top. you may assume that $\sint(M) \cap \p M = \eset$.
\end{exr}

\begin{exr}[source=Primary Source Material]
    show by ctrxmp that the gluing lemma fails if $\set{A_{i}}$ is an infinite
    closed cover.
\end{exr}

\newpage
\begin{exr}[source=Primary Source Material]
    show that every closed ball in $\bR^{n}$ is an $n$-mfld w/ bdry, as is the
    complement of every open ball. assuming invariance of bdry, show that the
    mfld bdry agrees w/ the topo bdry as a subset of $\bR^{n}$, namely a sphere.

    hint: consider $\pi\circ\sigma^{-1}:\bR^{n}\sto\bR^{n}$, where $\sigma$ is
    stereographic projection and $\pi$ omits some coordinate other than the last.
\end{exr}

\begin{exr}[source=Primary Source Material]
    prove the sum/product of cts (real/complex) functions are cts using the
    characteristic property of product spaces
\end{exr}

\begin{exr}[source=Primary Source Material]
    show that a finite product of open maps is open. give a ctrxmp to show the
    same need not hold for closed maps.
\end{exr}

\begin{exr}[source=Primary Source Material]
    show that $X$ hausdorff iff the diagonal $\Delta=\set{(x,x)}\subseteq
    X\times X$ closed.
\end{exr}

\begin{exr}[source=Primary Source Material]
    let $X = \prod_{i=1}^{\infty}\bR$ be the space of real sequences.
    define a basis as the collection of all product sets of the form:
    \begin{equation*}
        \prod_{i=1}^{\infty}U_{i}
    \end{equation*}
    where $U_{i}$ is open in $X_{i}$ for each $i$. the topology generated by this
    basis is known as the \textbf{box topology}.

    define a map $f:\bR\sto X$ as $f(x)=(x,x,x,\dots)$. show that $f$ is not
    cts.
\end{exr}

\newpage
\begin{exr}[source=Primary Source Material]
    with $X$ as above, consider $X^{+}\subseteq X$ as the subsp consisting of
    sequences of strictly positive reals. let $z$ denote the 0 sequence. show
    $z$ in closure of $X^{+}$, but there is no sequence converging to it. then
    use the squeeze lemma to conclude that $X$ is not first ctbl, and thus not
    metrizable.
\end{exr}

\begin{exr}[source=Primary Source Material]
    let $q:X\sto Y$ be a map, $U\subseteq X$ any subset. show the following are
    equivalent
    \begin{itemize}
        \item $U$ is saturated
        \item $U = q^{-1}(q(U))$
        \item $U$ is a union of fibers
        \item if $x\in U$, every $x'\in X$ with $q(x)=q(x')$ also in $U$
    \end{itemize}
\end{exr}

\begin{exr}[source=Primary Source Material]
    fix $X$ topsp, $(X_{\alpha})$ indexed family of topsps.
    \begin{itemize}
        \item for any $S \subseteq X$, show the subsptop is the coarsest topo on
            $S$ s.t. $\iota_{S}:S\ito X$ is cts.
        \item show the prodtop is the coarsest topo on $\prod_{\alpha}X_{\alpha}$
            s.t. every $\pi_{\alpha}:\prod_{\alpha}X_{\alpha}\sto X_{\alpha}$ is
            cts.
        \item show the disjuniontop is the finest topo on
            $\coprod_{\alpha}X_{\alpha}$ s.t. each $\iota_{\alpha}:X_{\alpha}\sto
            \coprod_{\alpha}X_{\alpha}$ is cts.
        \item show that if $q:X\sto Y$ surj map, the qtopo on $Y$ is the finest
            topo s.t. $q$ cts.
    \end{itemize}
\end{exr}

\newpage
\begin{exr}[source=Primary Source Material]
    show real proj space $\bb{P}^{n}$ is an $n$-mfld. as a hint: consider subsets
    $U_{i}\subseteq\bR^{n+1}$ where $x_{i}=1$.
\end{exr}

\begin{exr}[source=Primary Source Material]
    let $\bb{CP}^{n}$ denote the set of all $1$-dim complex subsps of $\bC^{n+1}$
    called ``$n$-dim complex projective space". topologize as the quotient
    $(\bC^{n+1}\setminus\set{0})/\bC^{\times}$, where $\bC^{\times}$ is the grp
    of nonzero complexes acting by scalar mult. show that $\bb{CP}^{n}$ is a
    $2n$-mfld. hint: mimic exercise 13.11.
\end{exr}

\begin{exr}[source=Primary Source Material]
    let $X=(\bR\times\set{0}\cup\bR\times\set{1})$. define an equiv rel generated
    by $(x,0)\sim(x,1)$ iff $x\neq0$. show $X/\sim$ is locally euclidean and 2nd
    ctbl, but not hausdorff. this is known as the ``line with two origins".
\end{exr}

\begin{exr}[source=Primary Source Material]
    this exercise shows necessity of openness in prop 9.3. let $X$ be:
    \begin{equation*}
        X = (0,1)^{2}\cup\set{(0,0)}\cup\set{(1,0)}
    \end{equation*}
    for any $\ep\in(0,1)$, define:
    \begin{equation*}
        C_{\ep}=\set{(0,0)}\cup((0,\frac{1}{2})\times(0,\ep)) \qquad
        D_{\ep}=\set{(1,0)}\cup((\frac{1}{2},1)\times(0,\ep))
    \end{equation*}
    define a basis $B$ on $X$ consisting of all open rectangles in $X$, along
    with all subsets of the form $C_{\ep}$ or $D_{\ep}$.
    \begin{itemize}
        \item show that $B$ is a basis for a topology on $X$
        \item show that the topology is hausdorff
        \item show that $A=\set{(0,0)}\cup((0,\frac{1}{2}]\times(0,1))$ is closed
            in $X$
        \item let $\sim$ be generated by collapsing $A$ to a point. show that
            $\sim$ is closed in $X^{2}$
        \item show that $X/A$ is not hausdorff
    \end{itemize}
\end{exr}

\begin{exr}[source=Primary Source Material]
    let $A=\bZ$, and $X=\bR/A$ obtained by collapsing $A$ to a point. we avoid
    $\bR/\bZ$ to not conflict with the existing example regarding $\bb{S}^{1}$.
    \begin{itemize}
        \item show $X$ is homeo to a wedge sum of ctbly infinitely many circles.
            hint: express both as quotients of a disjoint union of intervals.
        \item show that the equivalence class of $A$ does not have a ctbl nbhd
            basis in $X$, so $X$ is not 1st/2nd ctbl.
    \end{itemize}
\end{exr}

\begin{exr}[source=Primary Source Material]
    verify the brief list of examples of topo grps. for both GLs, as a hint,
    recall/look up ``Cramer's Rule".
\end{exr}

\begin{exr}[source=Primary Source Material]
    let $G$ be a topogrp and $H \leq G$. show $\bar{H}\leq G$.
\end{exr}

\begin{exr}[source=Primary Source Material]
    let $G$ be a topgrp and $H$ a subgrp.
    \begin{itemize}
        \item for each $g\in G$, show there exists a homeo $\theta_{g}:G/H\sto
            G/H$ s.t.:
            \begin{equation*}
                q\circ L_{g} = \theta_{g}\circ q
            \end{equation*}
            where $L_{g}$ is left translation by $g$. alternatively, this diagram
            commutes: [insert diagram]
        \item show that every coset space is topologically homogeneous
    \end{itemize}
\end{exr}

\newpage
\begin{exr}[source=Primary Source Material]
    let $G$ be acting by homeos on a space $X$, and define $\cl{O}$ as:
    \begin{equation*}
        \cl{O} = \set{(x_{1},x_{2}):x_{1}=gx_{2} \trm{ for some } g\in G}
    \end{equation*}
    this is called the \textbf{orbit relation}.
    \begin{itemize}
        \item show the qmap $X\sto X/G$ is an open map
        \item show $X/G$ hausdorff iff $\cl{O}$ closed in $X^{2}$.
    \end{itemize}
\end{exr}

\begin{exr}[source=Primary Source Material]
    sps $H \ngrp G$. show $G/H$ is a topogrp w/ the qtopo. hint: use the prev
    exercise and 13.6
\end{exr}

\begin{exr}[source=Primary Source Material]
    consider the action of $O(n)$ on $\bR^{n}$. prove the quotient is homeo to
    $[0,\infty)$. hint: consider $f:\bR^{n}\sto[0,\infty)$ given as
    $f(x)=\norm{x}$.
\end{exr}

