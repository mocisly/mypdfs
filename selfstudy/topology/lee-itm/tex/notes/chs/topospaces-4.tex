\subsection{Manifolds}

We're almost ready to give the definition of a manifold. We just need one more
definition which captures the idea that a space ``locally" looks Euclidean.

\begin{defn}
    A space $ M $ is \textbf{locally Euclidean of dimension} $ n $ if every
    point of $ M $ has a neighbourhood in $ M $ that is homeomorphic to an open
    subset of $ \bR^{n} $.
\end{defn}

We can replace open subset with open ball or simply by all of $ \bR^{n} $ itself.
We know this. We're skipping this.

\begin{defn}
    Suppose $ M $ is locally Euclidean of dimension $ n $. Let $ U \subseteq M $
    be open such that it is homeomorphic to an open subset of $ \bR^{n} $.

    We call $ U $ a \textbf{coordinate domain}, and any homeomorphism $ \vphi $
    from $ U $ to a subset of $ \bR^{n} $ is a \textbf{coordinate map}.
    We call the pair $ (U, \vphi) $ a \textbf{coordinate chart} (or just chart)
    for $ M $.

    A coordinate domain homeomorphic to a ball in $ \bR^{n} $ is
    called a \textbf{coordinate ball} (sometimes \textbf{disk}, if $ n=2 $). For
    any $ p \in U $, we call $ U $ a \textbf{coordinate neighbourhood} or
    \textbf{Euclidean neighbourhood} of $ p $.
\end{defn}

The definition makes sense even when $ n=0 $, as $ \bR^{0} $ is a single point
(for us). Thus, a space is locally Euclidean of dimension 0 iff the space is
discrete.

We arrive at the culmination of the chapter: the definition of a manifold.

\begin{defn}
    An $ \bm{n} $\textbf{-dimensional topological manifold} is a second countable
    Hausdorff space which is locally Euclidean of dimension $ n $. We may shorten
    this name in a variety of ways, particularly if context is understood.
\end{defn}

Every open subset of an $ n $-manifold is an $ n $-manifold. We know this.
The proof follows from prior definitions, and... well that's it really.

\begin{exr}[source=Primary Source Material]
    Show that a space is a $ 0 $-manifold iff it is a countable discrete space.
\end{exr}

\begin{prop}[type=Theorem,title=Invariance of Dimension]
    A nonempty topological space cannot be both an $ m $-manifold and an
    $ n $-manifold for $ m \neq n $.
\end{prop}

Proof of $ n=0 $ is trivial and can be done now, $ n=1 $ needs a bit more
topology (but we know it), $ n=2 $ needs some homotopy (again, we know it), and
the general case requires homology. For this book, chapters 4, 8, and 13,
respectively.

\begin{prop}
    A separable locally Euclidean metric space of dimension $ n $ is an
    $ n $-manifold.
\end{prop}

See Exercise ?? for proof. Note that the converse is also true, but metrizability
is more difficult to prove; as we have no need for it or the converse, we will
ignore it.

We also define manifolds with boundary here, making the distinction that they are
not necessarily manifolds. When making this distinction, we specify between
\textbf{interior charts} and \textbf{boundary charts}. Note that when we simply
describe a set as a ``manifold" without any specification toward its boundary, we
will always mean a manifold \textit{without} boundary (in the sense that we
originally defined it). Furthermore, specifying a manifold \textit{with} boundary
does not necessarily mean that the boundary is nonempty.

\begin{prop}
    If $ M $ is an $ n $-manifold with boundary, then $ \sint(M) $ is an open
    subset of $ M $, which is itself an $ n $-manifold without boundary.
\end{prop}

See Exercise ?? for proof.

\newpage
\begin{prop}[type=Theorem,title=Invariance of Boundary]
    If $ M $ is a manifold with boundary, then $ \p M $ and $ \sint(M) $ are
    disjoint.
\end{prop}

This takes more machinery to prove, similar to invariance of dimension above.
We will assume it for now, along with the following corollary.

\begin{crll}
    If $ M $ is a nonempty $ n $-manifold with boundary, then $ \p M $ is closed
    in $ M $. Furthermore, $ M $ is an $ n $-manifold iff $ \p M = \eset $.
\end{crll}

