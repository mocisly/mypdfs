\subsection{Lebesgue Measure}

\begin{defn}
    $E\subseteq\bR^{d}$ is \textbf{(Lebesgue) measurable} if for any $\ep>0$,
    there exists open $O\subseteq\bR^{d}$ with $E\subseteq O$ and:
    \begin{equation*}
        m_{*}(O\setminus E)<\ep
    \end{equation*}
    in the above we define the (Lebesgue) measure of $E$ as $m(E)=m_{*}(E)$.
\end{defn}

it is easy to see that this inherits the same properties listed above.
furthermore, we immediately get that every open set is measurable.

\begin{prop}
    if $m_{*}(E)=0$, then $E$ measurable. in particular, if $F$ is a subset of a
    set with ext measure 0, then $F$ measurable.
\end{prop}

\begin{pf}[source=Primary Source Material]
    for all $\ep>0$ there is an open $O$ with $E\subseteq O$ and $m_{*}(O)<\ep$.
    since $O\setminus E\subseteq O$, then $m_{*}(O\setminus E)<\ep$.
\end{pf}

\newpage
\begin{prop}
    ctbl union of measurable is measurable
\end{prop} \

\begin{pf}[source=Primary Source Material]
    just kinda repeat the above with $\ep/2^{j}$ for each $j$
\end{pf}

\begin{prop}
    closed sets are measurable
\end{prop} \

\begin{pf}[source=Primary Source Material]
    we can write $F=\bigcup_{k=1}^{\infty}F\cap B_{k}$ with $B_{k}$ the closed
    ball of radius $k$ at the origin, so suppose wlog $F$ cpt.

    let $\ep>0$. then there exists open $O$ with $F\subseteq O$ and
    $m_{*}(O)\leq m_{*}(F)+\ep$. then $O\setminus F$ open, and is a ctbl union
    of almost disjoint (closed) cubes:
    \begin{equation*}
        O\setminus F = \bigcup_{j=1}^{\infty}Q_{j}
    \end{equation*}
    for fixed $N, K=\bigcup_{j=1}^{N}Q_{j}$ cpt, so $d(K,F)>0$.
    since $(K\cup F)\subseteq O$, then:
    \begin{equation*}
        m_{*}(O)\geq m_{*}(F)+m_{*}(K)=m_{*}(F)+\sum_{j=1}^{N}m_{*}(Q_{j})
    \end{equation*}
    so $\sum m_{*}(Q_{j})\leq m_{*}(O)-m_{*}(F)<\ep$, and this holds as
    $N\sto\infty$.
    by subadditivity, $m_{*}(O\setminus F)\leq\sum m_{*}(Q_{j}) <\ep$ as needed.
\end{pf}

complement of measurable is measurable. proof is kinda dumb. lwk not a fan of
this books proofs lol. follows that ctbl intersections is measurable.

ctbl disjoint union is sum etc etc. these pfs r still mega weird

\begin{defn}
    if $E_{1},E_{2},\dots$ ctbl subsets of $R^{d}$ with $E_{k}\subseteq E_{k+1}$
    for all $k$ and $E=\bigcup_{k}E_{k}$, then we write $E_{k}\nearrow E$.

    similarly, if $E_{k+1}\subseteq E_{k}$ and $E=\bigcap_{k}E_{k}$, we write
    $E_{k}\searrow E$.
\end{defn}

\begin{prop}
    if $E_{k}\nearrow E$, or $E_{k}\searrow E$ and $m(E_{k})<\infty$ for some
    $k$, then $m(E)=\lim_{N\sto\infty}m(E_{N})$.
\end{prop}

pf: too lazy 2 copy

\begin{prop}
    $E\subseteq\bR^{d}$ measurable, $\ep>0$
    \begin{itemize}
        \item $\exists \, \trm{ open }E\subseteq O, m(O\setminus E)<\ep$
        \item $\exists \, \trm{ closed }E\subseteq F, m(F\setminus E)<\ep$
        \item $m(E)<\infty\ \implies \ \exists \, K\subseteq E \trm{ cpt },
            m(E\setminus K)<\ep$
        \item $m(E)<\infty\ \implies \ \exists \, F\bigcup_{j=1}^{n}Q_{j}$ closed
            cubes, $m(E\triangle F)<\ep$
    \end{itemize}
\end{prop}

\begin{pf}[source=Primary Source Material]
    1,2 easy.
    for 3, take closed $F\subseteq E$, and intersect w closed balls.
    for 4, take a cover and do some algebra or whatever just go look at the book
\end{pf}

