\subsection{Semidirect Product}
\lecdate{Lec 18 - Nov 8 (Week 9)}
Let's try semidirects properly this time.

Suppose $ N \ngrp G, H < G $ such that $ H \cap N = \set{e} $.
Then, we know that $ HN $ is a subgroup of $ G $.
Consider the set:
\begin{equation*}
    \set{(n, h) : n \in N, h \in H}
\end{equation*}
Then, this is a group under the following operation:
\begin{gather*}
    (n, h) \cdot (n', h') = nhn'h' = nhn'h^{-1}hh' = n\vphi_{h}(n')hh'
\end{gather*}
where $ \vphi_{h}(n') = hn'h^{-1} $ is conjugation by $ h $.

Clearly, the identity is given by $ (e, e) $.
What is the inverse?
%
% what is going on w the spacing here
\begin{align*}
    & (n, h)(\oline{n}, h^{-1}) = (e, e) \\
    \implies \ & (n\vphi_{h}(\oline{n}), hh^{-1}) = (e, e) \vsp
    \implies \ & n\vphi_{h}(\oline{n}) = e^{\phantom{eee}} \\
    \implies \ & \vphi_{h}(\oline{n}) = n^{-1} \\
    \implies \ & \oline{n} = h^{-1}n^{-1}h \\
    \implies \ & \oline{n} = \vphi_{h^{-1}}(n^{-1})
\end{align*}

The group we just constructed is called the semi-direct product of $ N $ and $ H $.

\newpage
\begin{defn}
    Suppose $ N \ngrp G $ and $ H < G $ such that $ H \cap N = \set{e} $.
    Then, the \textbf{semi-direct product} of $ N $ and $ H $ is:
    \begin{equation*}
        \set{(n, h) : n \in N, h \in H}
    \end{equation*}
    defined with the following operation:
    \begin{equation*}
        (n, h) \cdot (n', h') = nhn'h' = nhn'h^{-1}hh' = n\vphi_{h}(n')hh'
    \end{equation*}
    where $ \vphi_{h}(n') = hn'h^{-1} $ is conjugation by $ h $. \vsp
    %
    The identity is given by $ (e, e) $, and the inverse is given by:
    \begin{equation*}
        (n, h)^{-1} = (\vphi_{h^{-1}}(n^{-1}), h^{-1})
    \end{equation*}
    We denote the semi-direct product of $ N $ and $ H $ as $ N \rtimes_{\vphi} H $,
    or simply $ N \rtimes H $.
\end{defn}
We can also start with the groups $ N $ and $ H $ and
a homomorphism $ \vphi: H \rightarrow \Aut(N) $.

The same formulas construct a group which we denote the same way.
In this manner, we can define semidirect products between more general groups.

Note that if $ \vphi(h) = \trm{id} \in \Aut(N) $ for all $ h $,
then $ N \rtimes H \cong N \times H $.

In general, $ N \rtimes H $ contains:
\begin{gather*}
    \oline{N} = \set{(n, e) : n \in N} \cong N \\
    \oline{H} = \set{(e, h) : h \in H} \cong H
\end{gather*}
We also allow ourselves to call them $ N $ and $ H $ respectively. Note that:
\begin{equation*}
    N \cap H = \set{e} \qquad N \ngrp N \rtimes H \qquad N \rtimes H = NH
\end{equation*}

\begin{xmp}[source=Primary Source Material]
    Consider the following two groups:
    \begin{itemize}
        \item $ \SO_{3} \rtimes \bb{R}^{3} $ - the special Euclidean group
        \item $ \trm{O}_{3} \rtimes \bb{R}^{3} $ - the Euclidean group
    \end{itemize}
    These groups ``preserve" Euclidean geometry - the first represents rotations and translations
    in $ \bb{R}^{3} $, and the second represents rotations, reflections,
    and translations in $ \bb{R}^{3} $.
\end{xmp}

\begin{xmp}[source=Primary Source Material,title=Lorentz and Poincare group]
    Relativity operates in $ \bb{R}^{4} $. \vsp
    %
    The Lorentz group is the group of matrices that preserve the non-positive-definite inner
    product, $ \SO(3, 1) $. There are also translations, given by $ \bb{R}^{4} $. \vsp
    %
    So, the group $ \bb{R}^{4} \rtimes \SO(3, 1) $ is called the Poincar\'e group.
    It is the group of things that ``preserve" special relativity.
\end{xmp}

\begin{xmp}[source=Primary Source Material]
    Consider the set of $ 4 \times 4 $ matrices of the form:
    \begin{center}
        \begin{tabular}{ccc|c}
            $  $ & $  $ & $  $ & $ x $ \\
            $  $ & $ \Omega $ & $  $ & $ y $ \\
            $  $ & $  $ & $  $ & $ z $ \\
            \hline
            $ 0 $ & $ 0 $ & $ 0 $ & $ 1 $
        \end{tabular}
    \end{center}
    where $ \Omega \in \SO_{3} $. This is isomorphic to $ \bb{R}^{3} \rtimes \SO_{3} $. \vsp
    %
    Analogously for the Poincar\'e group, we have that:
    \begin{center}
        \begin{tabular}{cccc|c}
            $  $ & $  $ & $  $ & $  $ & $ x $ \\
            $  $ & $  $ & $  $ & $  $ & $ y $ \\
            $  $ & $  $ & $ \Omega $ & $  $ & $ z $ \\
            $  $ & $  $ & $  $ & $  $ & $ t $ \\
            \hline
            $ 0 $ & $ 0 $ & $ 0 $ & $ 0 $ & $ 1 $
        \end{tabular}
    \end{center}
\end{xmp}

\lecdate{Midterm - Nov 13 (Week 10)}
\lecdate{Lec 19 - Nov 15 (Week 10)}

If you have a direct product $ G = H \times K $,
then each copy of $ H $ and $ K $ is normal:
\begin{equation*}
    H \simeq H \times \set{e} \qquad K \simeq \set{e} \times K
\end{equation*}
Furthermore, $ H $ and $ K $ commute with each other.

In particular, the maps $ H \rightarrow \Aut(K) $ and $ K \rightarrow \Aut(H) $,
given by conjugation, are both trivial.

For a semi-direct product $ G = N \rtimes H $, we only have that $ N \ngrp G $.
In this case, the conjugate map $ H \rightarrow \Aut(N) $ is non-trivial.

Again we write $ N = N \times \set{e} $ and $ H = \set{e} \times H $.
Note that $ N \cap H = \set{e} $.

So each element $ g \in G $ can be written as $ g = (n, h) $ for some $ n \in N, h \in H $,
such that $ n $ and $ h $ are uniquely determined by $ g $.

\begin{xmp}[source=Primary Source Material]
    Let $ G = \bb{Z} $, and $ N = 2\bb{Z} \ngrp G $. Note $ N $ has no complement.
\end{xmp}

\begin{xmp}[source=Primary Source Material]
    Consider $ G = S_{3} $, with $ A_{3} \ngrp S_{3} $.
    Since $ \abs{G} = 6 $ and $ \abs{A_{3}} = 3 $, our complement must have order $ 2 $. \vsp
    %
    Indeed, we see that $ H = \la (12) \ra $ is a group of order $ 2 $.
    Clearly, $ H \cap A_{3} = \set{e} $.
    Note that $ H' = \la (13) \ra $ and $ H'' = \la (23) \ra $ also work.
\end{xmp}
