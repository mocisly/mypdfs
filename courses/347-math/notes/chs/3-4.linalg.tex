\subsection{Back to Linear Algebra}
Recalling some linear algebra:

Given a transformation $ T: V \rightarrow W $ over a field $ \bb{F} $,
We can associate with $ T $ a matrix that depends on the choice of basis for
$ V, W $. In the special case that $ V = W $ with the same basis, we get a
square matrix.

im not writing all this. we know what eigenvalues are lol

jcf mention?!?!?!?!

next time: rational canonical form

\lecdate{Lec 39 - Mar 12 (Week 21)}
(last friday was cancelled since joe got sick)

Consider $ R = \bb{F}[x] $ for some field $ \bb{F} $; $ R $ is a PID.
If $ A $ is an $ n\times n $ matrix, then there is a natural action of $ R $
on $ \bb{F}^{n} $ by letting $ x $ act as multiplication by $ A $.

If $ \frk{a} $ is an ideal in $ R $, then there exist ideals
$ \frk{a}_{1}, \dots, \frk{a}_{k} $ such that the action of $ R $ on
$ \bb{F}^{n} $ is isomorphic to:
\begin{equation*}
    \bb{F}[x]/\frk{a}_{1} \oplus \dots \oplus \bb{F}[x]/\frk{a}_{k} \oplus R^{r}
\end{equation*}
and $ \frk{a}_{1} \supseteq \frk{a}_{2} \supseteq \dots \supseteq \frk{a}_{k} $.
The $ \frk{a}_{i} $'s are the \textbf{invariant factors}.

\begin{defn}
    Suppose $ B $ is a square matrix with characteristic polynomial given by:
    \begin{equation*}
        c_{B}(x) = \det(xI - B) = x^{m}+b_{m-1}x^{m-1}+\dots+b_{1}x+b_{0}
    \end{equation*}
    Then the \textbf{companion matrix} to this polynomial is given by:
    \begin{equation*}
        C_{c_{B}} = \begin{pmatrix}
            0 & 0 & 0 & \dots & 0 & -b_{0} \\
            1 & 0 & 0 & \dots & 0 & -b_{1} \\
            0 & 1 & 0 & \dots & 0 & -b_{2} \\
            0 & 0 & 1 & \dots & 0 & -b_{3} \\
            \vdots & \vdots & \vdots & \ddots & \vdots & \vdots \\
            0 & 0 & 0 & \dots & 1 & -b_{m-1}
        \end{pmatrix}
    \end{equation*}
\end{defn}

Note that this has the same characteristic polynomial as $ B $, but in general
may not be similar to $ B $:
\begin{equation*}
    \begin{pmatrix}
        1 & 0 \\ 1 & 1
    \end{pmatrix}
    \qquad
    \begin{pmatrix}
        1 & 1 \\ 0 & 1
    \end{pmatrix}
\end{equation*}
The rational canonical form (RCF) will be a sum of blocks, each of which is a
companion matrix with characteristic equal to a generator of one of the
invariant factors. To make it unique, we assume these generators are monic
polynomials. But, this will only work if we can find the invariant factors.

There is a procedure for finding (monic) generators for the invariant factors:
given a matrix $ A $, write out $ xI - A $:
\begin{equation*}
    xI - A = \begin{pmatrix}
        x-a_{11} & -a_{12} & \dots & -a_{1n} \\
        -a_{21} & x-a_{22} & \dots & -a_{2n} \\
        \vdots & \vdots & \ddots & \vdots \\
        -a_{n1} & -a_{n2} & \dots & x-a_{nn}
    \end{pmatrix}
\end{equation*}
We will apply a kind of row and column reduction, with the goal of changing
$ xI - A $ to something that looks like:
\begin{equation*}
    \begin{pmatrix}
        1 &  &  &  &  & \\
        & \ddots & & & & \\
        & & 1 & & & \\
        & & & f_{1}(x) & & \\
        & & & & \ddots & \\
        & & & & & f_{k}(x) \\
    \end{pmatrix}
\end{equation*}
where $ f_{1}, \dots, f_{k} $ are monic generators of $ \frk{a}_{1}, \dots,
\frk{a}_{k} $ respectively.
Allowed operations are:
\begin{itemize}
    \item Multiplying a row or column by a unit $ c $, i.e. a nonzero scalar.
    \item Switch two rows or columns.
    \item If $ g(x) \in \bb{F}[x] $, add $ g(x)\cdot R_{i} $ to $ R_{j} $.
\end{itemize}
where $ R_{i}, R_{j} $ are any two rows (this holds analogously for columns).

\begin{xmp}[source=Primary Source Material]
    Let $ A = \begin{pmatrix}
        1 & 0 \\ 0 & 2
    \end{pmatrix} $. Then:
    \begin{center}
        \begin{tabular}{CCCC}
            & \begin{pmatrix}
                x-1 & 0 \\ 0 & x-2
            \end{pmatrix} & \xrightarrow{R_{1} + R_{2}}
            & \begin{pmatrix}
                x-1 & x-2 \\ 0 & x-2
            \end{pmatrix} \\ & & & \\
            \xrightarrow{C_{2}-C_{1}}
            & \begin{pmatrix}
                x-1 & -1 \\ 0 & x-2
            \end{pmatrix}
            & \xrightarrow{C_{1} \leftrightarrow C_{2}}
            & \begin{pmatrix}
                -1 & x-1 \\ x-2 & 0
            \end{pmatrix} \\ & & & \\
            \xrightarrow{-1 \cdot R_{1}}
            & \begin{pmatrix}
                1 & 1-x \\ x-2 & 0
            \end{pmatrix}
            & \xrightarrow{R_{2}-(x-2)R_{1}}
            & \begin{pmatrix}
                1 & 1-x \\ 0 & (x-1)(x-2)
            \end{pmatrix} \\ & & & \\
            \xrightarrow{C_{2}-(1-x)C_{1}}
            & \begin{pmatrix}
                1 & 0 \\ 0 & (x-1)(x-2)
            \end{pmatrix} & &
        \end{tabular}
    \end{center}
    So for $ A = \begin{pmatrix}
        1 & 0 \\ 0 & 2
    \end{pmatrix} $, there is only one invariant factor, and it is
    $ (x-1)(x-2) = x^{2}-3x+2 $. So:
    \begin{equation*}
        \trm{RCF}(A) \ = \
        \begin{pmatrix}
            0 & -2 \\
            1 & 3
        \end{pmatrix}
    \end{equation*}
\end{xmp}

\begin{xmp}[source=Primary Source Material]
    time to put my latex skills to the test. \vsp
    %
    Suppose $ A = \begin{pmatrix}
        -4 & 2 & 1 \\ -7 & 5 & 1 \\ -11 & 2 & 4
    \end{pmatrix} $. Then, $ xI - A = \begin{pmatrix}
        x+4 & -2 & -1 \\ 7 & x-5 & -1 \\ 11 & -2 & x-4
    \end{pmatrix} $, and:
    \begin{center}
        \begin{tabular}{CCCC}
            \xrightarrow{-1 \cdot C_3}
            & \begin{pmatrix}
                x+4 & -2 & 1 \\
                7 & x-5 & 1 \\
                11 & -2 & 4-x
            \end{pmatrix}
            & \xrightarrow{C_1 \leftrightarrow C_3}
            & \begin{pmatrix}
                1 & -2 & x+4 \\
                1 & x-5 & 7 \\
                4-x & -2 & 11
            \end{pmatrix} \\ & & & \\
            \xrightarrow{R_2 - R_1}
            & \begin{pmatrix}
                1 & -2 & x+4 \\
                0 & x-3 & 3-x \\
                4-x & -2 & 11
            \end{pmatrix}
            & \xrightarrow {C_2 + 2C_1}
            & \begin{pmatrix}
                1 & 0 & x+4 \\
                0 & x-3 & 3-x \\
                4-x & 6-2x & 11
            \end{pmatrix} \\ & & & \\
            \xrightarrow{C_3 - (x+4)C_1}
            & \begin{pmatrix}
                1 & 0 & 0 \\
                0 & x-3 & 3-x \\
                4-x & 6-2x & x^2-5
            \end{pmatrix}
            & \xrightarrow{R_3 - (4-x)R_1}
            & \begin{pmatrix}
                1 & 0 & 0 \\
                0 & x-3 & 3-x \\
                0 & 6-2x & x^2-5
            \end{pmatrix} \\ & & & \\
            \xrightarrow{C_3 + C_2}
            & \begin{pmatrix}
                1 & 0 & 0 \\
                0 & x-3 & 0 \\
                0 & 6-2x & x^2-2x+1
            \end{pmatrix}
            & \xrightarrow{R_3+2R_2}
            & \begin{pmatrix}
                1 & 0 & 0 \\
                0 & x-3 & 0 \\
                0 & 0 & x^2-2x+1
            \end{pmatrix}
        \end{tabular}
    \end{center}
    This looks like the answer, however $ x-3 $ does not divide $ x^{2}-2x+1 $,
    so we're not done. How can we get a 1 into the centre position? \vsp
    %
    We \textit{could} do it - it's not impossible.
    But, observing that the final result has to be of the form
    $ \begin{pmatrix}
        1 & 0 & 0 \\ 0 & 1 & 0 \\ 0 & 0 & c_{A}
    \end{pmatrix} $ and $ c_{A} = (x-3)(x-1)^{2} $,
    we can just write down the final result:
    \begin{equation*}
        \begin{pmatrix}
            1 & 0 & 0 \\
            0 & 1 & 0 \\
            0 & 0 & (x-3)(x-1)^{2}
        \end{pmatrix} \qquad
        (x-3)(x-1)^{2} = x^{3}-5x^{2}+7x-3
    \end{equation*}
    So the RCF is given by:
    \begin{equation*}
        \trm{RCF}_{A} = \begin{pmatrix}
            0 & 0 & 5 \\
            1 & 0 & -7 \\
            0 & 1 & 3
        \end{pmatrix}
    \end{equation*}
\end{xmp}
In the above example, suppose we got:
$ \begin{pmatrix}
    1 & & \\
      & x-1 & \\
      & & (x-1)(x-3)
\end{pmatrix} $ where $ (x-1)(x-3) = x^{2}-4x+4 $.
Then:
\begin{equation*}
    \trm{RCF}_{A} = \begin{bmatrix}
        1 & 0 & 0 \\ 0 & 0 & -3 \\ 0 & 1 & 4
    \end{bmatrix}
\end{equation*}
Clearly, these are not conjugate.
What are the JCFs corresponding to each?

In the second case, the corresponding JCF is in fact diagonal, and is given by
$ \begin{pmatrix}
    1 & & \\ & 1 & \\ & & 3
\end{pmatrix} $.
For the original one, however, we have a cubic, and so the JCF is given by:
$ \begin{pmatrix}
    3 & & \\ & 1 & 1 \\ & & 1
\end{pmatrix} $.
This has two blocks in JCF, and one block in RCF.

We can ask ourselves about the annihilators. Notice that:
\begin{equation*}
    \begin{pmatrix}
        1 & 1 \\ 0 & 1
    \end{pmatrix} - I =
    \begin{pmatrix}
        0 & 1 \\ 0 & 0
    \end{pmatrix}
    \qquad
    \begin{pmatrix}
        0 & 1 \\ 0 & 0
    \end{pmatrix}^{2} = 0
\end{equation*}
So $ x-1 $ doesn't kill it, but $ (x-1)^{2} $ does.
If you have a Jordan block of size $ k\times k $, it needs $ (x-\lambda)^{k} $
in the annihilator.

\begin{thm}
    Two matrices $ A $ and $ B $ are similar if and only if they have the
    same $ RCF $.
\end{thm}

\begin{pf}[source=Primary Source Material]
    essentially follows from uniqueness of invariant factors .....
\end{pf}

\begin{exr}[source=Primary Source Material]
    Find a quiet place and convince yourself that the procedure works.
\end{exr}

\begin{xmp}[source=Primary Source Material]
    Let $ A = \begin{pmatrix}
        0 & 1 \\ -1 & 0
    \end{pmatrix} $. Note $ c_{A}(x) = x^{2}+1 $, so $ A $ has eigenvalues
    $ \pm i $. \vsp
    %
    Over $ \bb{R} $, we see that $ A $ does not have a JCF.
    However, over $ \bb{C} $, it is indeed diagonalizable:
    \begin{equation*}
        \trm{JCF}_{\bb{C}} = \begin{pmatrix}
            i & 0 \\ 0 & -i
        \end{pmatrix}
    \end{equation*}
    But notice that:
    \begin{equation*}
        \trm{RCF} = \begin{pmatrix}
            0 & -1 \\ 1 & 0
        \end{pmatrix}
    \end{equation*}
    Interestingly, this is the RCF over $ \bb{R} $ \textit{or} $ \bb{C} $ - or
    any field of characteristic 0, for that matter.
\end{xmp}
fun fact: if you go through the entire process we just did, but isolate only the
column operations (ignoring the row ops), write them as elementary matrices, and
multiply them together, then multiply the resulting matrix by the original, you
get the RCF. however, this is ``much worse", which is why joe didn't mention it
beforehand. oh okay its Storytime With Joe\texttrademark $ \ $ now
