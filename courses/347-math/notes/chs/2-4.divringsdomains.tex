\subsection{Division Rings and Domains}

If we start with $ \bb{Z} $, we can't divide, so we'd like to ``enlarge" $ \bb{Z} $ into something
with division. We construct $ \bb{Q} = \set{\frac{a}{b} : a,b \in \bb{Z}, b \neq 0} $ and define
addition and multiplication.

Problem: $ \dfrac{a}{b} $ is not unique. Thus, we need to check that $ +, \cdot $ are well-defined.

Doing the same construction with any ring $ R $ instead of $ \bb{Z} $ yields what we call the
\textbf{Ring of Fractions}.

Given a commutative unital ring $ R $, consider ordered pairs $ (a, b) $ where $ a, b \in R $ and
$ b $ is not a zero divisor. We want to define addition and multiplication on this set.
\begin{equation*}
    (a,b)\cdot(c,d) = (ac,bd) \qquad (a,b)+(c,d) = (ad+bc, bd)
\end{equation*}
Note that the product of two non zero divisors is a non zero divisor.

Next, we define an equivalence relation:
\begin{equation*}
    (a,b)\sim(c,d) \iff ad = bc
\end{equation*}
Of course, we need to check this is preserved by $ +, \cdot $.

Suppose $ (a,b) \sim (a',b') $ and $ (c,d) \sim (c',d') $; that is $ ab' = a'b $ and $ cd' = c'd $.
We want to check if $ (a,b)\cdot(c,d) \sim (a',b')\cdot(c',d') $. Indeed:
\begin{equation*}
    (a,b)\cdot(c,d) = (ac,bd) \quad (a',b')\cdot(c',d') = (a'c',b'd') \qquad
    acb'd' = a'c'bd = ac'b'd = acb'd'
\end{equation*}
Addition is similar, but worse, so we skip it for now.
This gives us the definition (construction) of the ring of fractions.

\begin{defn}
    Given a commutative unital ring, the \textbf{ring of fractions} is the set given by
    $ R^{2}/\sim $ equipped with addition and multiplication, where $ \sim, +, \cdot $ are defined
    as above.
\end{defn}
Clearly for $ \bb{Z} $, we get $ \bb{Q} $. For $ R = \bb{F}[x] $, then we get what is known as the
rational functions:
\begin{equation*}
    \set{\frac{f(x)}{g(x)} : f,g \in R}
\end{equation*}
Note that strictly speaking, these are not functions on $ \bb{F} $, as $ g $ is undefined on at
least 0. Restricting the domain does turn these into functions.

Note, however, that the points at which $ g $ is 0 might be different for an equivalent
representation.

\lecdate{Lec 25 - Jan 10 (Week 13)}

Suppose $ R $ is a commutative, not necessarily unital ring.
Suppose $ D $ is a subset of $ R $ that does not contain any zero divisors,
and which is closed under multiplication.
The elements of $ D $ are thus allowed to be denominators in the ring of fractions.
If $ R $ has no zero divisors, then an obvious choice is $ D = R \setminus \set{0} $.

Observe that there is a ring $ Q $ which contains an isomorphic copy of $ R $ such that
every element of $ D $ in $ R $ has an inverse in $ Q $.
Moreover, $ Q $ is the smallest ring with these properties.

If $ R $ has no zero divisors and we choose the obvious choice of $ D $, then
$ Q $ is a field, called the \textbf{field of fractions}, or sometimes the \textbf{quotient field}.

\begin{xmp}[source=Primary Source Material]
    Let $ R = \bb{Z} $. Choose a prime number $ p $, and let $ D $ be defined as:
    \begin{equation*}
        D = \set{d \in R : p \nmid d} = \set{d \in R : \gcd(p, d) = 1}
    \end{equation*}
    The ordered pairs we consider are $ (n, d) $, where $ p \nmid d $.
    In particular, every prime $ q \neq p $ is in $ D $, and so $ \dfrac{1}{q} \in D $.
    So in the ring of fractions $ Q $, every prime other than $ p $ is invertible.
\end{xmp}

\begin{exr}[source=Primary Source Material]
    Not too difficult: Show that in the above, $ (p) $ is the unique prime ideal in $ Q $.
\end{exr}

\begin{defn}
    A ring with a single prime ideal is called a \textbf{local ring}.
\end{defn}

In $ \bb{C}[x_{1}, \dots, x_{n}] $, the maximal ideals are of the form
\begin{equation*}
    I_{a} = \set{f : f(a) = 0} \qquad a \in \bb{C}
\end{equation*}
In particular, they are the \textit{only} maximal ideals.
Thus, we observe that maximal ideals represent points in $ \bb{C}^{n} $.
Vaguely: a local ring focuses attention on what happens ``near" a point $ a $, ignoring the
global behaviour. (This is the start of algebraic geometry.)

\begin{lm}
    Suppose $ \gcd(M, N) = 1 $ for $ M, N \in \bb{Z}^{>0} $, and suppose $ a, b \in \bb{Z} $.
    Then there exists a unique $ n \in \bb{Z}_{MN} $ such that:
    \begin{equation*}
        n \equiv a \mod M \qquad n \equiv b \mod N
    \end{equation*}
\end{lm}

This is a familiar theorem that's not hard to show. We expand this now to our rings.

\begin{defn}
    Let $ R $ be a commutative unital ring.
    Two ideals $ A, B \subseteq R $ are \textbf{comaximal} if:
    \begin{equation*}
        A + B = R \quad \trm{i.e.} \quad A + B = (1)
    \end{equation*}
    This generalizes the idea of coprimality in $ \bb{Z} $.
\end{defn}

Suppose $ A_{1}, \dots, A_{k} $ are pairwise comaximal ideals in $ R $.
Consider the map $ \vphi : R \rightarrow R/A_{1} \times \dots \times R/A_{k} $ given by:
\begin{equation*}
    \vphi(r) = (r + A_{1}, \dots, r + A_{k})
\end{equation*}
Note that $ \ker(\vphi) = A_{1} \cap \dots \cap A_{k} $.
Furthermore, the reverse inclusion $ (\supseteq) $ is always true.
\begin{thm}[title=Chinese Remainder Theorem(?)]
    In the above scenario, we have that:
    \begin{equation*}
        A_{1} \cap \dots \cap A_{k} = A_{1}\cdots A_{k}
    \end{equation*}
\end{thm}

Recall that the product of ideals $ I, J $ is defined as:
\begin{equation*}
    IJ = \set{i_{1}j_{1} + i_{2}j_{2} + \dots + i_{t}j_{t} : i_{k} \in I, j_{k} \in J}
\end{equation*}

\begin{pf}[source=Primary Source Material]
    We prove $ A \cap B = AB $ if $ A + B = R $, as the rest follows from induction. \vsp
    The key is that:
    \begin{equation*}
        \vphi(a + b) = \vphi(1) = 1 \quad \vphi(a) = (1, 0) \in R/A + R/B \quad \vphi(b) = (0, 1)
    \end{equation*}
\end{pf} \

\lecdate{Lec 26 - Jan 15 (Week 14)}

Consider 26180 and 80262. What is their $ \gcd $?
We can use the Euclidean algorithm to find it:
\begin{equation*}
    80262 = 26180 \cdot 3 + 1722
\end{equation*}
Notice that if $ d \mid 26180 $ and $ d \mid 1722 $, then $ d \mid 80262 $. So:
\begin{equation*}
    \gcd(80262, 26180) = \gcd(26180, 1722)
\end{equation*}
We can repeat this process to deduce that $ \gcd(80262, 26180) = 14 $.

\begin{defn}
    An integral domain $ R $ is called a \textbf{Euclidean domain} if it has a ``norm" $ N $
    such that:
    \begin{itemize}
        \item $ N:R\rightarrow\bb{Z}^{\geq0} $
        \item $ N(0) = 0 $
        \item Given $ a, b \in R $, we can write a = bq + r with $ N(a) < N(b) $, unless $ r=0 $.
    \end{itemize}
\end{defn}
Note that a given ring $ R $ may have many norms, and some may even have a Euclidean algorithm,
but the $ \gcd $ you end up with will be the same.

\begin{defn}
    An integral domain $ R $ is called a \textbf{Principal Integral Domain (PID)} if every
    ideal is a principal ideal.
\end{defn}

\begin{thm}
    Every Euclidean domain is a PID.
\end{thm}

\begin{pf}[source=Primary Source Material]
    Sps $ R $ is Euclidean with $ I $ as some ideal.
    Choose $ d \in I $ such that $ N(d) $ is minimal. \vsp
    %
    Then, given any $ a \in I $, we have that $ a = dq + r $. Since $ a, dq \in I $, then
    we must have $ r \in I $ and $ N(r) < N(d) $. This gives a contradiction, so we must have
    that $ a = dq $.
\end{pf}

\begin{xmp}[source=Primary Source Material]
    $ \bb{F}[x] $ is a Euclidean domain with norm given by the degree. Thus, it is a PID.
\end{xmp}
Beware:
\begin{itemize}
    \item $ \bb{Z}[x] $ is \textit{not} a PID; consider $ (x, 5) $.
    \item $ \bb{F}[x, y] $ is \textit{not} a PID; consider $ (x, y) $.
    \item Similarly, $ \bb{F}[x_{1}, \dots, x_{n}] $ is \textit{not} a PID.
\end{itemize}

\begin{thm}
    If $ R $ is an integral domain and $ R[x] $ is a PID, then $ R $ is a field.
\end{thm}

\begin{pf}[source=Primary Source Material]
    Take $ R[x]/(x) \simeq R $ by FIT. Since $ R $ is an integral domain, $ (x) $ is prime.
    Is $ (x) $ maximal? \vsp
    %
    Suppose not, and suppose $ (x) \subsetneq (f(x)) $ for some $ f(x) $.
    In particular, $ x \in (f(x)) $, and so $ x = f(x)g(x) $. Thus, $ f, g $ must be of degrees
    1 and 0. \vsp
    %
    If $ f $ has degree 0, it must be constant and non-zero. Thus:
    \begin{equation*}
        g(x) = a+bx \ \implies \ f(x)g(x) = c(a+bx) \ \implies \ ca + cbx
    \end{equation*}
    Thus, we must have that $ a = 0, cb = 1 $, and so $ f(x) = 1 $. But then it follows that
    $ (f(x)) = (1) = R[x] $, which is not a proper subset. \npgh

    Now suppose $ g(x) $ has degree 0, the degree of $ f(x) $ is 1.
    Then, $ g(x) = c \neq 0 $ and $ f = a+bx $. Similarly as before:
    \begin{equation*}
        f(x)g(x) \ \implies \ f(x)g(x) = x \ \implies \  ca + cbx \ \implies \ a = 0, c \trm{ unit}
        \ \implies \ (f(x)) = (bx) \ \implies \ x
    \end{equation*}
    Hence $ (x) $ is indeed maximal, and so $ R[x]/(x) $ is a field.
\end{pf}

\begin{defn}
    An integral domain $ R $ is a \textbf{Dedekind domain (DD)} if, given any ideal $ I $, it
    is possible to find prime ideals $ p_{1}, \dots, p_{k} $ and natural numbers
    $ r_{1}, \dots, r_{k} $ such that each $ p_{i} $ is distinct (non-isomorphic), and:
    \begin{equation*}
        I = p_{1}^{r_{1}}\dots p_{k}^{r_{k}}
    \end{equation*}
    Furthermore, this decomposition is unique up to order.
\end{defn}

\begin{defn}
    An integral domain $ R $ is a \textbf{Unique Factorization domain (UFD)} if, given $ a \in R $
    such that $ a \neq 0 $, it can be written as:
    \begin{equation*}
        a = p_{1}^{r_{1}}\dots p_{k}^{r_{k}}
    \end{equation*}
    where the $ p_{i} $'s are irreducible elements, and each $ p_{i}, p_{j} $ are distinct even
    up to multiplication by units. This decomposition is also unique up to ordering.
\end{defn}

Between the different domains, we can describe them as such:
\begin{equation*}
    \trm{Euclidean domain } \subseteq
    \trm{ Principal Ideal domain } \subseteq
    \trm{ Unique Factorization domain } \subseteq
    \trm{ Integral domain}
\end{equation*}
Also note that D\&F discusses Hasse norms. Having a Hasse norm is equivalent to being a PID.

\begin{xmp}[source=Primary Source Material,title=Lagrange Interpolation]
    Given points $ a_{1}, \dots, a_{n}, b_{1}, \dots, b_{n} $, with each $ a_{i} $ distinct,
    is it possible to find a polynomial $ f $ with degree less than $ n $ such that
    $ f(a_{i}) = b_{i} $ for each $ i $? \vsp
    %
    In $ \bb{F}[x], I_{i} = (x - a_{i}) $ is the ideal of all polynomials vanishing at $ a_{i} $.
    Set:
    \begin{equation*}
        I = I_{1}\cdots I_{n} = I_{1} \cap \cdots \cap I_{n}
    \end{equation*}
    Then, a polynomial with the specified values amounts to a point in $ \bb{F}[x]/I $.
    CRT says that this is determined uniquely by a point in each $ \bb{F}[x]/I_{i} $.
\end{xmp}

\lecdate{Lec 27 - Jan 17 (Week 14)}

Recall that in an integral domain, an element $ r $ being prime means that:
\begin{equation*}
    r \mid ab \implies r \mid a \trm{ or } r \mid b
\end{equation*}

\begin{defn}
    An element $ r $ is \textbf{irreducible} if whenever $ r=ab $, either $ a $ or $ b $ is
    a unit.
\end{defn}
Notice that in $ \bb{Z} $, $ p $ is prime iff $ p $ is irreducible.

\begin{xmp}[source=Primary Source Material]
    Let $ R = \bb{Z}[\sqrt{-5}] $. Then, there is a norm on $ R $ given by:
    \begin{equation*}
        N(a+b\sqrt{-5}) \ = \ a^{2}+5b^{2} \ = \ (a+b\sqrt{-5})(a-b\sqrt{-5})
    \end{equation*}
    It is easy to see that $ N $ is multiplicative: $ N(\alpha\beta) = N(\alpha)N(\beta) $. \vsp
    %
    Consider $ 3 \in R $. Notice:
    \begin{equation*}
        (2+\sqrt{-5})(2-\sqrt{-5}) \ = \ 4+5 \ = \ 9
    \end{equation*}
    In particular:
    \begin{equation*}
        3 \mid (2+\sqrt{-5})(2-\sqrt{-5}) \qquad
        3 \nmid (2 \pm \sqrt{-5}) \qquad
        3(x+y\sqrt{-5}) = 3x + 3y\sqrt{-5}
    \end{equation*}
    So we see that in this case, $ 3 $ is not prime.
    However, we claim that $ 3 $ is irreducible. \vsp
    %
    Indeed, suppose that $ 3=\alpha(a+b\sqrt{-5}) $. Then:
    \begin{equation*}
        9 = N(\alpha)(a^{2}+5b^{2})
    \end{equation*}
    Since $ N(\alpha) \in \bb{Z} $, then $ a^{2}+5b^{2} $ must divide $ 9 $, so it must be
    $ 1, 3, $ or $ 9 $.
    \begin{itemize}
        \item If $ a^{2}+5b^{2} = 1 $, then $ b=0, a= \pm1 $, so $ a+b\sqrt{-5} $ is a unit.
        \item Similarly, if $ a^{2}+5b^{2}=9 $, then $ N(\alpha) = 1 $, so $ \alpha $ is a unit.
        \item Suppose $ a^{2}+5b^{2} = 3 $. Then, $ b=0 $ and $ a^{2}=3 $, which isn't possible.
            Indeed, if $ 3=\alpha\beta $, we can't have $ N(\alpha) = 3 $, so must have either
            $ N(\alpha)=1 $ or $ N(\beta)=1 $. Thus, one factor is a unit.
    \end{itemize}
\end{xmp}

\begin{thm}
    In an integral domain, $ r \trm{ prime } \implies r \trm{ irreducible} $.
\end{thm}

\begin{pf}[source=Primary Source Material]
    Suppose $ p $ is prime. If $ p=ab $, then (wlog) $ p\mid a $. Thus, $ a=px $ for some $ x $.
    Thus:
    \begin{equation*}
        p=ab=pxb\ \implies \ 1=xb
    \end{equation*}
    So $ b $ is a unit.
\end{pf}

\begin{thm}
    If $ R $ is a PID, then $ r \trm{ prime } \iff r \trm{ irreducible} $.
\end{thm}

\begin{pf}[source=Primary Source Material]
    Suppose $ p $ is irreducible. We show that $ (p) $ is maximal. \vsp
    %
    Suppose $ (p) \subseteq I = (q) $. Then $ p \in I \implies p=qr $, so $ r $ is a unit.
    Thus, $ q=pr^{-1} $, so $ (p) = (q) $.
\end{pf}
So for PID's, prime is equivalent to irreducible. How nice!

Suppose $ R $ is a UFD, with:
\begin{equation*}
    x = up_{1}^{r_{1}}\cdots p_{k}^{r_{k}} \qquad
    y = u'q_{1}^{s_{1}}\cdots q_{\ell}^{s_{\ell}}
\end{equation*}
What is $ \gcd(x,y) $? We can reorganize the $ p $'s and $ q $'s such that:
\begin{equation*}
    p_{1}=q_{1} \quad p_{2}=q_{2} \quad \dots \quad p_{j}=q_{j}
\end{equation*}
But the remaining $ p_{i} $'s are not among the $ q_{i} $'s and vice versa. Then:
\begin{equation*}
    \gcd(x,y) = p_{1}^{\min(r_{1},s_{1})}p_{2}^{\min(r_{2},s_{2})}\cdots p_{j}^{\min(r_{j},s_{j})}
\end{equation*}

\begin{thm}
    In a UFD, primes are equivalent to irreducibles.
\end{thm}
The proof is long; see D\&F.

\lecdate{Lec 29 - Jan 22 (Week 15)}

Recall quadradic fields $ \bb{Q}(\sqrt{D}) $, where $ D $ is a square-free integer.
The ring of integers $ O $ in $ \bb{Q}(\sqrt{D}) $ is given by:
\begin{equation*}
    O =
    \begin{cases}
        \bb{Z}[\sqrt{D}] & D \equiv 2 \trm{ or } 3 \mod 4 \\
        \bb{Z}[\frac{1+\sqrt{D}}{2}] & D \equiv 1 \mod 4
    \end{cases}
\end{equation*}
We write $ \omega = \sqrt{D} \trm{ or } \frac{1+\sqrt{D}}{2} $ accordingly, so:
\begin{equation*}
    O = \set{a+b\omega : a,b \in \bb{Z}}
\end{equation*}
Here, we have a (field) norm $ N: O \rightarrow \bb{Z} $ given as:
\begin{equation*}
    N(a+b\omega) = (a+b\omega)(a-b\oline{\omega})
\end{equation*}
where $ \oline{\omega} = -\sqrt{D} \trm{ or } \frac{1-\sqrt{D}}{2} $.
When $ \omega = \sqrt{D} $, we have that:
\begin{equation*}
    N(a+b\omega) = N(a+b\sqrt{D}) = (a+b\sqrt{D})(a-b\sqrt{D}) = a^{2}-b^{2}D
\end{equation*}
If $ \alpha $ is a unit, then there exists $ \beta \in O $ such that $ \alpha\beta = 1 $.
Since $ N(\alpha\beta) = N(\alpha)N(\beta) $, then $ N(1) = 1 $.
Since $ N(\alpha), N(\beta) \in \bb{Z} $, then we must have that $ N(\alpha) = N(\beta) = \pm1 $.

The converse also happens to be true; if $ N(\alpha) = \pm1 $, then $ \alpha $ is a unit:
\begin{equation*}
    1 = N(\alpha) = \alpha\bar{\alpha} \in O \ \implies \ \bar{\alpha} = \alpha^{-1}
\end{equation*}
So $ \alpha \in O $ is a unit iff $ N(\alpha) = \pm1 $.

\begin{thm}
    Suppose $ \pi \in O $ is a prime element, so $ (\pi) $ is a prime ideal.
    Then, $ (\pi) \cap \bb{Z} $ is a prime ideal.
\end{thm}

\begin{pf}[source=Primary Source Material]
    Let $ a, b \in \bb{Z} $ such that $ ab \in (\pi) \cap \bb{Z} $.
    Then, $ a, b $ can be regarded as elements of $ O $, and $ ab \in (\pi) $, so
    $ a \in (\pi) $ or $ b \in (\pi) $. Hence $ a \in (\pi) \cap \bb{Z} $ or
    $ b \in (\pi) \cap \bb{Z} $, and thus it is a prime ideal.
    We'll write $ (\pi) \cap \bb{Z} = (p) $.
\end{pf}

Suppose $ p = \pi_{1}\pi_{2} \in O $. Then $ N(p) = N(\pi_{1})N(\pi_{2}) = p^{2} $.
Then, we must have one of the following:
\begin{gather*}
    N(\pi_{1})=\pm1 \quad , \quad N(\pi_{2})=\pm p^{2} \\
    N(\pi_{1})=N(\pi_{2})= \pm p
\end{gather*}
If $ N(\pi_{1}) = \pm 1 $, then $ \pi_{1} $ is a unit, and so $ p $ is irreducible as an element
of $ O $. However, if we have that $ N(\pi_{1}) = N(\pi_{2}) = \pm p $, then $ p=\pi_{1}\pi_{2} $
factors as the product of two irreducibles in $ O $.

Special case: in the Gaussian integers ($ D = -1 $), we have that
$ O = \bb{Z}[\sqrt{-1}] = \bb{Z}[i] $. Here:
\begin{equation*}
    N(a+bi) = (a+bi)(a-bi) = a^{2}+b^{2}
\end{equation*}
It's easy to see that this works as a ``norm" for the Euclidean algorithm, and so $ \bb{Z}[i] $ is
a PID.

In $ \bb{Z}[i] $, notice that not every norm is achievable:
\begin{equation*}
    \begin{tabular}{CCCC}
        1 = 1^{2} + 0^{2} & 2 = 1^{2} + 1^{2} & 3 = \eset & 4 = 2^{2} + 0^{2} \\
        5 = 2^{2} + 1^{2} & 6 = \eset & 7 = \eset & 8 = 2^{2} + 2^{2} \\
                        \ & 9 = 3^{2} + 0^{2} & 10 = 3^{2} + 1^{2} & \
    \end{tabular}
\end{equation*}
Which integers can be written as a sum of two squares? This was actually proven by Fermat.
As a hint, start with the following: which primes $ p $ can be written as the sum of two squares?
