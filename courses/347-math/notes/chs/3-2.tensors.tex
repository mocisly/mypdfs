\newpage %bruh
\subsection{Tensor Products}
\lecdate{Lec 36 - Feb 26 (Week 19)}

It's time for tensor products.

Recall that if $ M, N $ are $ R $-modules, then:
\begin{equation*}
    M \oplus N \simeq M \times N = \set{(m, n) : m \in M, n \in N}
\end{equation*}
If $ \vphi : M \times N \rightarrow K $ is a homomorphism, that is, an $ R $-linear map,
then:
\begin{gather*}
    \vphi((a, b) + (c, d)) = \vphi(a, b) + \vphi(c, d) \\
    \vphi((ra, rb)) = \vphi(r(a, b)) = r\vphi(a, b)
\end{gather*}
We're going to discuss a \textit{different} kind of map.

\begin{defn}
    An $ R $\textbf{-bilinear map} is a map $ \vphi $ which is linear on each
    component, separately:
    \begin{gather*}
        \vphi((ra, b)) = \vphi(r(a,b)) = \vphi((a,rb)) = r\vphi(a, b) \\
        \vphi((a+b), c) = \vphi(a,c) + \vphi(b, c) \\
        \vphi(a, (b+c)) = \vphi(a,b) + \vphi(a,c)
    \end{gather*}
    Notice that in this case, we have that:
    \begin{equation*}
        \vphi((ra, rb)) = r\vphi((a, rb)) = r^{2}\vphi(a, b)
    \end{equation*}
\end{defn}
A \textit{tensor product} is a convenient way of keeping track of bilinear maps.

Let $ R $ be a commutative unital ring.
Suppose $ \vphi: M \times N \rightarrow K $ is an $ R $-bilinear map. Thus:
\begin{equation*}
    \vphi((ra, b)) = r\vphi(a, b)
\end{equation*}
We can't just take any map, we need to define things carefully.

Start with $ M \times N $ as an $ R $-module.
Take all the elements that correspond to the above relations:
\begin{gather*}
    (ra, b) - r(a, b) \\
    (a, rb) - r(a, b) \\
    (a+b,c) - (a,c) - (b,c) \\
    (a,b+c) - (a,b) - (a,c)
\end{gather*}
Clearly, these are all taken to $ 0 $ by $ \vphi $.
Take the submodule generated by these relations; call it $ L $.
We consider $ (M \times N)/L $, an $ R $-module, denoted $ M \otimes_{R}N $;
this is the tensor product.

Note that $ \vphi $ gives a well-defined map on $ M \otimes_{R} N $ since
it kills everything in $ L $; for this map, we'll use the following function:
\begin{equation*}
    \Phi: M \otimes_{R} N \rightarrow K
\end{equation*}
We now get the following diagram:

[diagram]

This is kind of like FIT; we quotient out by a certian thing and get a map
from the quotient space to $ K $. Notice that the same quotient
$ M \otimes_{R} N $ will work for \textit{any} bilinear map $ \vphi $;
they must ``factor through" $ M \otimes_{R} N $.

Digression: D\&F \textit{doesn't} assume that $ R $ is commutative. So instead,
they have to assume that $ M $ is a \textit{right} $ R $-module, and that
$ N $ is a \textit{left} $ R $-module. A bilinear map $ \vphi $ now satisfies:
\begin{equation*}
    \vphi((mr, n)) = \vphi((m, rn))
\end{equation*}
as well. Then, everything else is defined similarly, though very complicated.

\begin{thm}
    The construction above defines an $ R $-module $ M \otimes_{R} N $. \vsp
    %
    Given any $ R $-bilinear map $ \vphi : M \times N \rightarrow K $,
    there is a unique $ R $-module map, given by:
    \begin{equation*}
        \Phi: M \otimes_{R} N \rightarrow K
    \end{equation*}
    such that the below diagram commutes:

    [diagram]

    Furthermore, if $ T $ is another $ R $-module with the same property, then:

    [diagram]

    In this case, $ T \simeq M \otimes_{R} N $, and $ \Phi' $
    corresponds to $ \Phi $.
\end{thm}

Note that this tells us that tensor products count bilinear maps.

In linear algebra, let $ \bb{F} $ be a field and $ U, V, W $ vector spaces.
Then, we can think about $ U \otimes_{\bb{F}} V $.

As it turns out, a basis for $ U \otimes_{\bb{F}} V $ is given by:
\begin{equation*}
    \set{e_{i} \otimes f_{j}}
\end{equation*}
where $ e_{i} \otimes f_{j} $ is the image of $ (e_{i}, f_{j}) $ in
$ U \otimes_{\bb{F}} V $, with $ \set{e_{i}}, \set{f_{j}} $ as bases for
$ U $ and $ V $ respectively. Thus, we see that:
\begin{equation*}
    \dim(U\otimes_{\bb{F}}V) = \dim(U) \cdot \dim(V)
\end{equation*}
This is similar to how $ \dim(U \times V) = \dim(U) + \dim(V) $.
A typical element of $ U \otimes_{\bb{F}} V $ is given by:
\begin{equation*}
    \sum c_{ij}e_{i}\otimes f_{j}
\end{equation*}
and an element of the form $ a \otimes b $ is known as a \textbf{simple tensor}.
NOTE: Not all tensors are simple!

\begin{xmp}[source=Primary Source Material]
    Consider bilinear maps of the form $ \vphi: \bb{F}^{m} \times \bb{F}^{n}
    \rightarrow \bb{F} $. What are the linear maps? \vsp
    %
    Recall that $ \bb{F}^{m} \times \bb{F}^{n} \simeq \bb{F}^{m+n} $.
    Then, an element $ \bb{F}^{m+n} \rightarrow \bb{F} $ is an element of the
    dual space:
    \begin{equation*}
        \widehat{\bb{F}^{m+n}} \simeq \bb{F}^{m+n}
    \end{equation*}
    A good way of visualizing bilinear maps is as such:
    \begin{equation*}
        \vphi(u, v) \quad = \quad
        \begin{pmatrix}
            & \mbf{u} &
        \end{pmatrix}
        \begin{bmatrix}
            & & \\
            & \trm{something determined} & \\
            & \trm{by } \vphi & \\
            & &
        \end{bmatrix}
        \begin{pmatrix}
            \ \\ \mbf{v} \\ \
        \end{pmatrix}
    \end{equation*}
    So in this case, $ \bb{F}^{m} \otimes_{\bb{F}} \bb{F}^{n} \simeq
    M_{m\times n}(\bb{F}) $, and so it has dimension $ mn $.
    Here the basis element $ e_{i} \otimes f_{j} $ corresponds to the
    matrix $ E_{ij} \in M_{m\times n}(\bb{F}) $.
\end{xmp}

\begin{xmp}[source=Primary Source Material]
    Consider $ R = \bb{Z} $. Let $ M = \bb{Z}/3\bb{Z}, N = \bb{Z}/5\bb{Z} $.
    What is $ M \otimes_{R} N = \bb{Z}/3\bb{Z} \otimes \bb{Z}/5\bb{Z} $? \vsp
    %
    Consider $ \oline{(1, 0)} \in M \otimes_{R} N $. Then, we have:
    \begin{equation*}
        \oline{(1, 0)} = \oline{(1, 3\cdot0)}
        = \oline{(3\cdot1, 0)} = \oline{(0, 0)}
    \end{equation*}
    Similarly, $ \oline{(0, 1)} = 0 $.
    So, we see that $ \bb{Z}/3\bb{Z} \otimes_{R} \bb{Z}/5\bb{Z} = 0 $,
    and thus there are no maps. Indeed:
    \begin{align*}
        \oline{(a, b)} = \oline{(10a, b)} & = \oline{(5 \cdot 2a, b)} \\
                                          & = \oline{(2a, 5b)} \\
                                          & = \oline{(2a, 0)} \\
                                          & = \oline{(2a, 3\cdot0)} \\
                                          & = \oline{(6a, 0)} \\
                                          & = \oline{(0, 0)}
    \end{align*}
    Thus, there are no bilinear maps $ \bb{Z}/3\bb{Z} \times \bb{Z}/5\bb{Z}
    \rightarrow M $ for any $ \bb{Z} $-module $ M $.
\end{xmp}

\lecdate{Lec 37 - Feb 28 (Week 19)}

Given an $ n \times n $ matrix over $ \bb{C} $ (although any field works):
\begin{equation*}
    \begin{pmatrix}
        & & & & \\
        & & & & \\
        \mbf{v_{1}} & \mbf{v_{2}} & \cdots & \mbf{v_{n-1}} & \mbf{v_{n}} \\
        & & & & \\
        & & & &
    \end{pmatrix}
    \quad \in \quad \underbrace{\bb{C}^{n} \otimes \cdots \otimes \bb{C}^{n}}
    _{n \trm{ times}}
\end{equation*}
where each $ \bb{C}^{n} $ represents the columns.
$ S_{n} $ \textit{acts} on this tensor product by permuting the ``columns".
Let's look for a subspace on which the action of $ S_{n} $ is \textit{odd}
(switching two columns will result in a $ - $ sign).
Surprisingly, it turns out that only \textit{one} subspace that satisfies this;
it has dimension 1. Up to scalars, it is in fact
given by the \textbf{determinant}.

Bilinear maps helped define the tensor product.
We can extend this to \textit{multi-linear maps}, which gives us:
\begin{equation*}
    M_{1} \otimes \dots \otimes M_{n}
\end{equation*}
