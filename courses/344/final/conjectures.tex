\documentclass{article}
\usepackage{preamble}
\usepackage{env}
% \usepackage{configure}

% available environments:
% theorem: thm
% definition: defn
% proof: pf
% corollary: crll
% lemma: lm
% question: qu
% solution: soln
% example: xmp
% exercise: exr
%
% options: title=<title>   {all}
%          source=<source> {pf, qu, soln, xmp, exr}  Note: if content is taken directly from the main resource, cite the main resource as ``Primary source material"


% define these variables!
\def\coursecode{}
\def\coursename{} % use \relax for non-course stuff
\def\studytype{} % 1: Personal Self-Study Notes / 2: Course Lecture Notes / 3: Revised Notes
\def\author{Emerald (Emmy) Gu}
\def\createdate{}
\def\updatedate{\today}
\def\source{} % name, ed. of textbook, or `Class Lectures` for class notes
\def\sourceauthor{} % for class notes, put lecturer
\def\leftmark{MAT344 Project Conjectures} % set text in header; should only be necessary in assignments etc.
\pagenumbering{arabic} % force revert numbering to default; should only be necessary in assignments etc.

\begin{document}

% start here
\counterwithout{thm}{subsection}

\begin{thm}
Let $ G_{1}, G_{2} $ be simple, undirected, unweighted graphs.
Denote by $ A_{1}, A_{2} $ their adjacency matrices, respectively. \vsp
Suppose $ G_{1} $ and $ G_{2} $ are isomorphic. Then, $ A_{1} $ and $ A_{2} $ are similar matrices.
\end{thm}



\begin{thm}
Let $ G $ be a simple, undirected, unweighted graph.
Denote by $ A $ its adjacency matrix. \vsp
Suppose that all eigenvalues of $ A $ are integers. Then, $ G $ is a regular graph.
\end{thm}



\begin{thm}
Let $ G $ be a simple, undirected, unweighted graph on $ n $ vertices.
Denote by $ A $ its adjacency matrix. \vsp
Let $ \lambda $ be any eigenvalue of $ A $. Then, $ \abs{\lambda} \leq n - 1 $.
Furthermore, if $ G $ is \textit{not} the complete graph, then equality does \textit{not} hold.
\end{thm}

\newpage
\textbf{Write up: The questions the other person asked, and your brief responses to them.} \npgh

Q: Why do cycle graphs on odd number of vertices seem to have ``worse" eigenvalues? \vsp
A: I'm not sure, I didn't investigate this much seeing as even cycle graphs on even number of vertices can still get out of hand.
I imagine it has to do with the fact that the max degree of the characteristic polynomial is odd, but I'm not sure. \vsp
Q: For an $ n $-regular graph, do we expect $ n $ to be an eigenvalue? \vsp
A: Yes. Consider the vector $ v $ which has a 1 in every entry. Then, the product $ Av $ has the effect of counting the number of 1s in each row.
If the graph is $ n $-regular, then each row will have $ n $ 1s, which means each entry of $ Av $ will be $ n $. \vsp
Q: There seems to be a lot of room left to go from a fact about eigenvalues to what the graph has to be like. \vsp
A: I agree. I did try to make an effort in this direction by examining eigenvalues of subgraphs of $ K_{4} $, and I do have a conjecture in this direction.
However it does seem significantly more difficult to find results of this nature, and even if you do, they seem more tricky to prove.

\newpage
\textbf{Write up: What part of your explanations did they have the hardest time understanding? What will you do differently in your final project?} \npgh

While there wasn't too much that my partner seemed to have trouble understanding, there were a few things they inquired about.
Namely, although they had some experience with graph theory and common types of graphs (complete graphs, cycle graphs, trees, etc.), they were not familiar with star graphs.
For the final project, I plan on including some details for certain definitions, so that there is hopefully less confusion regarding this. \vsp
They also inquired a few times about the motivation for my method of investigation.
Because of this, I intend to outline my thought process in the final project in order to hopefully clarify why I chose to investigate the things I did.

\newpage
\textbf{Who did you work with?} \npgh

My partner was Jack McAsh. \npgh

\textbf{What did you learn from the other person?} \npgh

Jack had a conjecture relating eigenvalues to the maximum degree of vertices. It was somewhat similar to one of mine, however I had not thought to consider degree.
This gave a clearer connection to me regarding eigenvalues and degrees, and why one might expect certain (integer) eigenvalues to appear for a graph. \npgh

\textbf{What did you teach the other person?} \npgh

Jack said that I helped him with the formalization of the idea of maximum eigenvalues and degrees, and solidifying the connection between them. \npgh

\textbf{What was your common question, and what progress did you make on it?} \npgh

Seeing as we had two pretty similar conjectures, we both worked on solving it (It is listed as my third conjecture in check-in 3).
We each had our own ideas of why it might be true, and they came together pretty nicely, ultimately allowing us to prove the conjecture. \npgh

\textbf{Did you find working with someone else helpful?} \npgh

I did. I think it's very beneficial to have someone else with their own ideas and perspectives, and be able to learn from them and discuss together.
Talking to someone also helps to clarify your own ideas, and can reveal potential insights or gaps which may have been overlooked previously. \npgh

\textbf{Do you plan on working with them again?} \npgh

I do think it would be nice to work with Jack again. I think that there are times where you just have to sit down and work away at a problem individually,
and there are times when it's best to have someone else to bounce ideas off of. Jack was pleasant to work with, and I find that he is better than me at
noticing potential overarching connections between different properties, and maybe has better intutition as to why they might be true.

\end{document}
