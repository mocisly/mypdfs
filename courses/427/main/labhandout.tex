\documentclass{article}
\usepackage{otherpreamble}
\usepackage{otherenv}

% available environments:
% theorem: thm
% definition: defn
% proof: pf
% corollary: crll
% lemma: lm
% question: qu
% solution: soln
% example: xmp
% exercise: exr
%
% options: title=<title>   {all}
%          source=<source> {pf, qu, soln, xmp, exr}
%               Note: if content is taken directly from the main resource,
%                     cite the main resource as ``Primary source material"


% define these variables!
\def\coursecode{CSC427}
\def\coursename{} % use \relax for non-course stuff
\def\studytype{} % 1: Personal Self-Study Notes / 2: Course Lecture Notes / 3: Revised Notes / 4: Exercise Solution Sheet
\def\author{\me}
\def\createdate{}
\def\updatedate{\today}
\def\source{} % name, ed. of textbook, or `Class Lectures` for class notes
\def\sourceauthor{} % for class notes, put lecturer
\def\leftmark{Homomorphic Encryption - Lab Handout} % set text in header; should only be necessary in assignments etc.
\pagenumbering{arabic} % force revert numbering to default; should only be necessary in assignments etc.

\makeatletter
% settings for toc alignment
%
% Configuration
% -------------
% Horizonal alignment in \numberline:
%   l: left-aligned
%   c: centered
%   r: right-aligned
% \nl@align@: Default setting
% \nl@align@<levelname>: Setting for specific level

\def\nl@align@{l}% default
\def\nl@align@section{r}

\makeatother

\begin{document}

\coffeestainC{1}{0.5}{320}{125}{200}
\coffeestainD{0.3}{0.2}{75}{-20}{185}

During our presentation, we discussed how we can model the security of an encryption scheme against
chosen ciphertext attacks (CCA2, CCA1). We used a game to model a scenario of an attacker (Alice)
against a target (Bob). In this lab, you will be taking a closer look at the security of a given
encryption scheme.

Locate the Debian VM in the \texttt{/virtual/csc427/Debian 12.x 64-bit/} directory on the lab PC.
As you will be playing the role of the attacker, your login info will be:
\begin{itemize}
    \item Username: \texttt{alice}
    \item Password: \texttt{alice}
\end{itemize}
Navigate to the \texttt{/etc/cryptolab/} directory. There, you will find all the files you need.

\setcounter{subsection}{1}
\begin{qu}
    During the presentation, you may have noticed that we defined security against CCA2 attacks
    but did not discuss whether or not homomorphic encryption was sematically secure against such
    attacks. As it turns out, it is actually completely \textit{impossible} for a homomorphic
    encryption scheme to be secure against CCA2 attacks. \vsp
    %
    Explain how any such scheme can be broken by a CCA2 attack in only 2 queries.
\end{qu}

Naturally, an explanation alone won't be enough to convince us - so instead, you have the chance to
actually implement your strategy.

\begin{qu}
    In the \texttt{/etc/cryptolab} directory on the VM, you will find several scripts. Use these
    scripts to crack the homomorphic encryption by winning the CCA2 game. \vsp
    %
    Submit the keyword/keyphrase you receive upon winning as the answer to this question.
\end{qu}

Now that you've defeated the encryption system, you can take a closer look at its inner workings.

\begin{qu}
    The keyword/keyphrase you found in the last question is also the password to root.
    As root, examine the scripts and determine what encryption/decryption scheme was used.
\end{qu}

\begin{qu}
    Explain why the circuit provided can be used to crack this encryption scheme. \vsp
    %
    More precisely, suppose $ \pi_{1} $ and $ \pi_{2} $ are plaintexts, and $ \psi_{1} $ and
    $ \psi_{2} $ are their corresponding ciphertexts. Suppose $ f(x) $ represents the encryption
    scheme given in the game. Explain why encrypting the sum of the plaintexts is the same as
    adding the ciphertexts:
    \begin{equation*}
        \psi_{1} + \psi_{2} = f(\pi_{1} + \pi_{2})
    \end{equation*}
    Your answer should include/involve the encryption scheme you found in Question 3.
\end{qu}

\end{document}
