\documentclass{article}
\usepackage{preamble}
\usepackage{env}
\usepackage{configure}

\usepackage{colortbl}
\usepackage{arydshln}

\newcommand{\y}{\cellcolor{SeaGreen}Y}
\newcommand{\n}{\cellcolor[HTML]{fb7373}N}

% available environments:
% theorem: thm
% definition: defn
% proof: pf
% corollary: crll
% lemma: lm
% question: qu
% solution: soln
% example: xmp
% exercise: exr
%
% options: title=<title>   {all}
%          source=<source> {pf, qu, soln, xmp, exr}  Note: if content is taken directly from the main resource, cite the main resource as ``Primary source material"


% define these variables!
\def\coursecode{MAT244H1}
\def\coursename{Ordinary Differential Equations} % use \relax for non-course stuff
\def\studytype{} % 1: Personal Self-Study Notes / 2: Course Lecture Notes / 3: Revised Notes / 4: Exercise Solution Sheet
\def\author{\me}
\def\createdate{November 20, 2024}
\def\updatedate{\today}
\def\source{} % name, ed. of textbook, or `Class Lectures` for class notes
\def\sourceauthor{} % for class notes, put lecturer
% \def\leftmark{} % set text in header; should only be necessary in assignments etc.
% \pagenumbering{arabic} % force revert numbering to default; should only be necessary in assignments etc.

\makeatletter
% settings for toc alignment
%
% Configuration
% -------------
% Horizonal alignment in \numberline:
%   l: left-aligned
%   c: centered
%   r: right-aligned
% \nl@align@: Default setting
% \nl@align@<levelname>: Setting for specific level

\def\nl@align@{l}% default
\def\nl@align@section{r}

\makeatother

% custom title page
\def\typetext{Final Report}
\newcommand*{\customcover}{
\begin{titlepage}
\newgeometry{top=1.5in}
\hrule
\vspace*{0.5in}
\centering\Huge\textbf{\coursecode} \\[0.1in]
\huge\textbf{\coursename} \\[0.3in]
\LARGE\textsf{\typetext} \\[1.1in]
\large\textit{\author} \\[0.1in]
\textit{em.gu@mail.utoronto.ca} \\[0.1in]
\textit{1009185221} \\[0.6in]
\normalsize Last Updated: \\[0.1in]
\updatedate \\[0.8in]
\Large \ \\[0.2in]
\large\textit{\source} \\[0.3in]
\Large \ \\[0.2in]
\large\textit{\sourceauthor}
\vspace*{1in}
\hrule
\restoregeometry
\end{titlepage}
}

\begin{document}

\customcover
\toc
\pagenumbering{arabic}

% start here

\section{Background Assumptions}
\subsection{Assumptions Given}

We will first list the assumptions which form the basis of both the given model, as well as the
modified model of the system:

\begin{enumerate}[label=(A\arabic*)]
    \item The city land has a natural carrying capacity.
    \item People are attracted by people, and will move to the city if many other people already
        live in the city.
    \item People will move away from the city if rent is too high.
    \item Rent increases with inflation.
\end{enumerate}

We maintain these assumptions as they were used to create the initial model given, and because we
agree with these assumptions as reasonable, and use them as well as the given model to fuel our
analysis.

\newpage
\section{Executive Summary}
\subsection{Current Outlook}

We begin by analyzing the current model and its predictions.

The first observation we make stems from assumption (A4), as well as the fact that there are no
other assumptions directly affecting rent. Since (A4) asserts that rent increases with inflation,
and given that the average monthly rent per unit is currently $ \$2500 $, then we can conclude that
rent will always increase with nothing to slow it down. This gives an initial impression that if
left unmanaged, rent will skyrocket over time and more and more of the city populace will leave,
until the city is completely empty.

To form more definite conclusions, we can simulate the current system and analyze its behaviour over
time. We do so using the Improved Euler's method, and we get the following timeline:

\begin{itemize}
    \item In 1 year, the city's population will be roughly 7.09 million people. The average monthly
        rent per unit will be roughly $ \$2,625 $.
    \item In 2 years, the city's population will be roughly 8.49 million people. The average monthly
        rent per unit will be roughly $ \$2,760 $.
    \item In 3 years, the city's population will be roughly 9.2 million people. The average monthly
        rent per unit will be roughly $ \$2,900 $.
    \item In 4 years, the city's population will be roughly 9.5 million people. The average monthly
        rent per unit will be roughly $ \$3,050 $.
    \item In 5 years, the city's population will be roughly 9.6 million people. The average monthly
        rent per unit will be roughly $ \$3,210 $.
    \item In 6.4 years, the city's population will be roughly 9.64 million people.
        The population will peak at this time; from here on out, it will only decrease.
        The average monthly rent per unit at this time will be roughly $ \$3,440 $.
    \item In 10 years, the city's population will be roughly 9.59 million people.
        The average monthly rent per unit will be roughly $ \$4,120 $.
    \item In 15 years, the city's population will be roughly 9.47 million people.
        The average monthly rent per unit will be roughly $ \$5,290 $.
    \item In 20 years, the city's population will be roughly 9.31 million people.
        The average monthly rent per unit will be roughly $ \$6,790 $.
    \item In 30 years, the city's population will be roughly 8.8 million people. The average monthly
        rent per unit will be roughly $ \$11,200 $.
    \item In 40 years, the city's population will be roughly 7.82 million people.
        The average monthly rent per unit will be roughly $ \$18,470 $.
    \item In 50 years, the city's population will be roughly 5.15 million people.
        The average monthly rent per unit will be roughly $ \$30,450 $.
    \item In 53.9 years, the last person living in the city will move out, officially leaving the
        city empty as the average monthly rent per unit reaches roughly $ \$37,000 $.
\end{itemize}

Evidently, this is a grim forecast - within the decade, more people will emigrate from the city than
immigrate in, leading to the gradual departure of all citizens. This is a result of the combined
effects of assumptions (A1) and (A3); the inherent carrying capacity of the land limits the amount
of people that can move in, while the ever-increasing rent drives them away. Unfortunately,
legislation alone cannot increase the capacity of the land. However, we can instead suggest
legislations to control rent, in order to prevent it from reaching heights which drive out too many
people.

\subsection{Suggested Legislation}

To prevent rent from getting too high, we suggest the following legislations in order to assist in
controlling the fluctuations of the market:

\begin{enumerate}[label=(B\arabic*)]
    \item \textbf{Reduce the frequency at which rent can be increased.} This prevents owners from
        increasing rent too rapidly, and subtlely slows down the increase of rent. This is best
        used in combination with one of the below suggestions, as this targets frequency of increase
        rather than the value of the increase.
    \item \textbf{Owners charging rent above certain thresholds must pay higher taxes.} This
        heavily discourages owners from increasing rent above certain thresholds, as they would
        likely wish to save money where they can. This also has the advantage of not being a strict
        cutoff, which allows for some flexibility for owners in certain scenarios. However, this
        may not affect the particularly affluent.
    \item \textbf{Rent can only be increased proportionally to the current value beyond a certain
        threshold amount.} In other words, once monthly rent reaches a certain amount, any increase
        beyond the threshold must be made as a percentage of the current rent amount. Additionally,
        if a slower frequency of rent increase is also imposed, then this can result in a
        significant slowing.
    \item \textbf{Rent cannot be increased beyond a certain threshold amount.} This legislation
        would impose a strict limit to how high an individual unit can be rented for. This would
        strongly dampen the increase of rent, however not everyone will agree with this legislation
        as it is a more harsh rule.
\end{enumerate}

Each of the above legislations has advantages listed underneath the respective suggestion, however
they also each have some drawbacks. For instance, if the City does not wish to heavily impose taxes,
then suggestion (B2) is discouraged. However, if the City does not wish to impose strict cutoff
rules, then suggestion (B4) is discouraged.

As an example, we will simulate the system using legislation (B2), with a threshold of $ \$5,000 $.
Suppose that beyond this point, higher taxes are imposed, and owners wish to avoid paying these
higher taxes.

\end{document}
