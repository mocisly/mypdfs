% q10.1
\newpage
\label{q10}
\begin{qu}[title=Bolzano-Weierstrass Theorem,num=10.1]
    Prove that every bounded sequence in $ \bb{R}^{d}, (\norm{\cdot}_{2}) $ has
    a convergent subsequence.
\end{qu}

\begin{soln}
    We prove by induction on $ d $. \vsp
    %
    Base case: Suppose $ \bb{R}^{d} = \bb{R} $. This has already been proven.
    \vsp
    %
    Induction step: Suppose the claim holds for some natural number $ k $.
    We want to show it holds for $ k + 1 $. \vsp
    %
    Suppose $ (x_{i})_{i \geq 1} $ is a bounded sequence in $ \bb{R}^{k+1} $.
    Then for each $ x_{i} $, we have that:
    \begin{equation*}
        \norm{x_{i}} = \sqrt{\sum_{j=1}^{k+1} x_{ij}} \leq M
    \end{equation*}
    Note that:
    \begin{equation*}
        \sqrt{\sum_{j=1}^{k} x_{ij}} \leq \sqrt{\sum_{j=1}^{k+1}x_{ij}}
        = \norm{x_{i}} \leq M
    \end{equation*}
    That is, each term in the sequence is bounded in the first $ k $ coordinates.
    \vsp
    %
    By our I.H., we know that there is a subsequence $ (x_{i_{j}})_{j \geq 1} $
    such that the first $ k $ coordinates converge. For readability, we will
    re-index this sequence as $ (x_{n})_{n\geq1} $, using $ n $ for Nigel. \vsp
    %
    Since $ (x_{n}) $ is a subsequence of a bounded sequence, then it is also
    bounded. Therefore, by our base case, there is a subsequence $ (x_{n_{m}}) $
    such that the $ (k+1) $-th coordinate also converges. \vsp
    %
    Since each coordinate of the subsequence $ (x_{n_{m}}) $ converges, then
    the sequence converges pointwise to some $ x \in \bb{R}^{k+1} $.
    Therefore, by induction, every bounded sequence must have a convergent
    subsequence as needed.
\end{soln}

% q10.2
\begin{qu}[num=10.2]
    Give an example of a normed vector space $(X, \norm{\cdot})$ containing a
    bounded sequence $(x_n)$ which has no convergent subsequences.
\end{qu}

\begin{soln}
    Consider the normed vector space $ (C[0, 1], \norm{\cdot}_{\infty}) $, and
    the sequence given by $ (f_{n})_{n \geq 1} $ as $ f_{n} = x^{n} $. \vsp
    %
    Clearly, this is a bounded sequence as for each $ x^{n} $, we have
    $ \norm{x^{n}} = 1 $. So the sequence is bounded, but it has no convergent
    subsequences as needed.
\end{soln}


% q11.1
\newpage
\label{q11}
\begin{qu}[title=Banach Fixed Point Theorem,num=11.1]
    Suppose $ (X, d) $ is a complete metric space and $ f : X \gto X $ a
    contraction mapping. Prove that $ f $ has a unique fixed point.
\end{qu}

\begin{soln}
    Let $ x_{0} \in X $ be a point, and define a sequence $ (x_{n})_{n \geq 1} $
    given as:
    \begin{equation*}
        x_{n} = f(x_{n-1})
    \end{equation*}
    Note that for each $ n $-th term:
    \begin{equation*}
        d(x_{n+1}, x_{n}) \leq Md(x_{n}, x_{n-1}) \leq M^{2}d(x_{n-1}, x_{n-2})
        \leq \cdots \leq M^{n}d(x_{1}, x_{0})
    \end{equation*}
    Furthermore, by the triangle inequality, we have:
    \begin{align*}
        d(x_{n+m}, x_{n}) & \leq d(x_{n+m}, x_{n+m-1}) + d(x_{n+m-1}, x_{n}) \\
                          & \leq d(x_{n+m}, x_{n+m-1}) + d(x_{n+m-1}, x_{n+m-2})
                          + d(x_{n+m-2}, x_{n}) \\
                          & \ \: \vdots \\
                          & \leq \sum_{j=n}^{n+m-1} d(x_{j+1}, x_{j}) \\
                          & \leq \sum_{j=n}^{n+m-1} M^{j} d(x_{1}, x_{0}) \\
                          & \leq M^{n}d(x_{1}, x_{0})\sum_{j=0}^{m-1} M^{j} \\
                          & \leq M^{n}d(x_{1}, x_{0})\sum_{j=0}^{\infty}M^{j}
                          \vsp
                          & = \frac{M^{n}d(x_{1}, x_{0})}{1 - M} \vsp
                          & < M^{n}d(x_{1}, x_{0})
    \end{align*}
    Since $ M \in (0, 1) $, then $ \lim_{n \gto \infty} M^{n} = 0 $.
    Therefore as $ m \gto \infty $, we get that $ d(x_{n+m}, x_{n})
    \gto 0 $. Since this holds for any $ n $-th term, then we can conclude
    that the sequence is Cauchy, and therefore $ x_{n} \gto x $ for some
    $ x \in X $. Then, for any $ x_{n} $:
    \begin{equation*}
        d(f(x), x) \leq d(f(x), f(x_{n})) + d(f(x_{n}), x) \leq Md(x, x_{n})
        + d(x_{n+1}, x)
    \end{equation*}
    Since $ x_{n} \gto x $, then $ d(x_{n}, x) \gto 0 $.
    Therefore, we have that:
    \begin{equation*}
        d(f(x), x) = 0 \implies f(x) = x
    \end{equation*}
    So we have that $ x $ is a fixed point of $ f $ as needed.
\end{soln}

\begin{soln}[title=Uniqueness]
    Next, we show that the fixed point is unique.
    For this, suppose $ M \in (0, 1) $ and $ f $ is a contraction mapping. \vsp
    %
    Suppose $ x_{1} $ and $ x_{2} $ are distinct fixed points of $ f $. Then:
    \begin{gather*}
        d(f(x_{1}), f(x_{2})) < d(x_{1}, x_{2}) \qquad
        f(x_{1}) = x_{1} \qquad
        f(x_{2}) = x_{2} \vsp
        \implies \ d(f(x_{1}), f(x_{2})) = d(x_{1}, x_{2}) < d(x_{1}, x_{2})
    \end{gather*}
    Clearly, this is a contradiction, so there cannot be multiple fixed points.
\end{soln}

% q11.2
\newpage
\begin{qu}[num=11.2]
    Give an example of a normed vector space and a contraction mapping such that
    $ f $ does \textbf{not} have a fixed point.
\end{qu}

\begin{soln}
    Consider $ X = (C[0,1], \norm{\cdot}_{\infty}) $.
    Let $ B = B \left( 0, \frac{1}{2} \right) $, and define:
    \begin{equation*}
        \kappa: B \gto B \qquad \kappa(f) = f^{2} + \frac{1}{4}
    \end{equation*}
    To see that it is a contraction mapping, we first show that the sup norm is
    submultiplicative. Indeed, for all $ x \in [0,1] $, we have that:
    \begin{equation*}
        \abs{f(x)g(x)} = \abs{f(x)}\abs{g(x)} \leq \abs{f(x)}\norm{g} \leq
        \norm{f}\norm{g}
    \end{equation*}
    for some $ f, g \in X $.
    It then follows that $ \norm{fg} \leq \norm{f}\norm{g} $ as needed.
    Now, let $ f, g \in B $ such that $ f \neq g $. Then:
    \begin{align*}
        &  \norm{f^{2}(x) - g^{2}(x)} \\
        = \ & \norm{(f(x) - g(x))(f(x) + g(x))} \\
        \leq \ & \norm{f - g} \norm{f + g} \\
        < \ & \norm{f - g}
    \end{align*}
    Note that the last inequality follows from the fact that $ f, g \in B $.
    Next, to see that it has no fixed points, notice that any fixed point would
    satisfy:
    \begin{equation*}
        f = f^{2} + \frac{1}{4}
        \ \implies \ f^{2} - f + \frac{1}{4} = 0
        \ \implies \ \left( f - \frac{1}{2} \right)^{2} = 0
        \ \implies \ f = \frac{1}{2}
    \end{equation*}
    However, the constant function $ f = \frac{1}{2} \notin B $.
    Therefore, $ \kappa $ is a contraction mapping with no fixed points
    as needed.
\end{soln}


% q12.1
\newpage
\label{q12}
\begin{qu}[num=12.1]
    Let $ I \subseteq \bb{R} $.
    Prove that $ I $ is an interval if and only if $ I $ is connected.
\end{qu}

\begin{soln}
    First, we show that if $ I $ is disconnected, then it is not an interval.
    \vsp
    %
    Suppose $ I $ is disconnected. Then, $ I = A \sqcup B $ for some
    non-empty disjoint open sets $ A, B \subseteq \bb{R} $.
    Let $ a, b \in I $ such that $ a \in A, b \in B $. WLOG, assume $ a < b $.
    \vsp
    %
    Define $ x = \sup([a,b] \cap A) $. Clearly, $ x \in \bar{A} $ since supremums
    are contained in the set's closure. Furthermore, $ x \notin B $. Note that if
    $ x \notin A $, then $ x \notin I $, and so $ I $ cannot be an interval.
    But if $ x \in A $, then $ x \notin \bar{B} $, and so there exists some $ y $
    such that $ x < y < b $ and $ y \notin B $. But then it follows that $ I $
    cannot be an interval as needed. \vsp
    %
    Next, we show that if $ I $ is not an interval, then it is disconnected. \vsp
    %
    Suppose $ I $ is not an interval.
    Then, there exist $ a, b \in I $ such that $ [a, b] \nsubseteq I $.
    We must also have that $ (a, b) \nsubseteq I $; indeed, if this is not the
    case, then it follows that either $ a $ or $ b $ is not in $ I $, which is a
    contradiction. Therefore, we fix $ c \in (a, b) $ such that $ c \notin I $.
    Consider:
    \begin{equation*}
        A_{c} = (-\infty, c) \qquad B_{c} = (c, \infty)
    \end{equation*}
    Note that $ A_{c} \cap B_{c} = \varnothing $, so they are disjoint. Consider
    the sets:
    \begin{equation*}
        A = I \cap A_{c} \qquad B = I \cap B_{c}
    \end{equation*}
    Clearly, $ A $ and $ B $ are disjoint, and $ I = A \cup B $.
    Since $ A $ and $ B $ are open sets with respect to the subspace topology,
    then $ I $ is disconnected as needed.
\end{soln}

% q12.2
\newpage
\begin{qu}[num=12.2]
    Let $ X, Y $ be two metric spaces, and $ f : X \longrightarrow Y $ a
    continuous function. Prove that if $ C $ is a connected subset of $ X $,
    then $ f(C) $ is a connected subset of $ Y $.
\end{qu}

\vspace{-0.2in}
\begin{soln}
    We prove the contrapositive: if $ f(C) $ is disconnected, then $ C $ is
    disconnected. \vsp
    %
    Suppose $ f(C) = A \cup B $ for some non-empty, disjoint, open sets $ A, B
    \subseteq Y $. Since $ f $ is continuous, then $ f^{-1}(A) $ and
    $ f^{-1}(B) $ are open. Additionally, since $ A $ and $ B $ are disjoint,
    then their pre-images must also be disjoint. We show that $ f^{-1}(A) \cup
    f^{-1}(B) = C $. \vsp
    %
    Suppose otherwise; that is, there exists $ c \in C \setminus (f^{-1}(A)
    \cup f^{-1}(B)) $. Since $ c \in C $, then:
    \begin{equation*}
        f(c) \in C \implies f(c) \in A \cup B
    \end{equation*}
    WLOG, suppose $ f(c) \in A $.
    But this implies that $ c \in f^{-1}(A) $, which is a contradiction.
    Therefore, we must have that $ C $ is the union of two disjoint, open
    subsets of $ X $, and is thus disconnected as needed.
\end{soln}

% q12.3
\vspace{-0.15in}
\begin{qu}[num=12.3]
    Prove that the Intermediate Value Theorem (in $ \bb{R} $) follows from parts
    1 and 2, and thus part 2 is a generalization of IVT.
\end{qu}

\vspace{-0.15in}
\begin{soln}
    Let $ I \subseteq \bb{R} $ be an interval, and $ f : I \gto \bb{R} $
    be continuous.
    Fix $ a, b \in f(I) $ such that $ a < b $, and fix some $ a < y < b $. \vsp
    %
    Since $ f $ is continuous and $ I $ connected, then $ f(I) $ is also
    connected. Therefore:
    \begin{equation*}
        [a, b] \subseteq f(I) \implies y \in f(I)
    \end{equation*}
    Since $ y \in f(I) $, then there exists some $ x \in I $ such that
    $ f(x) = y $ as needed.
\end{soln}


% q13.1
%\newpage
%\begin{qu}[num=13.1]
%    Let $ (X, \norm{\cdot}_{X}), (Y, \norm{\cdot}_{Y}) $ be two normed vector
%    spaces, and $ T: X \gto Y $ a linear mapping. Prove that $ T $ is
%    continuous if and only if $ T $ is bounded.
%\end{qu}
%
%\begin{qu}[num=13.2]
%    Suppose $ (Y, \norm{\cdot}_{Y}) = (\bR^{n},\norm{\cdot}_{2}) $. Prove that if
%    $ T $ is continuous, then $ \ker(T) $ is closed.
%\end{qu}
%
%
%% q14
%\newpage
%\begin{qu}[num=14]
%    Let $ T: X \gto Y $ be a bounded linear mapping between complete normed
%    vector spaces $ X $ and $ Y $. Is it necessarily true that $ T^{-1} $ is
%    also bounded?
%\end{qu}
%
%
%% q15
%\newpage
%\begin{qu}[num=15]
%    Give an example of a complete normed vector space of countably infinite
%    dimension, or prove that no such example exists.
%\end{qu}


% q16
\newpage
\label{q16}
\begin{qu}[num=16]
    Let $ X $ be a normed vector space, and $ B(X) $ the space of bounded linear
    operators on $ X $. Equip $ B(X) $ with the operator norm
    $ \norm{\cdot}_{\trm{op}} $. \vsp
    %
    Prove that if $ X $ is complete, then $ \GL(X) $ is an open subset of
    $ (B(X), \norm{\cdot}_{\trm{op}}) $.
    %Bonus: Give an example showing that
    %$ \GL(X) $ need not be an open set in general.
\end{qu}

\begin{soln}
    Notice that since these are bounded linear operators, we can consider them
    as matrices in $ M_{n}(X) $. Consider the determinant:
    \begin{equation*}
        \det(A) \ = \ \sum_{i=1}^{n}(-1)^{i+1}A_{1i}\det(\bar{A}_{1i})
    \end{equation*}
    where $ \bar{A}_{1i} $ is the matrix obtained by removing the first row and
    $ i $th column. Notice that this is a polynomial over the entries given by:
    \begin{equation*}
        \set{E_{ij}} \qquad (E_{ij})_{ab} \ = \ \delta_{ai}\delta_{bj}
    \end{equation*}
    In other words, this is a basis of matrices which are 0 in all but one entry.
    In particular, since $ X $ is complete, then the determinant function is in
    fact continuous. \vsp
    %
    Recall that a matrix $ A $ is invertible if and only if $ \det(A) \neq 0 $.
    Thus, we have that:
    \begin{equation*}
        \GL(X) = (\det)^{-1}(\bR\setminus\set{0}) \qquad
        \bR\setminus\set{0} = (-\infty,0) \cup (0,\infty)
    \end{equation*}
    So $ \GL(X) $ is the continuous preimage of an open set, and so is open.
\end{soln}


% q17
\newpage
\label{q17}
\begin{qu}[title=Lebesgue Number Lemma,num=17]
    Let $ X $ be a clustering metric space. Prove that every open cover has a
    Lebesgue number.
\end{qu}

\begin{soln}
    Let $ \cl{U} = \set{U_{i}}_{i \in I} $ be an open cover of $ X $.
    Suppose for the sake of contradiction that $ \cl{U} $ does not have a
    Lebesgue number.
    In particular, it must be true that for all $ \delta > 0 $, there exists
    $ x_{\delta} \in X $ such that $ B(x_{\delta}, \delta) \nsubseteq U_{i} $
    for all $ U_{i} \in \cl{U} $. \vsp
    %
    Define a sequence $ (x_{n})_{n \geq 1} $ where each $ x_{n} $ is such that:
    \begin{equation*}
        B\left( x_{n}, \frac{1}{n} \right) \nsubseteq U_{i}, \quad U_{i} \in
        \cl{U}
    \end{equation*}
    Since $ X $ is clustering, then WLOG, the sequence converges to some
    $ x \in X $.
    Then, there must exist some $ U_{x} \in \cl{U} $ such that $ x \in U_{x} $.
    It follows that:
    \begin{equation*}
        B(x, \ep) \subseteq U_{x}, \quad \trm{for some } \ep > 0
    \end{equation*}
    Since $ x_{n} \gto x $, then there exist $ M_{1}, M_{2} \in \bb{N} $
    such that for all $ m_{1} \geq M_{1}, m_{2} \geq M_{2} $, we have that:
    \begin{equation*}
        x_{m_{1}} \in B(x, \ep) \qquad d(x, x_{m_{2}}) < \ep - \frac{1}{m_{2}}
    \end{equation*}
    Let $ M = \max(M_{1}, M_{2}) $. Then, for all $ m \geq M $, we see that:
    \begin{equation*}
        p \in B \left( x_{m}, \frac{1}{m} \right)
        \ \implies \ d(x, p) \leq d(x, x_{m}) + d(x_{m}, p) < \ep - \frac{1}{m}
        + \frac{1}{m} = \ep
    \end{equation*}
    This implies that $ B \left( x_{m}, \frac{1}{m} \right)
    \subseteq B(x, \ep) $, and so:
    \begin{equation*}
        B \left( x_{m}, \frac{1}{m} \right) \subseteq B(x, \ep) \subseteq U_{x}
        \ \implies \ B \left( x_{m}, \frac{1}{m} \right) \subseteq U_{x}
    \end{equation*}
    But this is a contradiction, so $ \cl{U} $ must indeed have a Lebesgue number
    as needed.
\end{soln}


% q18.1
\newpage
\label{q18}
\begin{qu}[num=18.1]
    Is $ (\ell^{1},\norm{\cdot}_{\infty}) $ separable?
\end{qu}

\begin{soln}
    No; to show this, we construct an uncountable subset $ S $ of $ \ell^{1} $
    such that for any $ x \neq y \in S $, we have that $ \norm{x-y} = 2 $. To do
    this, define $ S $ as:
    \begin{equation*}
        S \ = \ \set{(x_{n}):x_{n} \in \set{0, 2}}
    \end{equation*}
    This is uncountable by Cantor's Diagonalization argument; alternatively since
    each entry has two options, then there are $ 2^{\bN} = \abs{\cl{P}(\bN)} $
    possible sequences. Furthermore, for any two sequences $ x \neq y \in S $:
    \begin{equation*}
        \norm{x-y} \ = \ \sup\set{\abs{x_{n}-y_{n}}} \ = \ 2
    \end{equation*}
    Thus, we can place a ball of radius 1 around every point in $ S $. Now, if
    there was a countable dense subset $ D $, then there must be some element of
    $ D $ in each ball. However, because there are uncountably many balls, each
    of which are disjoint, this contradicts $ D $ being countable. Thus, we
    conclude that $ (\ell^{1},\norm{\cdot}_{\infty}) $ is not separable as
    needed.
\end{soln}

% q18.2
\begin{qu}[num=18.2]
    Is $ (\ell^{1},\norm{\cdot}_{1}) $ separable?
\end{qu}

\begin{soln}
    Yes. We construct a countable dense subset $ D $ as follows:
    \begin{equation*}
        D \ = \ \set{(x_{n}):x_{n}\in\bQ, x_{n} = 0 \trm{ for all } n \geq m
        \trm{ for some } m}
    \end{equation*}
    To see that this is dense, let $ y \in \ell^{1} $ be any sequence and fix
    $ \ep > 0 $. Note that there must exist some $ M $ such that:
    \begin{equation*}
        \sum_{n=M+1}^{\infty}\abs{y_{n}} < \frac{\ep}{2}
    \end{equation*}
    Furthermore, since $ \bQ $ is dense in $ \bR $, then there exists a sequence
    $ x \in D $ such that for all $ n \leq M $:
    \begin{equation*}
        \abs{x_{n}-y_{n}} < \frac{\ep}{2M}
    \end{equation*}
    Thus, we have that:
    \begin{equation*}
        \norm{x-y} = \sum_{n=1}^{\infty}\abs{x_{n}-y_{n}} \ = \
        \sum_{n=1}^{M}\abs{x_{n}-y_{n}} + \sum_{n=M+1}^{\infty}\abs{y_{n}} \ < \
        \sum_{n=1}^{M}\frac{\ep}{2M} + \frac{\ep}{2} \ = \ \frac{\ep}{2}+
        \frac{\ep}{2} = \ep
    \end{equation*}
    Thus we have that $ D $ is dense in $ (\ell^{1},\norm{\cdot}_{1}) $ as
    needed.
\end{soln}


% q19.1
\newpage
\label{q19}
\begin{qu}[num=19.1]
    Let $ U \subseteq \bb{R}^{n} $ be a bounded open set. For all $ n \in
    \bb{N} $, define:
    \begin{equation*}
        K_{n} = \set{p \in U : \norm{p - x} \geq \frac{1}{n} \trm{ for all } x
        \in \partial U}
    \end{equation*}
    Show that this is a compact exhaustion of $ U $.
\end{qu}

\begin{soln}
    First, we show that each $ K_{n} $ is compact.
    Clearly each $ K_{n} \subseteq U $, so we show that $ K_{n} $ is closed.
    Consider $ K_{n}^{c} $. We can write it as:
    \begin{align*}
        K_{n}^{c} = \ & U^{c} \cup \set{p \in U: \norm{p - x} <
        \frac{1}{n} \trm{ for some } x \in \partial U} \vsp
        = \ & U^{c} \cup \left( \bigcup_{x \in \partial U}
        B \left( x, \frac{1}{n} \right) \right) \vsp
        = \ & (U^{c} \setminus \partial U) \cup \left( \bigcup_{x \in \partial U}
        B \left( x, \frac{1}{n} \right) \right)
    \end{align*}
    Note that we can remove the boundary points of $ U $, as each boundary point
    is contained in one of the open balls in the right-hand union.
    Then, we see that each of these sets are open, and so the entire union is an
    open set. Therefore, $ K_{n}^{c} $ is open, so $ K_{n} $ is closed, and thus
    compact as needed. \vsp
    %
    Next, we show that $ U = \bigcup_{n \geq 1}K_{n} $.
    Indeed, since $ U $ is open, then for any $ x \in U $, there exists some
    $ \ep > 0 $ such that $ B(x, \ep) \subseteq U $. It clearly follows that
    there exists a sufficiently large $ N $ such that:
    \begin{equation*}
        B \left( x, \frac{1}{N} \right) \subseteq B(x, \ep) \subseteq U
    \end{equation*}
    Since $ x \notin \p U $, we can then take sufficiently large $ n \geq N $
    such that:
    \begin{equation*}
        \left( \bigcup_{y \in \p U}B \left( y, \frac{1}{n} \right) \right) \cap B
        \left( x, \frac{1}{n} \right) = \eset
    \end{equation*}
    Thus, we have that $ x \in K_{n} $. \vsp
    %
    Lastly, we show that for all $ n $, we have that $ K_{n} \subseteq
    K_{n+1}^{\circ} $. Suppose $ p_{0} \in K_{n} $ for some fixed $ n $. \vsp
    Let $ \delta = \dfrac{1}{n(n+1)} $. We show that $ B(p_{0}, \delta) \subseteq
    K_{n+1} $. Indeed, let $ p \in B(p_{0}, \delta) $, and fix $ x_{0} \in
    \partial U $ where $ \norm{p - x_{0}} \leq \norm{p - x} $ for all $ x \in
    \partial U $. Then:
    \begin{align*}
        \norm{p_{0} - x_{0}} \leq \ & \norm{p_{0} - p} + \norm{p - x_{0}} \vsp
        \implies \ \norm{p - x_{0}} \geq \ & \norm{p_{0} - x_{0}} - \norm{p_{0}
        - p} \\
        > \ & \frac{1}{n} - \delta \\
        = \ & \frac{1}{n} - \frac{1}{n(n+1)} \\
        = \ & \frac{n}{n(n+1)} \\
        = \ & \frac{1}{n+1}
    \end{align*}
    It follows that $ \norm{p - x} \geq \dfrac{1}{n+1} $ for all $ p \in
    B(p_{0}, \delta) $ and $ x \in \partial U $, so $ B(p_{0}, \delta)
    \subseteq K_{n+1} $ as needed. \vsp
    Since this holds for all $ p_{0} \in K_{n} $, then we have that $ K_{n}
    \subseteq K_{n+1}^{\circ} $ as needed.
\end{soln}

% q19.2
\newpage
\begin{qu}[num=19.2]
    Now, show that \textit{every} open subset of $ \bb{R}^{n} $ has a compact
    exhaustion.
\end{qu}

\begin{soln}
    Let $ U $ be an open subset of $ \bb{R}^{n} $. Define the set $ U_{n} $ as:
    \begin{equation*}
        U_{n} = U \cap B(0, n)
    \end{equation*}
    Note that for all $ n $, we have that $ U_{n} \subseteq U_{n+1} $.
    Define $ K_{n} $ as follows:
    \begin{equation*}
        K_{n} = \bigcup_{k=1}^{n} K_{k, n}
    \end{equation*}
    where $ K_{k, n} $ is the set defined in part a), replacing $ U $ with
    $ U_{k} $. Clearly, $ K_{n} $ is compact as it is the finite union of compact
    sets. Next, we show that $ U = \bigcup_{n \geq 1} K_{n} $. \vsp
    %
    Let $ x \in U $. Then, fix any $ k \in \bb{N} $ such that $ k > \norm{x} $.
    Then, it follows that:
    \begin{equation*}
        k > \norm{x} \implies x \in U_{k}
    \end{equation*}
    By the same argument in part a), there exists some $ n $ such that $ x \in
    K_{k, n} $, so $ x \in K_{n} $ as needed. \vsp
    %
    Lastly, we show that $ K_{n} \subseteq K_{n+1}^{\circ} $.
    Note that by part a), we have that $ K_{k, n} \subseteq K_{k, n+1}^{\circ} $.
    Recall that:
    \begin{gather*}
        K_{n} = \bigcup_{k=1}^{n} K_{k, n} \vsp
        K_{n+1}^{\circ} = \left( \bigcup_{k=1}^{n+1}K_{k,n+1} \right)^{\circ}
        = \bigcup_{k=1}^{n+1}K_{k,n+1}^{\circ}
    \end{gather*}
    Since $ (A \cup B)^{\circ} = A^{\circ} \cup B^{\circ} $ for any
    $ A, B \subseteq \bb{R}^{n} $, then the result follows.
\end{soln}
