\newpage
\subsection{Discontinuities in Integrable Functions}

The goal of this project is to prove a necessary and sufficient condition about
the integrability of a function. Recall the definition of a null set:
\begin{defn}
    A set $ N \subseteq \bR^{n} $ is called a \textbf{(Lebesgue) null set} if
    for all $ \ep > 0 $, there exist a countable collection
    $ \set{B_{i}}_{i\in\bN} $ of boxes $ B_{i} $ such that:
    \begin{equation*}
        N \subseteq \bigcup_{i=1}^{\infty}B_{i} \qquad
        \sum_{i=1}^{\infty}B_{i} < \ep
    \end{equation*}
\end{defn}
We wish to prove the following result:
\begin{thm}
    Let $ U \subseteq \bR^{n} $ be a box and $ f: U \gto \bR $ a bounded
    function. \vsp
    %
    Then $ f $ is integrable on $ U $ if and only if there exists a null set
    $ E \subseteq U $ such that $ f $ is continuous on $ V = U \setminus E $.
    Furthermore, the integral is given by:
    \begin{equation*}
        \int_{U}f \ = \ \int_{V}f
    \end{equation*}
\end{thm}

To assist us in this, we will define a familiar tool.
\begin{defn}
    Let $ U \subseteq \bR^{n} $ be any set and $ f: U \gto \bR $ a function.
    We define the \textbf{oscillation of} $ f $ \textbf{over} $ U $ as:
    \begin{equation*}
        \omega(f, U) \ = \ \sup_{x \in U}f - \inf_{x \in U}f
    \end{equation*}
\end{defn}
Intuitively, this measures how much $ f $ ``fluctuates" or ``oscillates" over a
particular set (hence the name). Applying this to boxes allows us to essentially
measure the ``error" of a function, and also ties in nicely with integrals via
the Cauchy criterion.

But first, recalling our statement: we need to verify that the integral on the
right is well-defined.
\begin{lm}
    Let $ U \subseteq \bR^{n} $ be measurable, and $ E \subseteq U $ a null set.
    Then $ U \setminus E $ is measurable.
\end{lm}

\begin{pf}
    Fix $ \ep > 0 $. Since $ U $ is measurable, there exist polyboxes $ O, I $
    (where $ O $ is an outer polybox and $ I $ is an inner polybox) of $ U $
    such that:
    \begin{equation*}
        \vol(O) - \vol(I) \ < \ \ep
    \end{equation*}
    Fix a countable collection of boxes $ \set{B_{i}} $ containing $ E $, such
    that $ \sum_{i}B_{i} < \frac{\ep}{2} $. Then, it follows that
    $ \vol(I \cap E) < \frac{\ep}{2} $, and we have:
    \begin{align*}
        \vol(O \setminus E) - \vol(I \setminus E)
        & \ \leq \ \vol(O) - \vol(I \setminus E) \\
        & \ = \ \vol(O) - \vol(I) + \vol(I) - \vol(I \setminus E) \\
        & \ = \ \vol(O) - \vol(I) + \vol(I\cap E) \\
        & \ < \ \frac{\ep}{2} + \frac{\ep}{2} \\
        & \ = \ \ep
    \end{align*}
    So $ U \setminus E $ is indeed measurable as needed.
\end{pf}

Next, we want to use oscillation together with the Cauchy criterion; for this,
we expect that it behaves the way we might want it to. The next theorem confirms
that this is indeed the case.

\begin{thm}
    Let $ U \subseteq \bR^{n} $ be any set, $ f: U \gto \bR $ a function, and
    $ p_{0} \in U $ any point. \vsp
    %
    Then $ f $ is continuous at $ p_{0} $ if and only if for all $ \ep > 0 $,
    there exists a neighbourhood $ U_{0} $ around $ p_{0} $ such that
    $ \omega(f, U_{0}) < \ep $.
\end{thm}

\begin{pf}
    Let $ \ep > 0 $. First, suppose $ f $ is continuous at $ p_{0} $. Then,
    there exists some $ \delta > 0 $ such that $ \norm{p-p_{0}} < \delta
    \ \implies \ \abs{f(p)-f(p_{0})} < \frac{\ep}{2} $. Consider $ U_{0} =
    B(p_{0}, \delta) \cap U $. Then, we see that:
    \begin{equation*}
        \omega(f,U_{0})\ = \ \sup_{U_{0}}f-\inf_{U_{0}}f\ = \
        \sup_{U_{0}}f-f(p_{0}) +f(p_{0})-\inf_{U_{0}}f \ < \
        \frac{\ep}{2}+\frac{\ep}{2}\ = \ \ep
    \end{equation*}
    This proves the sufficiency condition as needed. \vsp
    %
    Next, suppose there exists some neighbourhood $ U_{0} $ around $ p_{0} $ such
    that $ \omega(f,U_{0}) < \ep $. Pick $ \delta $ such that $ U' = B(p_{0},
    \delta) \cap U \ \subseteq \ U_{0} $. It thus follows that:
    \begin{equation*}
        \norm{p-p_{0}} \ < \ \delta \ \implies \
        p \in U' \ \implies \ \abs{f(p)-f(p_{0})} < \omega(f,U_{0}) < \ep
    \end{equation*}
    Thus, we have that $ f $ is continuous at $ p_{0} $ as needed.
\end{pf}

We are now ready to address the original statement.
We restate the theorem, then follow with a proof.

\begin{thm}
    Let $ U \subseteq \bR^{n} $ be a box and $ f: U \gto \bR $ a bounded
    function. \vsp
    %
    Then $ f $ is integrable on $ U $ if and only if there exists a null set
    $ E \subseteq U $ such that $ f $ is continuous on $ V = U \setminus E $.
    Furthermore, the integral is given by:
    \begin{equation*}
        \int_{U}f \ = \ \int_{V}f
    \end{equation*}
\end{thm}

\begin{pf}
    We begin by showing sufficiency. Suppose $ f $ is integrable on $ U $.
    Set $ E $ as the set of all points at which $ f $ is discontinuous:
    \begin{equation*}
        E \ = \ \set{x \in U: f \trm{ is not continuous at } x}
    \end{equation*}
    By definition, $ f $ is thus continuous on $ V $. It remains to show that
    $ E $ is a null set. Take a decreasing sequence $ \set{\ep_{i}} $ where each
    $ \ep_{i} > 0 $. For each $ \ep_{i} $, we have a partition $ P_{i} $
    containing $ E $ where:
    \begin{equation*}
        U(f,P_{i})-L(f,P_{i}) = \sum_{j}\omega(f,P_{i_{j}})\vol(P_{i_{j}}) < \ep
    \end{equation*}
    Now, suppose by contradiction that $ E $ is not a null set. Then, there
    exists some $ \ep > 0 $ such that for any countable partition $ Q $
    containing $ E $:
    \begin{equation*}
        \sum_{i}\vol(Q_{i}) \geq \ep
    \end{equation*}
    In particular, there exists some $ m > 0 $ such that for all $ i \geq m $:
    \begin{equation*}
        \sum_{j}\vol(P_{i_{j}}) \geq \ep_{i}
    \end{equation*}
    Fix any such $ i $. Then, we must have that both of the following hold:
    \begin{equation*}
        \sum_{j}\vol(P_{i_{j}}) \geq \ep_{i} \qquad
        \sum_{j}\omega(f,P_{i_{j}})\vol(P_{i_{j}}) < \ep_{i}
    \end{equation*}
    But as $ \ep_{i} $ is a decreasing sequence, it follows that
    $ \omega(f,P_{i}) \sto 0 $, as $ i \sto 0 $, which contradicts the fact that
    $ f $ is not continuous on $ E $. Since $ E $ is thus a null set, the
    identity also follows as a consequence.
    \vsp
    %
    Next, we show necessity. Suppose $ f $ is continuous on $ V $. First, notice
    that if $ \omega(f, U) = 0 $, then $ f $ must be constant, and the result is
    trivial. Therefore, suppose $ \omega(f, U) \neq 0 $. Fix $ \ep > 0 $, and
    set $ \delta = \frac{\ep}{2\omega(f,U)} $. \vsp
    %
    Since $ f $ is continuous on $ V $, it is integrable, and so we have a
    partition $ P $ of $ V $ such that:
    \begin{equation*}
        U(f,P) - L(f,P) \ = \ \sum_{i}\omega(f,P_{i})\vol(P_{i})
        \ < \ \frac{\ep}{2}
    \end{equation*}
    Since $ E $ is a null set, let $ B = \set{B_{i}} $ be a countable set of
    boxes such that $ \sum_{i}\vol(B_{i}) < \delta $. WLOG, we assume that each
    box in $ P $ and $ B $ are pairwise disjoint. Then, we have:
    \begin{align*}
        U(f,Q) - L(f,Q) & \ = \ \sum_{i}\omega(f,Q_{i})\vol(Q_{i}) \\
        & \ = \ \sum_{i}\omega(f,P_{i})\vol(P_{i})
            \ + \  \sum_{j}\omega(f,B_{j})\vol(B_{j}) \\
        & \ < \ \sum_{i}\omega(f,P_{i})\vol(P_{i})
            \ + \  \sum_{j}\omega(f,U)\vol(B_{j}) \\
        & \ = \ \sum_{i}\omega(f,P_{i})\vol(P_{i})
            \ + \  \omega(f,U)\sum_{j}\vol(B_{j}) \\
        & \ < \ \frac{\ep}{2} \ + \ \omega(f,U)\frac{\ep}{2\omega(f,U)} \\
        & \ = \ \frac{\ep}{2} \ + \ \frac{\ep}{2} \\
        & \ = \ \ep
    \end{align*}
    So we see that $ f $ is indeed integrable on $ U $ as needed.
    Notice the equality of the integrals follows from the construction of $ Q $.
\end{pf}

Note that this was shown over a given box; however, since we can extend
integrable functions to arbitrary measurable sets using polyboxes, the analogous
result holds generally.
