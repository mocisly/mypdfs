\subsection{Integrating over Manifolds}
Note: this project roughly began after the ``Parametrized Manifolds" worksheet
and before the more general ``(Smooth) Manifolds" worksheet.

The goal of this project is to define integration of functions defined on a
manifold. Of course, we first need to define what a manifold is; to fix some
issues with our working definition of (parametrized) manfiolds. We can ``reuse"
our current definition to make this work.

\begin{defn}
    Let $ M \subseteq \bR^{m} $ be a set. An $ n $\textbf{-chart} on $ M $ is a
    pair $ (\vphi, U) $ consisting of a relatively open set $ U \subseteq M $ and
    a diffeomorphism $ \vphi: \hat{U} \gto U $ for some open $ \hat{U}
    \subseteq \bR^{n} $, such that $ J\vphi $ has rank $ n $. \vsp
    %
    We define an \textbf{atlas on} $ M $ as a collection of charts on $ M $ which
    cover $ M $. \vsp
    %
    We say $ M $ is an $ n $\textbf{-manifold} if there is an atlas $ A $ on
    $ M $ such that each chart in $ A $ is an $ n $-chart for some fixed $ n $.
\end{defn}

Given that an atlas on $ M $ is a cover of charts, we want to make sure that it's
reasonable to work with these manifolds before attempting to do anything. For
this, we show the following.
\vspace{-0.2in}
\begin{lm}
    Any manifold $ M \subseteq \bR^{m} $ is locally compact.
\end{lm}
\vspace{-0.2in}
\begin{pf}
    Fix $ p \in M $. Note that $ p $ is contained in some chart
    $ (\vphi, U_{i}) $. Let $ \hat{p} = \vphi_{i}^{-1}(p) $. Since $ \bR^{n} $ is
    locally compact, there exists some compact set $ C \subseteq \hat{U_{i}} $
    containing $ \hat{p} $. Then $ \vphi_{i}(C) $ is a compact subset of $ M $
    containing $ p $.
\end{pf}

With that out of the way, we can work towards integration on a general manifold.
We'll do this in a few parts, putting pieces together as we go. The first part
considers the case where the support of $ f $ is contained within a single chart.

\begin{defn}
    Let $ f:M \gto \bR $ be a function with $ \supp(f) \subseteq U_{i} $ for
    some chart $ (\vphi_{i}, U_{i}) $, such that $ f \circ \vphi_{i} $ is
    integrable on $ \hat{U_{i}} $. Then, we define:
    \begin{equation*}
        \int_{U_{i}}f \ = \ \int_{\hat{U_{i}}}(f\circ\vphi_{i})\cdot V(J\vphi)
    \end{equation*}
    where $ V(A) = \sqrt{\det(A^{T}A)} $ is the function defined in handout.
\end{defn}

This shouldn't feel too strange, as it's quite similar to what we've already
worked with for the volume of a parametrized manifold. Already, however, we
might have that a set is contained within multiple charts; thus, we need to check
that the integral is well-defined.

\begin{thm}
    Suppose $ \supp(f) \subseteq U \cap V $ for some charts $ (\vphi, U),
    (\phi, V) $. Then:
    \begin{equation*}
        \int_{U}f \ = \ \int_{V}f
    \end{equation*}
    is well-defined; that is, it is independent of which chart we integrate over.
\end{thm}

\begin{pf}
    Recall that $ \phi^{-1}\circ\vphi:\vphi^{-1}(U\cap V)\gto\phi^{-1}
    (U\cap V) $ is a diffeomorphism. Thus:
    \begin{align*}
        & \int_{V}f \ = \ \int_{\hat{V}}(f\circ\phi)\cdot V(J\phi) \ = \
        \int_{\hat{U}}(f\circ\phi\circ\phi^{-1}\circ\vphi)\cdot V(J\phi)\cdot
        \abs{\det J(\phi^{-1}\circ\vphi)} \vsp
        \ = \ & \int_{\hat{U}}(f\circ\phi\circ\phi^{-1}\circ\vphi)\cdot V(J\phi)
        \cdot\abs{\det J(\phi^{-1}\circ\vphi)} \ \textcolor{blue}{=} \
        \int_{\hat{U}}(f\circ\vphi)\cdot V(J\vphi) \ = \ \int_{U}f
    \end{align*}
    where the equality in blue comes from a handout exercise.
\end{pf}

Now that we have integration for supports contained within a single chart, we can
consider general supports. To do so, we make use of atlases as a cover for our
manifold, and define the integral using a partition of unity.

\begin{defn}
    Suppose $ U \subseteq M $, and $ f:U \gto \bR $ is a function such that
    $ \supp(f) $ is compact in $ M $. Let $ A $ be an atlas on $ M $. Then we
    define:
    \begin{equation*}
        \int_{U}f \ = \ \sum_{i}\int_{U}\psi_{i}f
    \end{equation*}
    where $ \set{\psi_{i}} $ is a partition of unity subordinate to $ A $.
    Note that this is well-defined, as $ M $ is locally compact; thus, every
    sum is locally finite.
\end{defn}

Once again, we have another issue of well-definedness, since the integral is
defined in terms of the choice of partition of unity. We show next that this is
in fact independent of the choice of partition of unity.

\begin{thm}
    Suppose $ \set{\psi_{i}}, \set{\theta_{i}} $ are two partitions subordinate
    to $ A $. Then:
    \begin{equation*}
        \int_{U}f \ = \ \sum_{i}\int_{U}\psi_{i}f
        \ = \ \sum_{j}\int_{U}\theta_{j}f
    \end{equation*}
\end{thm}
\vspace{-0.2in}
\begin{pf}
    Notice:
    \begin{equation*}
        \sum_{i}\int_{U}\psi_{i}f \ = \
        \sum_{i}\left( \sum_{j}\int_{U}\theta_{j}\psi_{i}f \right) \ = \
        \sum_{j}\left( \sum_{i}\int_{U}\psi_{i}\theta_{j}f \right) \ = \
        \sum_{j}\int_{U}\theta_{j}f
    \end{equation*}
    by considering $ \int_{U}\psi_{i}f $ as its own integral.
\end{pf}

We also note that the integral requires a choice of atlas; however, at this point
the issue is already taken care of. This is because we showed that the integral
is locally independent of choice of charts; using partitions of unity, our sum
becomes a local statement, and the choice of partition also does not affect the
final integral.

\begin{crll}
    The integral is independent of choice of atlas.
\end{crll}

Now that the integral is properly defined, we can use it to define the volume of
a manifold; more generally, we can define the volume of a subset of a manifold
analogously to the $ \bR^{n} $ case, by integrating the constant $ 1 $ function.

\begin{crll}
    The volume of $ U \subseteq M $ is given by:
    \begin{equation*}
        \vol(U) \ = \ \int_{U}1 \ = \ \int_{M}\chi_{U}
    \end{equation*}
    where:
    \begin{equation*}
        \chi_{E} \ = \
        \begin{cases}
            1 & x \in E \\
            0 & x \notin E
        \end{cases}
    \end{equation*}
    is defined in the handout.
\end{crll}

Finally, we remark that the usual properties of integrals in $ \bR^{n} $ hold
over a manifold as well; namely, linearity, subadditivity, and monotonicity all
follow from the same properties in $ \bR^{n} $. This includes improper integrals,
as $ M $ is locally diffeomorphic to (a subset of) $ \bR^{n} $, and therefore we
can take a compact exhaustion by taking a compact exhaustion in the pre-image in
$ \bR^{n} $.

