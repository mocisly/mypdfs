% q30
\newpage
\label{q30}
\begin{qu}[num=30]
    Let $ U \subseteq \bR^{n} $ be open, and $ K \subseteq U $ compact.
    Prove that there exists an infinitely differentiable function $ \vphi:
    \bR^{n} \gto [0, 1] $ such that $ \vphi(p) = 1 $ for all $ p \in K $, and
    $ \vphi(p) = 0 $ for all $ p \in \bR^{n} \setminus U $. This is called a
    \textbf{bump function} supported on $ U $.
\end{qu}

\begin{soln}
    First, suppose $ K, U $ are closed/open rectangles respectively. Then, we can
    write:
    \begin{equation*}
        K = \prod_{i=1}^{n}[a_{i},b_{i}] \qquad
        U = \prod_{i=1}^{n}(c_{i},d_{i})
    \end{equation*}
    By Big List 4, we know there exist bump functions $ \vphi_{i}:\bR\gto[0,1] $
    such that $ \vphi_{i}(p_{i})=1 $ for all $ p \in [a_{i},b_{i}] $ and
    $ \vphi_{i}(p_{i})=0 $ for all $ p \in \bR\setminus(c_{i},d_{i}) $. Thus,
    we can define our function:
    \begin{equation*}
        \vphi(p) \ = \ \prod_{i=1}^{n}\vphi_{i}(p_{i})
    \end{equation*}
    Indeed, when $ p \in K $, each $ p_{i} \in [a_{i},b_{i}] $, and so we have
    $ \vphi_{i}(p_{i})=1 $. Similarly, if $ p \in \bR^{n}\setminus U $, then
    $ p_{i} \in \bR\setminus(c_{i},d_{i}) $ for some $ i $, and thus $ \vphi_{i}
    (p_{i})=0 $ as needed. \vsp
    %
    Next, suppose $ K $ is a finite union of rectangles; WLOG, write
    $ K = K_{1} \cup K_{2} $, where each $ K_{i} $ is a rectangle. Then, there
    are bump functions $ \alpha, \beta $ on $ K_{1}, K_{2} $ respectively. We
    define:
    \begin{equation*}
        \vphi(p) \ = \ \alpha(p) + \beta(p) - \alpha(p)\beta(p)
    \end{equation*}
    Here, we mimic the principle of inclusion/exclusion. Clearly, $ \vphi $ is
    smooth; for any $ p \in K_{1} $ or $ K_{2} $, we respectively have that:
    \begin{equation*}
        \vphi(p) \ = \ 1 + \beta(p) - \beta(p) \qquad
        \vphi(p) \ = \ \alpha(p) + 1 - \alpha(p)
    \end{equation*}
    So clearly $ \vphi(p) = 1 $ on $ K $. It is also easy to see that it is 0
    outside of $ U $ as needed. \vsp
    %
    Finally, consider any arbitrary $ K $. Since $ U $ is open, then it has a
    compact exhaustion $ \set{K_{i}} $. Then, there must exist some $ m-1 $
    such that $ K \subseteq K_{m-1} $. Otherwise, we would have that $ K $ is not
    a subset of any $ K_{i} $ for all $ i $, and since $ \bigcup_{i}K_{i} = U $,
    then $ K $ could not be a subset of $ U $. Thus, we have $ K_{m} $ compact
    where $ K \subseteq K_{m}^{\circ} \subseteq K_{m} \subseteq U $. \vsp
    %
    Now, let $ \set{C_{\alpha}} $ be a cover of rectangles of $ K_{m} $, with
    each $ C_{\alpha} \subsetneq U $. Since $ K_{m} $ compact, there exists a
    finite subcover $ \set{C_{r}}_{r=1}^{s} $. Since each $ C_{r} \subsetneq U $,
    we can define $ \vphi $ analogously to the finite union case above to obtain
    our bump function as needed.
\end{soln}

%% q31.1
%% has hint, kinda; see 9.10-9.12
%\newpage
%\begin{qu}[num=31.1]
%    Let $ A \in M_{n}(\bR) $ be symmetric, and let $ Q(\vec{x}) = \vec{x}^{T}
%    A\vec{x} $ be the corresponding quadratic form. Prove the following are
%    equivalent:
%    \begin{itemize}
%        \item $ Q(\vec{x}) > 0 $ for all $ \vec{x} \neq \vec{0} $
%        \item All eigenvalues of $ A $ are strictly positive
%    \end{itemize}
%\end{qu}
%
%% q31.2
%% has hint
%\begin{qu}[num=31.2]
%    Prove the following ``stay away" lemma: If $ Q: \bR^{n} \gto \bR $ is a
%    positive definite quadratic form, then there exists a constant $ \nu > 0 $
%    such that $ Q(\vec{x}) \geq \nu\norm{\vec{x}}^{2} $ for all $ \vec{x} \in
%    \bR^{n} $.
%\end{qu}
%
%% q31.3
%% has hint
%\begin{qu}[num=31.3]
%    Let $ U \subseteq \bR^{n} $ be open, $ f: U \gto \bR $ twice continuously
%    differentiable, and $ p_{0} \in U $ a point at which $ \nabla f(p_{0}) = 0 $.
%    Prove that if the Hessian $ Hf(p_{0}) $ is positive definite, then $ f $
%    achieves a local minimum at $ p_{0} $; i.e., $ p_{0} $ has an open
%    neighbourhood $ U_{0} \subseteq U $ such that $ f(p) \geq f(p_{0}) $ for all
%    $ p \in U_{0} $.
%\end{qu}


% q32.1
\newpage
\label{q32}
\begin{qu}[num=32.1]
    Let $ M \subseteq \bR^{n} $ be the zero set of some $ C^{1} $ function
    $ f: U \gto \bR $, for some open $ U \subseteq \bR^{n} $. Suppose $ p_{0}
    \in M $ such that $ \nabla f(p_{0}) \neq 0 $, and set $ v \in \bR^{n} $.
    Show that $ v $ is a tangent vector to $ M $ at $ p_{0} $ if and only if
    $ \nabla f(p_{0}) \cdot v = 0 $.
\end{qu}

\begin{soln}
    Suppose $ v $ is a tangent vector at $ p_{0} $. Then, there exists some curve
    $ \alpha $ such that $ \alpha(0) = p_{0} $, and $ \alpha'(0)=v $. Consider
    $ f\circ\alpha $. Clearly, we have that $ (f\circ\alpha)(0)=0 $;
    differentiating both sides, we see that:
    \begin{equation*}
        (f\circ\alpha)'(0)=(\nabla f\circ\alpha(0))\cdot\alpha'(0)=
        \nabla f(p_{o})\cdot v=0
    \end{equation*}
    On the other hand, if $ v $ is not a tangent vector, then there are no such
    curves. Since each curve can be given as a linear combination of curves along
    the axes (basis vectors), and the equation $ \nabla f(p_{0})\cdot v $ defines
    a subspace of dimension $ n-1 $, then we must have that $ \nabla f(p_{0})
    \cdot v \neq 0 $.
\end{soln}

% q32.2
\newpage
\begin{qu}[num=32.2]
    Let $ M $ be the ellipsoid in $ \bR^{3} $ defined by:
    \begin{equation*}
        x^{2} + yz + y^{2} - xy - xz + z^{2} = 3
    \end{equation*}
    Find the equation of the tangent plane to $ M $ at the point
    $ p_{0} = (1, 2, 0) $.
\end{qu}

\begin{soln}
    Using the process outlined in part one, we see that:
    \begin{gather*}
        f(x,y,z) \ = \ x^{2}+yz+y^{2}-xy-xz+z^{2}-3 \\
        \nabla f \ = \
        \begin{bmatrix}
            2x-y-z & z+2y-x & y-x+2z
        \end{bmatrix}
    \end{gather*}
    We see that $ \nabla f(1,2,0) \neq 0 $, and so the equation of the tangent
    plane is given as:
    \begin{equation*}
        \nabla f(1,2,0)\cdot v = 0 \ \implies \ 3v_{2}+v_{3} = 0
    \end{equation*}
    as needed.
\end{soln}


%% q33.1
%\newpage
%\begin{qu}[title=Lagrange Multipliers,num=33.1]
%    Let $ g: U \gto \bR $ be a $ C^{1} $ function defined on some open
%    $ U \subseteq \bR^{n} $, and let $ M \subseteq U $ be its zero set.
%    Suppose $ f: U \gto \bR $ is $ C^{1} $ and achieves its maximum on $ M $
%    at some point $ p_{0} $, and that $ \nabla g(p_{0}) \neq 0 $.
%    Prove there exists $ \lambda \in \bR $ such that:
%    \begin{equation*}
%        \nabla f(p_{0}) = \lambda\nabla g(p_{0})
%    \end{equation*}
%\end{qu}
%
%\begin{soln}
%    From Question 32, we know that for any tangent vector $ v $ at $ p_{0} $, we
%    have that $ \nabla g(p_{0})\cdot v = 0 $. We can repeat the proof to show
%    that $ \nabla f(p_{0})\cdot v' = 0 $ for any tangent vector $ v' $.
%    However, this directly implies that $ \nabla f(p_{0}) $ and $ \nabla
%    g(p_{0}) $ are parallel, and thus such a lambda must exist.
%\end{soln}
%
%% q33.2
%\begin{qu}[num=33.2]
%    Find the points on the ellipsoid:
%    \begin{equation*}
%        x^{2} + yz + y^{2} - xy - xz + z^{2} = 3
%    \end{equation*}
%    which are closest and furthest from the origin, using Lagrange multipliers.
%\end{qu}
%
%
%% q34.1
%\newpage
%\begin{qu}[num=34.1]
%    Let $ \Phi: \bR^{n} \gto \bR^{m} $ be $ C^{1} $, with $ n > 1 $ and
%    $ m = 1 $. Show that $ \Phi $ cannot be injective.
%\end{qu}
%
%% q34.2
%\begin{qu}[num=34.2]
%    Suppose $ n < m $. Show that if $ K \subseteq \bR^{n} $ is compact, then
%    $ \Phi(K) $ is Jordan measurable with measure 0.
%\end{qu}
%
%
%% q35.1
%% has hint
%\newpage
%\begin{qu}[num=35.1]
%    Let $ f(x) = \sum_{i=0}^{n} a_{i}x^{i} $ be a monic polynomial with no
%    repeated real roots. Let $ r $ be a root of $ f $. Prove that for all
%    $ \ep > 0 $, there exists $ \delta > 0 $ such that: \vsp
%    %
%    If $ g(x) = \sum_{i=0}^{n}b_{i}x^{i} $ is a monic polynomial with
%    $ \abs{a_{i} - b_{i}} < \delta $, then $ g(x) $ has at least one root in the
%    interval $ (r-\ep, r+\ep) $.
%\end{qu}
%
%% q35.2
%\begin{qu}[num=35.2]
%    Suppose that $ f $ has fewer than $ n $ roots. Prove that the number of real
%    roots does not change under small perturbation of the coefficients.
%\end{qu}


% q36.1
\newpage
\label{q36}
\begin{qu}[num=36.1]
    Let $ A \subseteq \bR^{n} $ be any set, $ f: A \gto \bR $ a function.
    Prove that $ f $ is differentiable at every point of $ A $ if and only if
    $ f $ extends to a differentiable function defined on an open set containing
    $ A $.
\end{qu}

\begin{soln}
    The reverse direction trivially follows from the definition; we will prove
    the forward direction. \vsp
    %
    Suppose $ f $ is differentiable. Then for all
    $ p \in A $, there exists an open neighbourhood $ U_{p} $ containing $ p $
    and a function $ \hat{f}_{p}: U_{p} \gto \bR $ which is differentiable at
    $ p $. Furthermore, we know $ \hat{f}_{p}\rvert_{U_{p}\cap A} =
    f\rvert_{U_{p}\cap A} $. Define $ U = \bigcup_{p\in A}U_{p} $. Note that
    $ U $ is open; furthermore, $ \set{U_{p}} $ forms an open cover of $ A $.
    \vsp
    %
    Let $ \set{\vphi_{i}} $ be a partition subordinate to $ U $. Define the
    function $ \hat{f}:U \gto \bR $ as:
    \begin{equation*}
        \hat{f}(p) \ = \ \sum_{p\in A}\vphi_{p}(p)\hat{f}_{p}(p)
    \end{equation*}
    Notice that $ \hat{f} $ is well-defined, as:
    \begin{equation*}
        \hat{f}_{p}\big\rvert_{U_{p}\cap A} = f\big\rvert_{U_{p}\cap A}
        \ \implies \ \hat{f}(p) = \sum_{p \in A}\vphi_{p}(p)\hat{f}_{p}(p)
        = \sum_{p \in A}\vphi_{p}(p)f(p) = f(p)
    \end{equation*}
    This is because $ \hat{f} $ is a locally finite sum. It follows that
    $ \hat{f} $ is differentiable on $ A $. \vsp
    %
    However, it is not true that $ \hat{f} $ can always be differentiable on 
    $ U $. Consider:
    \begin{equation*}
        \frac{1}{x(x-1)}:(0,1) \gto \bR
    \end{equation*}
    This is differentiable on its domain, but cannot be extended to a function
    which is differentiable on an open set containing $ (0,1) $.
\end{soln}

% q36.2
\begin{qu}[num=36.2]
    Suppose that $ A $ is closed. Prove that $ f $ is differentiable at every
    point of $ A $ if and only if $ f $ extends to a differentiable function on
    $ \bR^{n} $.
\end{qu}

\begin{soln}
    This is false as given: consider $ \dfrac{1}{x} $ defined on the closed
    interval $ [2,3] $. This is clearly differentiable on its domain, but cannot
    be extended to all of $ \bR $.
\end{soln}

% q37
% hint in the form of a certain yt video lolll
\newpage
\label{q37}
\begin{qu}[title=A Mathematician's Dream,num=37]
    The following is called the $ n $-simplex:
    \begin{equation*}
        \Delta_{n} :=
        \set{x = (x_{1}, \dots, x_{n}) \in \bR^{n}:
        x_{1}, \dots, x_{n} \geq 0 \trm{ and } x_{1} + \dots + x_{n} \leq 1}
    \end{equation*}
    Find, with proof, an explicit formula for $ \mu(\Delta_{n}) $ in terms of
    $ n $.
\end{qu}

\begin{soln}
    Note that we can model the first few simplexes as:
    \begin{gather*}
        \Delta_{1} = [0, 1] \qquad
        \Delta_{2} = \set{(x, 1-x) \in \bR^{2} : x \in [0, 1]} \\
        \Delta_{3} = \set{(x, y, 1-x-y) \in \bR^{3} : x \in [0, 1], y \in
        [0, 1-x]}
    \end{gather*}
    Thus, we can use the integral form of measure and integrate over these sets:
    \begin{gather*}
        \mu(\Delta_{n}) \ = \ \int_{\Delta_{n}}1 \vsp
        \mu(\Delta_{1}) \ = \ \int_{0}^{1}1\ \d x \qquad
        \mu(\Delta_{2}) \ = \ \int_{0}^{1}\int_{0}^{1-x}1\ \d y\d x \vsp
        \mu(\Delta_{3}) \ = \ \int_{0}^{1}\int_{0}^{1-x}\int_{0}^{1-x-y}
        1\ \d z\d y\d x
    \end{gather*}
    We can generalize this to $ \bR^{n} $:
    \begin{equation*}
        \mu(\Delta_{n}) \ = \ \underbrace{
            \int_{0}^{1}\int_{0}^{1-x_{1}}\cdots
            \int_{0}^{1-\sum_{i=1}^{n-1}x_{i}}
        }_{n \trm{ integrals}} 1\ \d x_{n}\d x_{n-1}\cdots \d x_{1}
    \end{equation*}
    To find a closed-form expression, we can evaluate the integral.
    To do this, we define an auxiliarry variable:
    \begin{equation*}
        c_{0} = 1 \qquad c_{k} = 1 - \sum_{i=1}^{n-k}x_{i}
    \end{equation*}
    We emphasize that:
    \begin{gather*}
        c_{k} = c_{k+1}-x_{n-k} \vsp
        \int_{0}^{1}\cdots
        \int_{0}^{1-\sum_{i=1}^{n-1}x_{i}}
        1\ \d x_{n}\cdots \d x_{1} \ = \
        \int_{0}^{c_{n}}\cdots
        \int_{0}^{c_{1}}
        c_{0}\ \d x_{n}\cdots \d x_{1}
    \end{gather*}
    Finally, to prove the closed form, we claim that:
    \begin{equation*}
        \int_{0}^{c_{k+1}}c_{k}^{k}\ \d x_{n-k} \ = \
        \frac{1}{k+1}c_{k+1}^{k+1}
    \end{equation*}
    Indeed, we see that:
    \begin{align*}
        \int_{0}^{c_{k+1}}c_{k}^{k}\ \d x_{n-k}
        & \ = \ \int_{0}^{c_{k+1}}(c_{k+1}-x_{n-k})^{k}\ \d x_{n-k} \vsp
        & \ = \ \int_{0}^{c_{k+1}}\sum_{i=0}^{k}\binom k i c_{k+1}^{k-i}
        (-x)_{n-k}^{i}\ \d x_{n-k} \vsp
        & \ = \ \sum_{i=0}^{k}\frac{(-1)^{i}}{i+1}\binom k ic_{k+1}^{k-i}
        x_{n-k}^{i+1}\bigg\rvert_{0}^{c_{k+1}} \vsp
        & \ = \ c_{k+1}^{k+1}\sum_{i=0}^{k}\frac{(-1)^{i}}{i+1}\binom k i \vsp
        & \ = \ c_{k+1}^{k+1}\sum_{i=0}^{k}(-1)^{i}\binom k i \int_{0}^{1}x^{i}\
        \d x \vsp
        & \ = \ c_{k+1}^{k+1}\int_{0}^{1}\sum_{i=0}^{k}(-1)^{i}\binom k ix^{i}\
        \d x \vsp
        & \ = \ c_{k+1}^{k+1}\int_{0}^{1}(1-x)^{k}\ \d x \vsp
        & \ = \ c_{k+1}^{k+1}\int_{0}^{1}u^{k}\ \d k \vsp
        & \ = \ c_{k+1}^{k+1}\frac{1}{k+1}
    \end{align*}
    Thus, evaluating the integral, we get:
    \begin{align*}
        \int_{0}^{c_{n}}\cdots \int_{0}^{c_{1}}
        c_{0}\ \d x_{n}\cdots \d x_{1} & \ = \ \frac{1}{1}
        \int_{0}^{c_{n}}\cdots \int_{0}^{c_{2}}
        c_{1}\ \d x_{n-1}\cdots \d x_{1} \vsp
        & \ = \ \frac{1}{2} \int_{0}^{c_{n}}\cdots \int_{0}^{c_{3}}
        c_{2}^{2}\ \d x_{n-2}\cdots \d x_{1} \vsp
        & \ = \ \frac{1}{2\cdot3} \int_{0}^{c_{n}}\cdots \int_{0}^{c_{4}}
        c_{3}^{3}\ \d x_{n-3}\cdots \d x_{1} \\
        & \ \ \ \vdots \\
        & \ = \ \frac{1}{n!}
    \end{align*}
    So we conclude that $ \mu(\Delta_{n}) = \dfrac{1}{n!} $ as needed.
\end{soln}


% q38.1
\newpage
\label{q38}
\begin{qu}[num=38.1]
    Let $ f: (0,1)^{2} \gto \bR $ be the function given by $ f(x,y) =
    \dfrac{1}{1-xy} $, and let:
    \begin{equation*}
        K_{N} = \left[ \frac{1}{N}, 1-\frac{1}{N} \right]^{2}
    \end{equation*}
    Evaluate $ \displaystyle\int_{K_{N}}f $ using Fubini's theorem.
\end{qu}

\begin{soln}
    First, notice that $ x,y \in (0,1) $. Thus, we evaluate:
    \begin{align*}
        \int_{\frac{1}{N}}^{1-\frac{1}{N}}\int_{\frac{1}{N}}^{1-\frac{1}{N}}
        \frac{1}{1-xy}\di y\di x & \ = \
        \int_{\frac{1}{N}}^{1-\frac{1}{N}}\int_{\frac{1}{N}}^{1-\frac{1}{N}}
        \left( \sum_{k=0}^{\infty}(xy)^{k} \right)\di y\di x \vsp
        & \ = \ \sum_{k=0}^{\infty}\int_{\frac{1}{N}}^{1-\frac{1}{N}}
        \int_{\frac{1}{N}}^{1-\frac{1}{N}} (xy)^{k}\di y\di x \vsp
        & \ = \ \sum_{k=0}^{\infty}\int_{\frac{1}{N}}^{1-\frac{1}{N}} x^{k}
        \int_{\frac{1}{N}}^{1-\frac{1}{N}} y^{k}\di y\di x \vsp
        & \ = \ \sum_{k=0}^{\infty}\left(\int_{\frac{1}{N}}^{1-\frac{1}{N}}x^{k}
        \di x \right)^{2} \vsp
        & \ = \ \sum_{k=0}^{\infty}\left(\frac{x^{k+1}}{k+1}\bigg\rvert_{\frac{1}
        {N}}^{1-\frac{1}{N}}\right)^{2} \vsp
        & \ = \ \sum_{k=0}^{\infty}\left(\frac{(1-\frac{1}{N})^{k+1}-(\frac{1}
        {N})^{k+1}}{k+1}\right)^{2} \vsp
        \lim_{N\sto\infty}\sum_{k=0}^{\infty}\left(\frac{(1-\frac{1}{N})^{k+1}-
                (\frac{1}{N})^{k+1}}{k+1}\right)^{2}
        & \ = \ \sum_{k=0}^{\infty}\lim_{N\sto\infty}\left(\frac{(1-\frac{1}
                {N})^{k+1}-(\frac{1}{N})^{k+1}}{k+1}\right)^{2} \vsp
        & \ = \ \sum_{k=0}^{\infty}\frac{1}{(k+1)^{2}} \vsp
        & \ = \ \sum_{n=1}^{\infty}\frac{1}{n^{2}}
    \end{align*}
    Note the swapping of the integral and sum, and sum and limit, come from
    uniform convergence.
\end{soln}

% q38.2
\begin{qu}[num=38.2]
    Evaluate $ \displaystyle\int_{K_{N}}f $ using Change of Variables twice:
    first using the linear diffeomorphism given by $ (x,y) = (u+v, u-v) $,
    then using the polar coordinates transform.
\end{qu}

\begin{soln}
    We know from Part 1 that the integral does indeed converge; thus we can
    manipulate the integral directly.
    \begin{align*}
        \int_{0}^{1}\int_{0}^{1}\frac{1}{1-xy}\di y\di x & \ = \
        2\int_{0}^{\frac{1}{2}}\int_{-u}^{u}\frac{\di v\di u}{1-u^{2}+v^{2}}
        \ + \ 2\int_{\frac{1}{2}}^{1}\int_{u-1}^{1-u}\frac{\di v\di u}
        {1-u^{2}+v^{2}} \vsp
        & \ = \ 2\int_{0}^{\frac{1}{2}}\frac{1}{\sqrt{1-u^{2}}}\cdot
        \left( \arctan \left( \frac{v}{\sqrt{1-u^{2}}} \right)\bigg\rvert_{-u}^u
        \right)\di u \vsp
        & \ + \ 2\int_{\frac{1}{2}}^{1}\frac{1}{\sqrt{1-u^{2}}}\cdot\left(
        \arctan \left( \frac{v}{\sqrt{1-u^{2}}} \right)\bigg\rvert_{1-u}^{u-1}
        \right)\di u \vsp
        & \ = \ 2\int_{0}^{\frac{\pi}{6}}\arctan\left(\frac{\sin\theta}
        {\cos\theta}\right)-\arctan\left(\frac{-\sin\theta}{\cos\theta}\right)
        \di\theta \vsp
        & \ + \ 2\int_{\frac{\pi}{6}}^{\frac{\pi}{2}}\arctan\left(\frac{1-\sin
        \theta}{\cos\theta}\right)-\arctan\left(\frac{\sin\theta-1}{\cos\theta}
        \right)\di\theta \vsp
        & \ = \ 2\int_{0}^{\frac{\pi}{6}}2\theta\di\theta \ + \
        2\int_{\frac{\pi}{6}}^{\frac{\pi}{2}}\frac{\pi}{2}-\theta\di\theta \vsp
        & \ = \ \frac{\pi^{2}}{18}+\frac{\pi^{2}}{3}-\frac{\pi^{2}}{4}
        +\frac{\pi^{2}}{36} \vsp
        & \ = \ \frac{\pi^{2}}{6}
    \end{align*}
    Note that a large amount of algebra was skipped for this solution.
\end{soln}

% q38.3
\begin{qu}[title=Basel Problem,num=38.3]
    Solve the Basel problem.
\end{qu}

\begin{soln}
    We see in part one that the integral evalutates to the reciprocal of squares,
    and in part two that the integral evaluates to a particular constant. Thus:
    \begin{equation*}
        \sum_{n=1}^{\infty}\frac{1}{n^{2}} \ = \ \frac{\pi^{2}}{6}
    \end{equation*}
    as needed.
\end{soln}


% q39.1
\newpage
\label{q39}
\begin{qu}[num=39.1]
    Let $ M \subseteq \bR^{N} $ be a set and $ \vphi:U \gto M $ be a smooth
    parametrization of $ M $, defined on some open $ U \subseteq \bR^{n} $.
    Prove that if $ \vphi $ is a regular parametrization of $ M $, then
    $ \vphi^{-1} $ is also smooth.
\end{qu}

\begin{soln}
    Let $ \vphi:U \rightarrow M $ be a smooth regular embedding.
    Fix $ p \in U $. Since $ \vphi $ is regular, $ J\vphi(p) $ has rank $ n $,
    so there is an invertible submatrix $ J'\phi $. WLOG, suppose it is the
    Jacobian of the first $ n $ component functions $ \vphi_{1}, \dots,
    \vphi_{n} $ of $ \vphi $. \vsp
    %
    Denote by $ \pi_{n}: \bR^{N} \gto \bR^{n} $ the projection onto the first
    $ n $ components. Then $ \pi_{n}\circ\vphi $ is a map from $ \bR^{n} \gto
    \bR^{n} $. Since $ J'\phi $ is rank $ n $, then there exists a point $ p $
    such that $ J'\phi $ is invertible, so by inverse function theorem,
    $ (\pi_{n}\circ\vphi)^{-1} $ is differentiable on some
    $ V_{0} = \pi_{n}(\vphi(U_{0})) $ for some neighbourhood $ U_{0} $ of $ p $.
    In particular, we have that $ \vphi^{-1} $ differentiable on this
    neighbourhood. \vsp
    %
    Since $ \vphi'(p) = J\vphi(p) $ is linear, then $ \vphi''(p)(q) = \vphi'(p) $
    for all $ q \in U $ and $ p \in U $. Thus, we repeat the argument inductively
    to conclude that $ \vphi^{-1} $ is smooth.
\end{soln}

% q39.2
\begin{qu}[num=39.2]
    Prove that if $ \vphi:U \gto M $ and $ \psi: V \gto M $ are two regular
    parametrizations of $ M $, then $ \vol_{\vphi}(M) = \vol_{\psi}(M) $.
\end{qu}

\begin{soln}
    Suppose $ \vphi, \psi $ are smooth regular embeddings.
    By part a, they are also diffeomorphisms. \vsp
    %
    Let $ \Phi: U \rightarrow V $ be given by $ \Phi = \psi^{-1} \circ \vphi $.
    Then, $ \Phi $ is a diffeomorphism, and so the result holds by Theorem 14.8
    from handout.
\end{soln}

% q39.3
\begin{qu}[num=39.3]
    Use regular parametrizations to find the surface area of a sphere of radius
    $ r $ in $ \bR^{4} $.
\end{qu}

\begin{soln}
    Note that any sphere can be isometrically shifted to be centered about the
    origin, so WLOG consider $ S = \set{(x, y, z, w) :
    x^{2}+y^{2}+z^{2}+w^{2} = r^{2}} $ to be our sphere of radius $ r $.
    We split $ S $ into the following two sets:
    \begin{equation*}
        S_{1} = \set{(x, y, z, w) \in S : y \geq 0} \qquad
        S_{2} = \set{(x, y, z, w) \in S : y < 0}
    \end{equation*}
    Note that while $ S_{1} $ is not open, integrating over an open measurable
    set and a closed set is equivalent wrt the boundary, and so this will not
    matter. \vsp
    %
    We can parametrize $ S_{1} $ and $ S_{2} $ using polar coordinates:
    \begin{equation*}
        (x, y, z, w) \mto (r\cos\theta\sin\phi\sin\rho,
        r\sin\theta\sin\phi\sin\rho,
        r\cos\phi\sin\rho, r\cos\rho)
    \end{equation*}
    Let $ \alpha: (0, \pi)^{3} $ and $ \beta : (\pi, 2\pi) \times (0, \pi)^{2} $
    be the parametrizations of $ S_{1} $ and $ S_{2} $ respectively.
    Their Jacobians are given by:
    \begin{equation*}
        J\alpha = J\beta =
        \begin{bmatrix}
            -r\sin\theta\sin\phi\sin\rho & r\cos\theta\cos\phi\sin\rho &
            r\cos\theta\sin\phi\cos\rho \\
            r\cos\theta\sin\phi\sin\rho & r\sin\theta\cos\phi\sin\rho &
            r\sin\theta\sin\phi\cos\rho \\
            0 & -r\sin\phi\sin\rho & r\cos\phi\cos\rho \\
            0 & 0 & -r\sin\rho
        \end{bmatrix}
    \end{equation*}
    After much simplifying, we get:
    \begin{equation*}
        (J\alpha)^{\intercal}(J\alpha) =
        \begin{bmatrix}
            r^{2}\sin^{2}\phi\sin^{2}\rho & 0 & 0 \\
            0 & r^{2}\sin^{2}\rho & 0 \\
            0 & 0 & r^{2}
        \end{bmatrix} \qquad \quad
        V(J\alpha) = r^{3}\sin\phi\sin^{2}\rho
    \end{equation*}
    Note $ V(J\alpha) = V(J\beta) $ since they only differ by domain.
    We then calculate:
    \begin{equation*}
        \trm{vol}(S) \ = \ \int_{(0, \pi)^{3}}V(J\alpha)
        + \int_{(\pi, 2\pi) \times (0,\pi)^{2}}V(J\beta)
    \end{equation*}
    Using Fubini's, we see that:
    \begin{gather*}
        \int_{(0, \pi)^{3}}V(J\alpha)
        \ = \ \int_{0}^{\pi} \int_{0}^{\pi} \int_{0}^{\pi}
        r^{3}\sin\phi\sin^{2}\rho \ 
        \d\phi\d\rho\d\theta \vsp
        \int_{(\pi, 2\pi) \times (0, \pi)^{2}}V(J\beta)
        \ = \ \int_{\pi}^{2\pi} \int_{0}^{\pi} \int_{0}^{\pi}
        r^{3}\sin\phi\sin^{2}\rho \ 
        \d\phi\d\rho\d\theta
    \end{gather*}
    Since $ V(J\alpha) $ is independent of $ \theta $, then $ \trm{vol}(S)
    = 2\trm{vol}(S_{1}) $. Evaluating, we get:
    \begin{align*}
        2\int_{0}^{\pi}\int_{0}^{\pi}\int_{0}^{\pi}
        r^{3}\sin\phi\sin^{2}\rho \ \d\phi\d\rho\d\theta \
        = \ & 4\pi r^{3} \int_{0}^{\pi}\sin^{2}\rho \ \d\rho \vsp
        = \ & 4\pi r^{3} \left(\frac{\rho}{2} - \frac{\sin\rho}{4} \
        \bigg\rvert_{0}^{\pi} \right) \vsp
        = \ & 4\pi r^{3} \frac{\pi}{2} \vsp
        = \ & 2\pi^{2}r^{3}
    \end{align*}
    So the surface area of a sphere with radius $ r $ in $ \bb{R}^{4} $ is
    given by $ 2\pi^{2} r^{3} $ as needed.
\end{soln}
