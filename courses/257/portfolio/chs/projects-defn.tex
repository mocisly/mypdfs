\subsection{Definition Project - Compactness}

This is a summary of the key points from the ``Definition Project" on compactness that I did
together with Kirat.

We begin with the following ``open cover" definition of compactness.

\begin{defn}
    Let $ (X, d) $ be a metric space. \vsp
    %
    Consider any collection of open subsets $ \set{U_{i}}_{i \in I} $ where each
    $ U_{i} \subseteq X $, for some indexing set $ I $.
    We say that $ U = \bigcup_{i \in I} U_{i} $ is an \underline{open cover} of $ X $ if
    $ X \subseteq U $. \vsp
    %
    We say that $ X $ is \textbf{compact} if for every open cover $ U $ of $ X $, there exists some
    natural number $ n \in \bb{N} $ such that $ X \subseteq \bigcup_{k=1}^{n} U_{i_{k}} $.
    We call such a cover a \underline{finite subcover}.
\end{defn}

In an effort to make sense of the definition and what kinds of subsets in a metric space are
compact, we first start by considering $ \bb{R} $. We find that on the real line, the only
intervals which are compact are closed, bounded intervals:
\begin{xmp}
    Consider the interval $ [0, 1] $. We want to check if this is compact. \vsp
    %
    Indeed, this interval is compact.
    To see this, let $ U = \bigcup_{i \in I} U_{i} $ be any open cover.
    Note that there exists some $ U_{0} $ such that $ 0 \in U_{0} $.
    Thus, $ [0, 0] $ has a finite subcover. \vsp
    %
    Now, consider $ [0, x] $ such that it has a finite subcover for some $ x \in (0, 1) $.
    Since there exists some $ U_{x} $ such that $ x \in U_{x} $, then there exists some
    $ \delta > 0 $ such that $ [x, x + \delta] \subseteq U_{x} $.
    Since $ [0, x] \subseteq \bigcup_{k=1}^{n} U_{i_{k}} $ has a finite cover, then
    $ [0, x] \cup [x, x + \delta] \subseteq \bigcup_{k=1}^{n} U_{i_{k}} \cup U_{x} $,
    so $ [0, x + \delta] $ also has a finite cover. \vsp
    %
    Finally, note that the above holds generally true for all $ x \in (0, 1) $, by a process
    similar to induction.
    Also, since $ \delta > 0 $, then $ [0, x] \subsetneq [0, x + \delta] $, thus the above
    is sufficient to show that $ [0, 1] $ has a finite subcover as needed.
\end{xmp}
As it turns out, any other interval in $ \bb{R} $ is not compact.
This gives us a good idea of which sets are compact in $ \bb{R} $ - that is, closed and bounded
sets - but before we prove this, we now turn to $ \bb{R}^{n} $ to see if we can generalize our
statement further.

Indeed, it's not hard to see that union and intersection preserve compactness.
The cartesian product is much tricker to see, however we can actually get a bidirectional claim.

\begin{lm}
    Let $ X, Y $ be two metric spaces. Then $ X \times Y $ is compact if and only if $ X $ and
    $ Y $ are both compact.
\end{lm}

\begin{pf}[source=Kirat] %TODO -- fix formatting etc
    $ (\implies) $ Let $X,Y$ be compact, and consider any open cover of $X\times Y \subseteq
    \bigcup_{i\in I} U_i$. now consider some arbitrary $x \in X$, then $x\times Y \subseteq
    \bigcup_{i\in I} U_i$. We also have $U_i \subseteq X\times Y$, and $U_i = \bigcup_{j\in J} V_j
    \times W_j$, where $V_j,W_j$ are open sets in $X,Y$ respectively. So we have that $x\times Y
    \subseteq \bigcup_{i\in I}\{\bigcup_{j\in J} V_j\times W_j\}_i$, and so $x\times Y \subseteq
    \bigcup_{i\in I}\{\bigcup_{j\in J} (V_j\times W_j)\cap (x\times Y)\}_i = \bigcup_{i\in I}
    \{\bigcup_{k\in K} (x\times W_k)\}_i$, where the sets that are empty after the intersection
    with $x\times Y$, are not included. Next we know that $Y\subseteq \bigcup_{i\in I}
    \{\bigcup_{k\in K} W_k\}_i$, but since $Y$ is compact and this is an open cover, we have that
    $Y\subseteq\bigcup_{a=1}^n W_{a,i_a}$, where $i_a$ is just what $U_i$ it came from. However
    since this finite open cover was chosen from $W_k$ who always came paired with $x$, we have
    that $x\times Y\subseteq \bigcup_{a=1}^n U_{i_a}$. \vsp
    %
    Now recall $U_i = \bigcup_{j\in J} V_j\times W_j$ and so we have that $x\times Y\subseteq
    \bigcup_{a=1}^n V_{a,i_a}\times W_{a,i_a}$. Since there are finitely many $V_{a,i_a}$, and all 
    of them are open, we have that $B(x,r_a) \subseteq V_{a,i_a}$ for each a. So if we take the
    min, we have some $r$ such that $B(x,r) \subseteq V_{a,i_a}$ for all $a$. Hence we have that
    $B(x,r)\times Y \subseteq\bigcup_{a=1}^n U_{i_a}$. \vsp
    %
    Since this was for an arbitrary point in $X$, we know that for any $x_m \in X$, where $m\in M$ 
    and $M$ is some index set such that $X\subseteq\bigcup_{m\in M} x_m$, there is some $r_m$ such 
    that $B(x_m,r_m)\times Y \subseteq\bigcup_{a_m=1}^{n_m} U_{i_{a_m}}$. Also since $X\subseteq
    \bigcup_{m\in M} x_m$, we have that $X\subseteq\bigcup_{m\in M} B(x_m,r_m)$. Next, since
    $B(x_m, r_m)$ is open and $X$ is compact there is some finite set such that $X\subseteq
    \bigcup_{b=1}^{n} B(x_{m_b}, r_{m_b})$. So we have $X\times Y \subseteq \bigcup_{b=1}^{n}
    (B(x_{m_b}, r_{m_b})\times Y)$, which is a finite union. We also have that $B(x_{m_b}, r_{m_b})
    \times Y\subseteq \bigcup_{a_{m_b}=1}^{n_{m_b}} U_{i_{a_{m_b}}}$. Finally since we have that
    $X\times Y \subseteq \bigcup_{b=1}^{n} \{\bigcup_{a_{m_b}=1}^{n_{m_b}} (U_{i_{a_{m_b}}})\}$,
    and a finite union of finite unions of sets is finite. Hence, we have that $X\times Y\subseteq 
    \bigcup_{c=1}^n U_{i_c}$, where this is some finite open cover of $X\times Y$. So if $X,Y$ are 
    compact then, $X\times Y$ is also compact. \npgh

    $ (\Longleftarrow) $ Suppose that $X\times Y$ is compact. Without loss of generality we just
    have to show that $X$ is compact. Suppose we have some open cover so that $X\subseteq
    \bigcup_{i\in I} U_i$. Then since each $U_i$ is open, we have that $U_i\times Y$ is open. So we
    know that $X\times Y \subseteq\bigcup_{i\in I} U_i \times Y$. However, since $X\times Y$ is
    compact we have some finite set of indices such that $X\times Y \subseteq\bigcup_{k=1}^n
    U_{i_k} \times Y$. So that means $X\subseteq \bigcup_{k=1}^n U_{i_k}$. This shows that $X$ is
    compact. Finally, this means that if $X\times Y$ is compact, we know that $X,Y$ are each
    compact.
\end{pf}

So given that closed and bounded intervals in $ \bb{R} $ are the only compact intervals, and that
compact sets can be ``constructed" using the three basic set operations, then this suggests that
sets in $ \bb{R}^{n} $ have the same structure. Indeed, this is what the Heine-Borel theorem tells
us.

\begin{thm}[title=Heine-Borel Theorem]
    Let $ S $ be a subset of $ \bb{R}^{n} $. Then, $ S $ is compact if and only if $ S $ is
    closed and bounded.
\end{thm}

\begin{pf}
    We begin by proving that closed and bounded implies compact. This is done in two steps.
    First, we show that any countable cover must have a finite subcover.
    Then, we show that any uncountable cover has a countable subcover,
    thereby reducing to the first step. \vsp
    %
    Suppose for the sake of contradiction that $ S $ is not compact.
    Let $ U = \set{U_{i}}_{i=1}^{\infty} $ be a countable open cover,
    such that there is no finite subcover.
    More precisely, this means that for any $ n \in \bb{N} $:
    \begin{equation*}
        S \nsubseteq \bigcup_{j = 1}^{n} \set{U_{i_{j}}} 
    \end{equation*}
    So any union of finitely many open sets from the cover do not completely cover $ S $.
    This allows us to define the following sequences:
    \begin{equation*}
        \set{p_{k}}_{k=1}^{\infty} \quad , \quad p_{k} \in
        S \setminus (U_{1} \cup U_{2} \cup \dots \cup U_{k})
    \end{equation*}
    Since $ S $ is a bounded set, then our sequence is bounded.
    Therefore, by the Bolzano-Weierstrass Theorem, there exists a subsequence
    $ \set{p_{k_{j}}}_{j \geq 1} $ such that $ p_{k_{j}} \rightarrow p $ for some $ p \in S $. \vsp
    %
    Since $ p_{k_{j}} \rightarrow p $, then $ p $ is a limit point of $ S $.
    Since $ S $ is closed, then $ p \in S $.
    Therefore, there exists some $ U_{p} $ such that $ p \in U_{p} $.
    Since $ U_{p} $ is an open set, then there must exist $ \delta > 0 $ such that
    we have that $ B(p, \delta) \subseteq U_{p} $. \vsp
    %
    Since $ p_{k_{j}} \rightarrow p $, then there exists some $ N $ such that
    for all $ j \geq N $, $ p_{k_{j}} \in U_{p} $.
    In particular, there exists some $ p_{k_{N}} $ such that:
    \begin{equation*}
        p_{k_{N}} \in U_{p} \quad , \quad p_{k_{N}} \in
        S \setminus (U_{1} \cup U_{2} \cup \dots \cup U_{p} \cup \dots \cup U_{k_{N}})
    \end{equation*}
    Clearly, this is a contradiction, as we have that
    $ p_{k_{N}} \in U_{p} $ and $ p_{k_{N}} \notin U_{p} $.
    Therefore, it must be true that $ S $ has a finite subcover as required. \vsp
    %
    Now, suppose that $ U = \set{U_{i}}_{i \in I} $ is an uncountable open cover of $ S $.
    As mentioned earlier, it suffices to show that there exists a countable subcover,
    thereby reducing it to the first step. \vsp
    %
    Recall that $ S $ is bounded. Then, by the Bolzano-Weierstrass Theorem,
    any sequence in $ S $ has a subsequence $ (x_{n_{i}}) $ which
    converges to some $ x \in S $. This gives us a cluster point $ x $ for the sequence $ (x_{n}) $.
    Since this is true of all sequences in $ S $, then $ S $ is a clustering subset of $ X $. \vsp
    %
    Since $ S $ is a clustering subset, then by the Lebesgue Number Lemma, there exists some
    $ \delta > 0 $ such that $ \delta $ is the magic number of our open cover $ U $. \vsp
    %
    Consider the set $ S' = S \cap \bb{Q}^{n} $. Since $ \bb{Q}^{n} $ is a countable set,
    then this is a countable set of points in $ S $.
    Therefore, it makes sense to define the set $ U_{s} $ as:
    \begin{equation*}
        U_{s} = \bigcup_{i=1}^{\infty} \set{B \left( s_{i}, \frac{\delta}{2} \right) } \quad
        s_{i} \in S'
    \end{equation*}
    This is indeed a cover, as $ \bb{Q}^{n} $ is dense in $ \bb{R}^{n} $.
    Furthermore, since each open ball has diameter less than $ \delta $,
    then there exists $ U_{s_{i}} $ such that:
    \begin{equation*}
        B \left( s_{i}, \frac{\delta}{2} \right) \subseteq U_{s_{i}}
    \end{equation*}
    Therefore, $ U_{s} $ is a countable subcover of $ S $, so by the first step,
    $ S $ must be compact, as required. \npgh

    Now, we show that if $ S $ is compact, then it is closed and bounded.
    We'll do this by contradiction, and take each case separately. \vsp
    %
    Suppose $ S $ is not closed. Then, there exists a limit point $ p \in \overline{S} $ such that
    $ p $ is not in $ S $ itself. \vsp
    %
    Consider the open cover defined as:
    \begin{equation*}
        \cl{U} = \bigcup_{i=1}^{\infty} U_{i} \qquad
        U_{i} = \left(\overline{B} \left( p, \frac{1}{i} \right) \right)^{c}
    \end{equation*}
    We can see that $ \cl{U} = \bb{R} \setminus \set{p} $, so $ S \subseteq \cl{U} $.
    Suppose by contradiction that $ S $ is compact; then, there exists a finite collection
    such that:
    \begin{equation*}
        S \subseteq \bigcup_{k = 1}^{j} U_{i_{k}}
    \end{equation*}
    Since this is finite, then we can consider the highest index $ m $.
    Since $ p $ is a limit point, then:
    \begin{equation*}
        U_{i_{m+1}} \cap S = B \left( p, \frac{1}{m+1} \right) \cap S \neq \varnothing
    \end{equation*}
    So there exists some $ x \in U_{i_{m+1}} \cap S $ such that $ x $ is not in the finite
    collection, but $ x \in S $, so this is a contradiction. Therefore, $ S $ is not compact. \npgh

    Now, suppose $ S $ is not bounded. Let $ x \in K $ be any point. Then, consider the open cover:
    \begin{equation*}
        \cl{U} = \bigcup_{i = 1}^{\infty} B(x, i)
    \end{equation*}
    Clearly, $ S \subseteq \cl{U} $.
    Now, suppose that $ S $ is compact. Then, there exists a finite subcover of $ \cl{U} $.
    Since it's finite, denote the maximum of these indices as $ m $. \vsp
    %
    Since $ S $ is unbounded, then there exist $ x_{1}, x_{2} \in S $ such that:
    \begin{gather*}
        d(x_{1}, x_{2}) > 2m \\
        \implies \ d(x_{1}, x) + d(x, x_{2}) \geq d(x_{1}, x_{2}) > 2m
    \end{gather*}
    So we must have that one of $ x_{1} $ and $ x_{2} $ has distance more than $ m $ from $ x $.
    WLOG, suppose this point is $ x_{1} $.
    Then, $ x_{1} \notin B(x, m) $ so is not in the finite subcover, but is a point in $ S $.
    This is a contradiction, so $ S $ must not be compact as needed.
\end{pf}

This gives us a \textit{complete} characterization of compact sets in Euclidean space. Furthermore,
we know from linear algebra that for finite-dimensional vector spaces, two vector spaces are
are isomorphic if they have the same dimension.

Recall from Exercise 4.24 that the continuous image of compact sets is compact. We also have that
isomorphisms between vector spaces (specifically, change of basis maps) are continuous, and so
we get the following generalization fairly easily:

\begin{thm}[title=Generalized Heine-Borel Theorem]
    Let $ S $ be a subset of \textbf{a finite-dimensional normed vector space} $ X $.
    Then, $ S $ is compact if and only if $ S $ is closed and bounded.
\end{thm}

The natural next step would be to investigate the infinite-dimensional case. However, we see that
the theorem holding even then would be too good to be true:

\begin{xmp}
    Consider the space $ (C[0,1], \norm{\cdot}_{\infty}) $.
    We show that the closed unit ball is not compact. \vsp
    %
    Indeed, we can take a sequence $ (x_{n})_{n\geq1} $ as $ x_{n} = x^{n} $.
    Clearly on $ [0, 1] $, the supremum of each term is $ 1 $.
    However, we can see that this sequence does not have a convergent subsequence.
\end{xmp}
\vspace{-0.2in}
Lastly, we can show one final equivalence - an alternate characterization of compactness which
warrants the explicit statement of the definition being the ``open cover" definition at the start
of this paper.

\begin{thm}
    A metric space is compact if and only if it is clustering.
\end{thm}
\vspace{-0.2in}
\begin{pf}
    $ (\implies) $ This direction was proven by the clustering group. \vsp
    %
    $ (\Longleftarrow) $ Note that it suffices to show that any open cover has a countable
    subcover. Suppose $ X $ is a clustering metric space. Let $ \set{U_{i}} $ be an open cover of
    $ X $. Assume for the sake of contradiction that $ \set{U_{i}} $ has no countable subcover.
    Since $ X $ is clustering, then we have a Lebesgue number $ \delta $. \vsp
    %
    Let $ (p_{n}) $ be a sequence in $ X $, defined as:
    \begin{equation*}
        p_{1} \in X \qquad p_{n} \in X \setminus \bigcup_{i=1}^{n-1}B(p_{i}, \delta)
    \end{equation*}
    Since $ X $ is clustering, then $ p_{n} \rightarrow p $ for some $ p \in X $. Note that the
    sequence convering implies that the sequence is Cauchy, so for any $ \ep > 0 $, there exists
    $ N > 0 $ such that for all $ m, n > N $, we have that $ d(p_{m}, p_{n}) < \ep $. \vsp
    %
    But for any $ \ep < \delta $, this directly contradicts the definition of the sequence.
    Therefore, $ \set{U_{i}} $ must have a countable subcover, which suffices to show that $ X $
    is compact as needed.
\end{pf}
