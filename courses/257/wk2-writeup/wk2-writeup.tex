\documentclass{article}
\usepackage{preamble}
\usepackage{env}
\usepackage{configure}

% available environments:
% theorem: thm
% definition: defn
% proof: pf
% corollary: crll
% lemma: lm
% question: qu
% solution: soln
% example: xmp
% exercise: exr
%
% options: title=<title>   {all}
%          source=<source> {pf, qu, soln, xmp, exr}  Note: if content is taken directly from the main resource, cite the main resource as ``Primary source material"


% define these variables!
\def\coursecode{}
\def\coursename{} % use \relax for non-course stuff
\def\studytype{} % 1: Personal Self-Study Notes / 2: Course Lecture Notes / 3: Revised Notes / 4: Exercise Solution Sheet
\def\author{\me}
\def\createdate{}
\def\updatedate{\today}
\def\source{} % name, ed. of textbook, or `Class Lectures` for class notes
\def\sourceauthor{} % for class notes, put lecturer
\def\leftmark{Week 2 - Topology} % set text in header; should only be necessary in assignments etc.
\pagenumbering{arabic} % force revert numbering to default; should only be necessary in assignments etc.

\begin{document}
\setcounter{subsection}{1}

\begin{thm}
    Let $ (X, d_{X}), (Y, d_{Y}) $ be metric spaces,
    and $ A \subseteq X, B \subseteq Y $ subsets.
    Then, $ \overline{A \times B} = \overline{A} \times \overline{B} $.
\end{thm}

\begin{pf}
    We prove by double subset inclusion. \npgh

    First, we show that $ \overline{A \times B} \subseteq \overline{A} \times \overline{B} $. (Daniel) \npgh
    
    Let $ (a, b) \in \overline{A \times B} $ be given.
    We must show that $ (a, b) \in \overline{A} \times \overline{B} $,
    for which it suffices to prove that given any $ \varepsilon > 0 $,
    the sets $ B_{\varepsilon}(a) \cap A $ and $ B_{\varepsilon}(b) \cap B $ are non-empty. \vsp
    %
    Let $ \varepsilon > 0 $ be given.
    Take $ (p, q) \in B_{\varepsilon}((a,b)) \cap (A \times B) $. \\
    Then we have that:
    \begin{equation*}
        d_{X \times Y}((a,b), (p,q)) = d_X(a, p) + d_Y(b, q) < \varepsilon
    \end{equation*}
    Rearranging, and using non-negativity of distance functions we get that:
    \begin{align*}
        & d_X(a, p) < \varepsilon - d_Y(b, q) \\
        \implies \ & d_X(a, p) < \varepsilon
    \end{align*}
    and
    \begin{align*}
        & d_Y(b, q) < \varepsilon - d_X(a, p) \\
        \implies \ & d_Y(b, q) < \varepsilon
    \end{align*}
    from which it follows that $p\in B_{\varepsilon}(a)\cap A$ and $q\in B_{\varepsilon}(b)\cap B$.
    This completes the proof. \npgh

    Next, we show that $ \overline{A} \times \overline{B} \subseteq \oline{A \times B} $. (Emerald) \npgh
    
    Let $ (a, b) \in \oline{A} \times \oline{B} $.
    Then, for all $ \ep > 0 $, we have that $ B(a, \frac{\ep}{2}) \cap A \neq \varnothing $,
    and $ B(b, \frac{\ep}{2}) \cap B \neq \varnothing $.
    We want to show that $ (a, b) \in \oline{A \times B} $. \vsp
    %
    To do this, choose $ p \in B(a, \frac{\ep}{2}), q \in B(b, \frac{\ep}{2}) $.
    We see that:
    \begin{align*}
        & d_{X \times Y}((a, b), (p, q)) \\
        = \ & d_{X}(a, p) + d_{Y}(b, q) \\
        < \ & \frac{\ep}{2} + \frac{\ep}{2} \\
        = \ & \ep
    \end{align*}
    Therefore, we have that $ (p, q) \in B((a, b), \ep) $.
    Since this is true for all $ \ep > 0 $, we therefore conclude that
    $ B_{\ep}(a, b) \cap A \times B \neq \varnothing $ for all $ \ep > 0 $.
    Therefore, $ (a, b) \in \oline{A \times B} $ as needed.
\end{pf}

\end{document}
