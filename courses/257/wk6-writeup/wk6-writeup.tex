\documentclass{article}
\usepackage{preamble}
\usepackage{env}
\usepackage{configure}

% available environments:
% theorem: thm
% definition: defn
% proof: pf
% corollary: crll
% lemma: lm
% question: qu
% solution: soln
% example: xmp
% exercise: exr
%
% options: title=<title>   {all}
%          source=<source> {pf, qu, soln, xmp, exr}  Note: if content is taken directly from the main resource, cite the main resource as ``Primary source material"


% define these variables!
\def\coursecode{}
\def\coursename{} % use \relax for non-course stuff
\def\studytype{} % 1: Personal Self-Study Notes / 2: Course Lecture Notes / 3: Revised Notes / 4: Exercise Solution Sheet
\def\author{\me}
\def\createdate{}
\def\updatedate{\today}
\def\source{} % name, ed. of textbook, or `Class Lectures` for class notes
\def\sourceauthor{} % for class notes, put lecturer
\def\leftmark{Week 6 - Equivalence} % set text in header; should only be necessary in assignments etc.
\pagenumbering{arabic} % force revert numbering to default; should only be necessary in assignments etc.

\begin{document}

\setcounter{subsection}{6}
\setcounter{exr}{30}

\begin{exr}
    Let $ (X, d) $ be a metric space.
    Prove that if $ d $ is topologically equivalent to the discrete metric,
    then $ (X, d) $ has no infinite compact subsets.
\end{exr}

\begin{pf}
    Suppose $ d \sim d_{\trm{disc}} $.
    Note that this implies that every subset is open; in particular, singleton sets are open. \vsp
    %
    Let $ A \subseteq X $ be an infinite subset.
    We can write $ A $ as:
    \begin{equation*}
        A \ = \ \bigcup_{x \in A} \set{x}
    \end{equation*}
    In particular, since each singleton set is open, then this is an open cover of $ A $.
    However, it clearly has no finite subcover, so $ A $ cannot be compact. \vsp
    %
    Since this holds true for any infinite subset, then $ X $ has no infinite compact subsets.
\end{pf}

\end{document}
