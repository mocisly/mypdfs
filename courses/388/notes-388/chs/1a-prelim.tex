
\subsection{Preliminaries}
\lecdate{Lec 1 - Sep 2 (Week 1)}

main question of the course: how many ways can we embed a circle into $\bR^{3}$?
we'll consider this up to ambient diffeo/homeo, or up to isotopy (comes later).

for example, we claim that there is no homeo $\vphi:\bR^{3}\sto\bR^{3}$ such
that $\vphi(U)=T$, from the unknot to the trefoil.

this gives a distinction between knotted and unknotted, but non-trivial knots
can also be distinct.

[theres an example here, but i dont have my stylus :)]

so we need some way to identify different knots, such as a particular property.
what can we use, how do we think about these distinctions?
note that there are infinitely many distinct, but we're really asking about the
underlying structure.

\begin{defn}
    a \textbf{knot diagram} is an \textit{immersion} (to be defined) of the
    circle $S^{1}\ito\bR^{2}$, so that:
    \begin{itemize}
        \item All non-injective points are 2:1 and (self-)transverse
        \item Add over/under-crossing data at all double points
    \end{itemize}
\end{defn}

[again, an example, but my stylus....]

\begin{defn}
    an \textbf{immersion} is a non-vanishing (full rank) derivative at all
    points. [precisely, it is a differentiable map whose pushforward is inj]
\end{defn}

\begin{prop}
    given any $C^{\infty}$ (or just $C^{1}$) embedding $K:S^{1}\ito\bR^{3}$,
    a ``generic" linear projection of $K(S^{1})$ gives a knot diagram.

    conversely, any knot diagram defines a knot $K(S^{1})\subseteq\bR^{3}$,
    which is unique up to ``isotopy".
\end{prop}


\begin{pf}[source=Primary Source Material]
    (sketch)(very sketchy)

    consider $K(S^{1})\subseteq\bR^{3}$; it has a tangent vector everywhere.
    consider $\set{\trm{directions}}\subseteq S^{2}$.
    this is a smooth map $S^{1}\sto S^{2}$; in particular, it is not surjective
    by Sard's theorem.
    as a consequence, the complement of the image is full measure (open + dense).

    pick a pt not in the image and project in this direction.
    the tangent vector to $K$ is thus never parallel, so the projection is an
    immersion.

    other issues: possibly self-tangent, possibly $n:1$.
    but, none of these are generic (i.e. do a dimension count, apply Sard's)

    for the converse:
    let $x,y$ coords be the coords in the projection.
    the indeterminacy is $z(\theta)$, since $(x(\theta),y(\theta))\in\bR^{2}$
    are determined by the diagram.

    crossings are then double pts
    $\set{\theta_{1}^{+},\theta_{1}^{-}},\set{\theta_{2}^{+},\theta_{2}^{-}}$
    as pairs of pts in $S^{1}$, say up to $k$ pts.
    (necessarily finite since they are necessarily isolated -> cpt -> finite,
    or sth)

    then choice is(of?) a function $z:S^{1}\sto\bR$ such that
    $z(\theta_{j}^{+})>z(\theta_{j}^{-})$ - this represents which strand is above
    the other, in a sense.

    the space of all functions $z$ is then convex and thus connected:
    \begin{equation*}
        z_{t}(\theta)=tz(\theta)+(1-t)\tilde{z}(\theta)
    \end{equation*}
    is a valid choice of $z$.
    so there are many possible choices, but all can be interpolated, thus are
    isotopic(?).
\end{pf}

\begin{defn}
    a \textbf{tri-colouring} of a knot diagram is a choice of colour in
    $\set{\trm{RGB}}$ for each ``arc" in the diagram, such that at each crossing,
    either \textit{one} or \textit{three} colours are used[meet].
    it is required to use all three colours.
\end{defn}

for instance, the trefoil can very easily be tri-coloured, as well as the
``$6_{1}$ knot", or more generally a ($k$-th) ``twist" knot.
as a non-example, the ``figure-8" knot and unknot have no tri-colouring.

it seems like this only depends on the diagram, but in fact it only depends on
a knot up to smooth isotopy. why? what?

(non-clarifying proof/explanation) we use a black box: reidemeister's theorem

two diagrams present isotopic knots in $\bR^{3}$ iff they differ by a finite
sequence of these moves:
\begin{itemize}
    \item R1 - twisting a strand
    \item R2 - moving two non-intersecting strands atop each other
    \item R3 - moving a strand behind a crossing if it is
        ``under both strands"
\end{itemize}
for example: [example]

pf of tri-colourability:
note that for each move, if there is a chosen tri-colouring before the move,
there is a unique tri-colouring after.

ok, but what's actually happening?
real answer: consider the group homomorphisms
$\pi_{1}(\bR^{3}\sm K(S^{1}))\sto S_{3}$.

sps $K\subseteq\bR^{3}$ is a knot (abusing notation).
if $K$ isotopic to $\tilde{K}$, or if there is a  homeo/diffeo
$\vphi:\bR^{3}\sto\bR^{3}$ with $\vphi(K)=\tilde{K}$, then
$\bR^{3}\sm K\simeq\bR^{3}\sm\tilde{K}$.
thus, we can equivalently as questions about the complements; are these the same?
what about $\pi_{1}(\bR^{3}\sm K)$?
this will give us our first powerful invariant for knots.

note for $K=U$ the unknot, we can draw a knot ``parallel" to the unknot on the
inside, and this is homotopic to $U$ with $\pi_{1}(\bR^{3}\sm K)\simeq\bZ$.
but for the trefoil, this may not be the case - a parallel curve may not be
homotopic to $T$.

this seems to distinguish $U$ and $T$, maybe, but how to define anything?

lets look at $U\subseteq\bR^{3}$. let $x_{0}$ be your eyeball.
if $f:[0,1]\sto\bR^{3}\sm U$ never passes behind the knot, then taking all the
light rays from $f(\theta)$ to $x_{0}$ defines a homotopy $H$, so $[f]=[e]$.
otherwise, if it passes through $U$, then we get a winding number, so
$\pi_{1}(\bR^{3}\sm U)\simeq\bZ$.

for a more complicated diagram, we have a group elem for each \textit{arc} of the
diagram. however, given a loop around some arc, we can slide it ``along" the arc
as long as we don't accidentally cross over any other arcs.

so we'd have the following relation:
[more diagrams]

this is known as the \textbf{wirtinger presentation} - we will see the proof
next time.

