
\subsection{Preliminaries}
\lecdate{Lec 1 - Sep 2 (Week 1)}

main question of the course: how many ways can we embed a circle into $\bR^{3}$?
we'll consider this up to ambient diffeo/homeo, or up to isotopy (comes later).

for example, we claim that there is no homeo $\vphi:\bR^{3}\mto\bR^{3}$ such
that $\vphi(U)=T$, from the unknot to the trefoil.

this gives a distinction between knotted and unknotted, but non-trivial knots
can also be distinct.

[theres an example here, but i dont have my stylus :)]

so we need some way to identify different knots, such as a particular property.
what can we use, how do we think about these distinctions?
note that there are infinitely many distinct, but we're really asking about the
underlying structure.

\begin{defn}
    a \textbf{knot diagram} is an \textit{immersion} (to be defined) of the
    circle $S^{1}\ito\bR^{2}$, so that:
    \begin{itemize}
        \item All non-injective points are 2:1 and (self-)transverse
        \item Add over/under-crossing data at all double points
    \end{itemize}
\end{defn}

[again, an example, but my stylus....]

\begin{defn}
    an \textbf{immersion} is a non-vanishing (full rank) derivative at all
    points. [precisely, it is a differentiable map whose pushforward is inj]
\end{defn}

\begin{prop}
    given any $C^{\infty}$ (or just $C^{1}$) embedding $K:S^{1}\ito\bR^{3}$,
    a ``generic" linear projection of $K(S^{1})$ gives a knot diagram.

    conversely, any knot diagram defines a knot $K(S^{1})\subseteq\bR^{3}$,
    which is unique up to ``isotopy".
\end{prop}


\begin{pf}[source=Primary Source Material]
    (sketch)(very sketchy)

    consider $K(S^{1})\subseteq\bR^{3}$; it has a tangent vector everywhere.
    consider $\set{\trm{directions}}\subseteq S^{2}$.
    this is a smooth map $S^{1}\mto S^{2}$; in particular, it is not surjective
    by Sard's theorem.
    as a consequence, the complement of the image is full measure (open + dense).

    pick a pt not in the image and project in this direction.
    the tangent vector to $K$ is thus never parallel, so the projection is an
    immersion.

    other issues: possibly self-tangent, possibly $n:1$.
    but, none of these are generic (i.e. do a dimension count, apply Sard's)

    for the converse:
    let $x,y$ coords be the coords in the projection.
    the indeterminacy is $z(\theta)$, since $(x(\theta),y(\theta))\in\bR^{2}$
    are determined by the diagram.

    crossings are then double pts
    $\set{\theta_{1}^{+},\theta_{1}^{-}},\set{\theta_{2}^{+},\theta_{2}^{-}}$
    as pairs of pts in $S^{1}$, say up to $k$ pts.
    (necessarily finite since they are necessarily isolated -> cpt -> finite,
    or sth)

    then choice is(of?) a function $z:S^{1}\mto\bR$ such that
    $z(\theta_{j}^{+})>z(\theta_{j}^{-})$ - this represents which strand is above
    the other, in a sense.

    the space of all functions $z$ is then convex and thus connected:
    \begin{equation*}
        z_{t}(\theta)=tz(\theta)+(1-t)\tilde{z}(\theta)
    \end{equation*}
    is a valid choice of $z$.
    so there are many possible choices, but all can be interpolated, thus are
    isotopic(?).
\end{pf}

\begin{defn}
    a \textbf{tri-colouring} of a knot diagram is a choice of colour in
    $\set{\trm{RGB}}$ for each ``arc" in the diagram, such that at each crossing,
    either \textit{one} or \textit{three} colours are used[meet].
    it is required to use all three colours.
\end{defn}

for instance, the trefoil can very easily be tri-coloured, as well as the
``$6_{1}$ knot", or more generally a ($k$-th) ``twist" knot.
as a non-example, the ``figure-8" knot and unknot have no tri-colouring.

it seems like this only depends on the diagram, but in fact it only depends on
a knot up to smooth isotopy. why? what?

(non-clarifying proof/explanation) we use a black box: reidemeister's theorem

two diagrams present isotopic knots in $\bR^{3}$ iff they differ by a finite
sequence of these moves:
\begin{itemize}
    \item R1 - twisting a strand
    \item R2 - moving two non-intersecting strands atop each other
    \item R3 - moving a strand behind a crossing if it is
        ``under both strands"
\end{itemize}
for example: [see diagram 1]

pf of tri-colourability:
note that for each move, if there is a chosen tri-colouring before the move,
there is a unique tri-colouring after.

ok, but what's actually happening?
real answer: consider the group homomorphisms
$\pi_{1}(\bR^{3}\sm K(S^{1}))\mto S_{3}$.

sps $K\subseteq\bR^{3}$ is a knot (abusing notation).
if $K$ isotopic to $\tilde{K}$, or if there is a  homeo/diffeo
$\vphi:\bR^{3}\mto\bR^{3}$ with $\vphi(K)=\tilde{K}$, then
$\bR^{3}\sm K\simeq\bR^{3}\sm\tilde{K}$.
thus, we can equivalently as questions about the complements; are these the same?
what about $\pi_{1}(\bR^{3}\sm K)$?
this will give us our first powerful invariant for knots.

note for $K=U$ the unknot, we can draw a knot ``parallel" to the unknot on the
inside, and this is homotopic to $U$ with $\pi_{1}(\bR^{3}\sm K)\simeq\bZ$.
but for the trefoil, this may not be the case - a parallel curve may not be
homotopic to $T$.

this seems to distinguish $U$ and $T$, maybe, but how to define anything?

lets look at $U\subseteq\bR^{3}$. let $x_{0}$ be your eyeball.
if $f:[0,1]\mto\bR^{3}\sm U$ never passes behind the knot, then taking all the
light rays from $f(\theta)$ to $x_{0}$ defines a homotopy $H$, so $[f]=[e]$.
otherwise, if it passes through $U$, then we get a winding number, so
$\pi_{1}(\bR^{3}\sm U)\simeq\bZ$.

for a more complicated diagram, we have a group elem for each \textit{arc} of the
diagram. however, given a loop around some arc, we can slide it ``along" the arc
as long as we don't accidentally cross over any other arcs.

so we'd have the following relation:
[see diagram 2]

this is known as the \textbf{wirtinger presentation} - we will see the proof
next time.

\lecdate{Lec 2 - Sep 4 (Week 2)}
on quercus - notes for $C^{\infty}$ topology.

coming up, we want to discuss van kampen's theorem - to do this, we need to
discuss free products/quotients, or ``amalgamated free products".

\begin{defn}
    let $G_{1}, G_{2}, H$ be groups and $f_{i}:H\mto G_{i}$ two group homos.
    we define the free product $G_{1}*_{H}G_{2}$ in two ways: \vspace{-2mm}
    \begin{itemize}
        \item the ``correct" defn coming from category thy says that for any
            maps $g_{i}:G_{i}\mto K$ s.t. $g_{1}\circ f_{1}=g_{2}\circ f_{2}$,
            then there exists a unique $\vphi:G_{1}*_{H}G_{2}\mto K$ such that
            the maps ``factor", that is:
            \begin{equation*}
                g_{j}=\vphi\circ\iota_{j}
            \end{equation*}
            where $\iota_{j}$ are the inclusion maps [see diagram 3]
        \item the ``useful" defn says that if $G_{1}$ has the presentation
            \begin{equation*}
                G_{1}=\ang{\gamma_{1}\cdots\gamma_{k}|r_{1}\cdots r_{\ell}}
            \end{equation*}
            with $r_{j}=1$ in $G_{1}$ and
            $G_{1}=\trm{Free}(\set{\gamma_{j}})/\set{r_{j}=1}$, and similarly
            \begin{equation*}
                G_{2}=\ang{\delta_{1}\cdots\delta_{n}|s_{1}\cdots s_{m}}
            \end{equation*}
            then:
            \begin{gather*}
                G_{1}*_{H}G_{2}=
                \ang{\gamma_{1}\cdots\gamma_{k}\delta_{1}\cdots\delta_{n}
                    \Bigg\rvert
                \begin{tabular}{C}
                    r_{1}\cdots r_{\ell} \\
                    s_{1}\cdots s_{m} \\
                    f_{1}(a_{1})\cdot f_{2}(a_{1})^{-1} \\
                    f_{1}(a_{2})\cdot f_{2}(a_{2})^{-1} \\
                    \vdots
                \end{tabular}} \\
                H=\ang{a_{1}\cdots a_{\alpha}|t_{1}\cdots t_{\beta}}
            \end{gather*}
            since $F_{1}(t_{i})=1$ in $G_{1}$.
    \end{itemize}
\end{defn}
lmao.

\begin{thm}[title=Van Kampen's Theorem]
    Let $U_{1}\cup U_{2}=X$ for open pconn $U_{i}$ s.t.
    $U_{1}\cap U_{2}$ is connected, $x_{0}\in U_{1}\cap U_{2}$.

    Then $\pi_{1}(X,x_{0})\simeq\pi_{1}(U_{1},x_{0})*_{H}\pi_{1}(U_{2},x_{0})$,
    where $H=\pi_{1}(U_{1}\cap U_{2},x_{0})$.
\end{thm}
proof sketch:

let $f:[0,1]\mto X$ where $f\in\Omega(X,x_{0})$.
we want to find finitely many $t_{1},\dots,t_{m}\in[0,1]$
such that $f\rvert_{[t_{j},t_{j+1}]}$ is entirely in $U_{1}$ or $U_{2}$
[see diagram 4].
note that these $t_{j}$ are different than the $t_{i}$ in [the defn].

take paths between $f(t_{j})$ and $x_{0}$ inside $U_{1}\cap U_{2}$.
then by homotopy, $[f]$ is in the smallest subgroup containing $\pi_{1}(U_{1})$
and $\pi_{1}(U_{2})$.

for any $[\gamma]\in\pi_{1}(U_{1}\cap U_{2})$, we see the same thing in
$\pi_{1}(X)$. [see diagram 5]
thus, we have a surjection $\pi_{1}(U_{1})*_{H}\pi_{1}(U_{2})\sto\pi_{1}(X)$,
where $H=\pi_{1}(U_{1}\cap U_{2})$.
it remains to show that the kernel is 0 - left as ``exercise".

this defn is used for an example below:
\begin{defn}
    the \textbf{genus} $g$ \textbf{handle body} is the bounded 3D region inside
    the surface as shown in [diagram 6] with examples.
\end{defn}

we show some examples of applications(?).
\begin{itemize}
    \item $\pi_{1}(S^{1})\simeq\bZ$, not proved with van kampen but a different
        way.
    \item $\pi_{1}(X\times Y)\simeq \pi_{1}(X)\oplus\pi_{1}(Y)$.
        pf: any $f:I\gto X\times Y$ is $f_{1}:I\gto X$ and $f_{2}:I\gto Y$,
        with $f=(f_{1},f_{2})$ - same for homotopies.
    \item $\pi_{1}(B^{n})=\set{e}$.
        pf: homotope any loop to $0$ along radial lines.
    \item $\pi_{1}(S^{1}\times D^{2})\simeq\bZ$ - note this is the solid torus.
    \item $\pi_{1}(H_{g})=\trm{Free}_{g}$.
        pf: for $g=2$, denote each handle by $U_{i}$, then $U_{1}\cap U_{2}
        \simeq B^{3}$.
        by VK, $\pi_{1}(H_{g})\simeq\bZ*_{\set{e}}\bZ\simeq\bZ*\bZ\simeq F_{2}$.
        then for $g>2$, induct. see [diagram 7].
    \item $\pi_{1}(S^{3})\simeq\set{e}$.
        pf: $\pi_{1}(S^{3})\simeq\bZ*_{H}\bZ$, where $H=\bZ\oplus\bZ$.
        note we have that
        $\bZ\oplus\bZ\simeq\pi_{1}(S^{1}\times S^{1}\times(-\ep,\ep))$ -
        see below
\end{itemize}
recall that $S^{n}\sm\set{x_{0}}\simeq\bR^{n}$ by stereographic projection.
we claim $S^{3}=H_{1}\cup_{S^{1}\times S^{1}}H_{1}$.
we'll prove this in two different ways (maam there are 5 minutes left).
[not an adjoint space, just notation - unioning ``by their boundaries" or
``over their boundaries" or sth. actually it may have been an adjoint space]

take $S^{3}\subseteq\bC^{2}$. then
\begin{equation*}
    S^{3}=\set{\abs{z_{1}}^{2}+\abs{z_{2}}^{2}=1}
    =\set{\abs{z_{1}}^{2}\leq\frac{1}{2},\abs{z_{2}}^{2}=1-\abs{z_{1}}^{2}}
\end{equation*}
note that in the second set, the first portion gives $D^{2}$ and the second gives
$S^{1}$, so we have $D^{2}\times S^{1}$.
unioning this with the same set with swapped coordinates gives the result.

for the second way, see diagram 8.



