\subsection{separation axioms i guess}

\lecdate{Lec 13 - Jul 2 (Week 8)}

we start with ``weaker" separation axioms.
\begin{itemize}
    \item $\msf{T}_{0}$: for any distinct pts, $\exists \, $ open set containing
        one but not the other
        \begin{itemize}
            \item also known as ``Kolmogorov space"
        \end{itemize}
    \item $\msf{T}_{1}$: for distinct pts, $\exists \, $ open $U,V\subseteq X$
        with $x\in U\sm V,y\in V\sm U$
        \begin{itemize}
            \item also known as ``Fre\'echet space"
        \end{itemize}
    \item $\msf{T}_{2}$: usual hausdorff
\end{itemize}
clearly $\msf{T}_{2}\implies\msf{T}_{1}\implies\msf{T}_{0}$.
on $\bR$, examples of $\msf{T}_{1}$ but not $\msf{T}_{2}$ include cofinite and
coctbl. examples of $\msf{T}_{0}$ but not $\msf{T}_{1}$ include ray and
particular pt.

\begin{prop}
    if $X$ is $\msf{T}_{1}$, constant seqs cvg to a unique pt.
\end{prop}

\begin{pf}[source=Primary Source Material]
    sps $a_{n}=x$ for all $n$ and $a_{n}\sto y\neq x$.
    by $\msf{T}_{1}$, there are open $U,V\subseteq X$ with $x\in U\sm V$ and
    $y\in V\sm U$. but $a_{n}\notin V$, contradicting $a_{n}\sto y$.
\end{pf}
this is not true for $\msf{T}_{0}$ spaces;
for instance, in $(\bR,\cl{T}_{p})$, the seq $a_{n}=p$ cvgs to every pt.

\newpage
\begin{prop}
    tfae
    \begin{enumerate}[i)]
        \item $X$ is $\msf{T}_{1}$
        \item singletons are closed
        \item finite sets are closed
        \item $\forall \, A\subseteq X,A=\displaystyle\bigcap_{A\subseteq U}U$
    \end{enumerate}
\end{prop}
exercise.

\begin{prop}
    if $X$ is $\msf{T}_{1}$, then $x\in A'$ iff every open $x\in U$ intersects
    $A$ infinitely
\end{prop}
see pp4 sth sth

we move on to strong separation axioms.

\begin{defn}
    a space $X$ is \textbf{regular} if for every closed $F\subseteq X$ and
    $x\notin F$, there are open sets $U,V\subseteq X$ s.t. $x\in U,F\subseteq V,
    U\cap V=\eset$.

    we define a space as $\bm{\msf{T}_{3}}$ if it is reg and $\msf{T}_{1}$.
\end{defn}
example: indiscrete on $\bR$ is (vacuously) regular but not hausdorff.
note $\msf{T}_{3}\implies\msf{T}_{2}$.

\newpage
\begin{xmp}[source=Primary Source Material]
    \vspace{-0.35in}
    \begin{itemize}
        \item discrete spaces
        \item metric spaces
        \item $\bR_{K}$ is $\msf{T}_{2}$ but not $\msf{T}_{3}$
    \end{itemize}
\end{xmp}

\begin{prop}
    if $X$ reg, then the weak axioms are equivalent
\end{prop}

\begin{pf}[source=Primary Source Material]
    we show $\msf{T}_{0}\implies\msf{T}_{2}$.
    let $x\neq y$; by $\msf{T}_{0}$, there is open $x\in U,y\notin U$.
    then $U^{c}$ closed; by reg, find $U_{0},V$ open with
    \begin{equation*}
        x\in U_{0},U^{c}\subseteq V,U_{0}\cap V=\eset
    \end{equation*}
    then $x\in U_{0},y\in V, U_{0}\cap V=\eset$.
\end{pf}
note: authors don't agree on defns for $\msf{T}_{3}$ and regular; munkres calls
as ``regular" our $\msf{T}_{3}$.

\begin{prop}
    $X$ reg iff for any open $x\in U\subseteq X$, there is open $V\subseteq X$
    with $x\in V\subseteq\bar{V}\subseteq U$.
\end{prop}

\begin{pf}[source=Primary Source Material]
    fix $x\in U$.
    by reg, let $U^{c}\subseteq U_{0}$ and $x\in V$ open with
    $U_{0}\cap V=\eset$.
    note $\bar{V}\subseteq U$ (or $\bar{V}\cap F=\eset$).
    \begin{block}
        if $y\in\bar{V}\cap F$, then $y\in U_{0}$.
        but $U_{0}\cap V=\eset$, contradicting $y\in\bar{V}$.
    \end{block}
    this proves the forward direction.

    fix $F$ closed and $x\notin F$. then $x\in F^{c}$.
    by assumption, there is open $V$ with:
    \begin{equation*}
        x\in V\subseteq\bar{V}\subseteq U
    \end{equation*}
    then, note $(\bar{V})^{c}\cap V=\eset$.
    since $F\cap\bar{V}=\eset$, then $F\subseteq U_{0}$.
\end{pf}

\begin{prop}
    cpt hausdorff implies regular
\end{prop}

\begin{pf}[source=Primary Source Material]
    fix closed $F$ and $x\notin F$. then $F$ cpt.
    for any $y\in F$, we can find $U_{y},V_{y}$ open s.t.:
    \begin{equation*}
        x\in U_{y} \quad y\in V_{y} \quad U_{y}\cap V_{y}=\eset
    \end{equation*}
    by cpt, there is finite $F_{0}\subseteq F$ s.t.:
    \begin{equation*}
        F\subseteq \bigcup_{y\in F_{0}}V_{y} =: V
    \end{equation*}
    let $U=\bigcap_{F_{0}}U_{y}$. we claim $U\cap V=\eset$.
    \begin{block}
        if $z\in U\cap V$ then $z\in V_{y}$ for some $y$.
        but then $z\in U_{y}\cap V_{y}$, contradicting $U_{y}\cap V_{y}=\eset$.
    \end{block}
    so $U\cap V=\eset$ as needed.
\end{pf}

\begin{defn}
    we say a space is \textbf{normal} if for every closed disjoint $E,F$ there
    are open $U,V$ with $E\subseteq U,F\subseteq V,U\cap V=\eset$.

    we define a space as $\bm{\msf{T}_{4}}$ if it is normal and $\msf{T}_{1}$.
\end{defn} \

\begin{xmp}[source=Primary Source Material]
    $X=\set{0,1}$ with $\cl{T}=\set{\eset,\set{0},X}$ is normal but not
    $\msf{T}_{2}$.
    this is because given two disjoint closed sets, one must be empty.

    note $X$ is $\msf{T}_{0}$, so it is not enough for the defn.
    also $\msf{T}_{4}\implies\msf{T}_{3}$.
\end{xmp} \

\begin{xmp}[source=Primary Source Material]
    \vspace{-0.35in}
    \begin{itemize}
        \item discrete spaces are normal (thus T4)
        \item metric spaces are normal (thus T4)
        \item $\bR_{\ell}$ is normal (thus T4)
        \item $\bR_{\ell}^{2}$ is reg (reg is fin. productive) but not normal
    \end{itemize}
\end{xmp}

\begin{prop}
    normall iff for every open $U$, closed $V$ with $V\subseteq U$,
    there exists open $W\subseteq X$ s.t.
    $V\subseteq W\subseteq\bar{W}\subseteq U$.
\end{prop}
pf: pp7

\begin{prop}
    cpt hausdorff implies normal
\end{prop} \

\begin{pf}[source=Primary Source Material]
    fix $E,F$ disj closed.
    since $X$ reg, for every $x\in E$ we have open $U_{x},V_{x}$ s.t.:
    \begin{equation*}
        x\in U_{x} \quad F\subseteq V_{x} \quad U_{x}\cap V_{x}=\eset \quad
        E\subseteq \bigcup_{x\in E}U_{x}
    \end{equation*}
    furthermore, $E$ cpt.
    thus there is finite $E_{0}\subseteq E$ s.t. bla bla. define:
    \begin{equation*}
        U:=\bigcup_{x\in E_{0}}U_{x} \qquad V:=\bigcap_{x\in E_{0}}V_{x}
    \end{equation*}
\end{pf}
regular is hereditary and fin. productive, but normality is neither.
SPACING AUGH


