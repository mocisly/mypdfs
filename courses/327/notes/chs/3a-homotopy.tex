\subsection{Path Homotopy}
\lecdate{Lec 19 - Jul 23 (Week 11)}
ALGTOP RAAAHHHHHHHHHHHHH

\begin{defn}
    given $\gamma_{0},\gamma_{1}$ from $x$ to $y$, a \textbf{path homotopy}
    from $\gamma_{0}$ to $\gamma_{1}$ is a cts $F:[0,1]^{2}\sto X$ such that
    the following holds:
    \begin{equation*}
        F(s,0)=\gamma_{0}(s) \trm{ and } F(s,1)=\gamma_{1} \qquad
        F(0,t)=x \trm{ and } F(1,t)=y
    \end{equation*}
    we say $\gamma_{0}$ is \textbf{path homotopic} to $\gamma_{1}$,
    or $\gamma_{0}\simeq_{p}\gamma_{1}$, if a pathtopy exists
\end{defn}
[pathtopy is certainly a. choice.]
note that its important for $F$ to be cts; this is strictly stronger than
requiring $F$ to be cts in each coordinate (beloved $xy/x^{2}+y^{2}$).

\begin{xmp}[source=Primary Source Material]
    $A\subseteq\bR^{n}$ cvx $\implies$ any two paths w/ same endpoints are
    homotopic: for fixed $s$, take $F(s,t)=(1-t)\gamma_{0}(s)+t\gamma_{1}(s)$.
    check that this is a pathtopy.
\end{xmp}

we can generalize pathtopy to deformations of general cts functions.
\begin{defn}
    sps $f,g:X\sto Y$ cts.
    a \textbf{homotopy} from $f$ to $g$ is a cts $F:X\times[0,1]\sto Y$ with
    \begin{equation*}
        F(x,0)=f(x) \qquad F(x,1) = g(x)
    \end{equation*}
    we write $f\simeq g$ in this case.
\end{defn}
clearly, both $\simeq, \simeq_{p}$ are equiv rels.
note transitivity uses pasting lemma [technically. its like a single pt tho].

\subsection{The Fundamental Group}
[yeah this should go here tbh]

\begin{defn}
    for a path $\gamma_{0}$ from $x$ to $y$ and $\gamma_{1}$ from $y$ to $z$,
    define:
    \begin{equation*}
        \gamma_{0}*\gamma_{1}(s) =
        \begin{cases}
            \gamma_{0}(2s) & s \in[0,1/2] \\
            \gamma_{1}(2s-1) & s \in[1/2,1]
        \end{cases}
    \end{equation*}
    this induces an operation $*$ on the set of equivalence classes.
\end{defn} \

\begin{prop}
    $*$ is well-defined on equivalence classes.
\end{prop} \

\begin{pf}[source=Primary Source Material]
    fix $\gamma_{0}\simeq\gamma_{0}'$ and $\gamma_{1}\simeq\gamma_{1}'$.
    let $F,G:[0,1]^{2}\sto X$ be pathtopies from $\gamma_{i}$ to $\gamma_{i}'$
    respectively. consider:
    \begin{equation*}
        H(s,t)=
        \begin{cases}
            F(2s,t) & (s,t)\in[0,1/2]\times[0,1] \\
            G(2s-1,t) & (s,t)\in[1/2,1]\times[0,1]
        \end{cases}
    \end{equation*}
    it is easy to see that $H$ is then a homotopy from $\gamma_{0}*\gamma_{1}$
    to $\gamma_{0}'*\gamma_{1}'$.
\end{pf} \

\begin{defn}
    a \textbf{loop} is a path $\gamma:[0,1]\sto X$ with $\gamma(0)=\gamma(1)$.
    we say that $\gamma$ is a loop at $x_{0}$ if $\gamma(0)=\gamma(1)=x_{0}$.

    given a fixed $x_{0}$, we denote by $\pi_{1}(X,x_{0})$ the set of all
    pathtopy equiv classes of loops at $x_{0}$.
\end{defn}
we define by $e_{x}:[0,1]\sto X$ and $\bar{\gamma}:[0,1]\sto X$ as:
\begin{equation*}
    e_{x}(s)=x \qquad \bar{\gamma}(s)=\gamma(1-s)
\end{equation*}
these are the ``constant" and ``inverse" paths respectively.

\begin{defn}
    given $\gamma$, let $\vphi:[0,1]\sto[0,1]$ be cts with $\vphi(0)=0$ and
    $\vphi(1)=1$.
    we call $\gamma\circ\vphi$ a \textbf{reparametrization} of $\gamma$.
\end{defn} \

\begin{lm}
    $\gamma\simeq_{p}\gamma\circ\vphi$
\end{lm}

\begin{pf}[source=Primary Source Material]
    $F(s,t)=\gamma((1-t)s + t\vphi(s))$ is a pathtopy.
\end{pf}

okay the rest is just proving that $\pi_{1}(X,x_{0})$ is a grp under $*$.
uhh the pfs look kinda annoying to write so im just        not gonna.
uses the reparametrization tho

\lecdate{Lec 20 - Jul 25 (Week 11)}

from last time: given $X$ and a basept $x_{0}\in X$,
we associated the group $\pi_{1}(X,x_{0})$ to $(X,x_{0})$
called the fundamental grp of $X$ w basept $x_{0}$.

$\pi_{1}$ is also known as a \textbf{functor}.
\begin{gather*}
    \trm{Topological space} \ \xrightarrow{\quad \pi_{1} \quad} \ \trm{Group} \\
    \trm{Continuous map} \ \xrightarrow{\quad \quad} \ \trm{Homomorphism} \\
    \trm{Homeomorphism} \ \xrightarrow{\quad \quad} \ \trm{Isomorphism}
\end{gather*}

today we will prove this! whatever that means.
in the meantime: did you know the torus has fundamental group $\bZ$?

Q: what happens to $\pi_{1}(X,x_{0})$ if we change the basept?

\begin{prop}
    fix $x_{0},x_{1}\in X$.
    let $\alpha$ be a path from $x_{0}$ to $x_{1}$.
    define $\hat{\alpha}:\pi_{1}(X,x_{0})\sto\pi_{1}(X,x_{1})$ as:
    \begin{equation*}
        \hat{\alpha}([\gamma])=[\bar{\alpha}*\gamma*\alpha]
    \end{equation*}
    then $\hat{\alpha}$ is well-defined, and
    $\hat{\alpha}$ is an isomorphism.
\end{prop} \

\begin{pf}[source=Primary Source Material]
    well-definedness is an exercise. show:
    \begin{equation*}
        \gamma_{0}\simeq_{p}\gamma_{1}\ \implies \
        \bar{\alpha}*\gamma_{0}*\alpha\simeq_{p}\bar{\alpha}*\gamma_{1}*\alpha
    \end{equation*}
    it is a homomorphism because:
    \begin{align*}
        \hat{\alpha}([\gamma_{0}]*[\gamma_{1}])
        = \hat{\alpha}([\gamma_{0}*\gamma_{1}])
        & = [\bar{\alpha}*\gamma_{0}*\gamma_{1}*\alpha] \\
        & = [\bar{\alpha}*\gamma_{0}*e_{x_{0}}*\gamma_{1}*\alpha] \\
        & = [\bar{\alpha}*\gamma_{0}*\alpha*\bar{\alpha}*\gamma_{1}*\alpha] \\
        & = [\bar{\alpha}*\gamma_{0}*\alpha]*[\bar{\alpha}*\gamma_{1}*\alpha] \\
        & = \hat{\alpha}([\gamma_{0}])*\hat{\alpha}([\gamma_{1}])
    \end{align*}
    it is bijective because:
    \begin{equation*}
        \hat{\alpha}\circ\hat{\bar{\alpha}}([\gamma])
        = \hat{\alpha}([\alpha*\gamma*\bar{\alpha}])
        = [\bar{\alpha}*\alpha*\gamma*\bar{\alpha}*\alpha]
        = [\gamma]
    \end{equation*}
    $\hat{\bar{\alpha}}\circ\hat{\alpha}$ is similarly id.
\end{pf} \

\begin{crll}
    $X$ pathconn then $\pi_{1}(X,x_{0})\simeq\pi_{1}(X,x_{1})$ for all
    $x_{0},x_{1}$.
    in this case, fundgrp does not depend on basept, and we can denote it
    $\pi_{1}(X)$.
\end{crll}

\begin{defn}
    $X$ is \textbf{simply connected} iff $X$ pathconn and fundgrp is trivial.
\end{defn}
for instance, any cvx subset of $\bR^{n}$ is simply conn.

notation: we write $\vphi:(X,x_{0})\sto(Y,y_{0})$ to mean that $\vphi$ cts,
$\vphi(x_{0})=x_{1}$.

any map $\vphi:(X,x_{0})\sto(X,x_{1})$ induces a homomorphism
$\vphi_{*}:\pi_{1}(X,x_{0})\sto\pi_{1}(X,x_{1})$ as:
\begin{equation*}
    \vphi_{*}([\gamma]) = [\vphi\circ\gamma]
\end{equation*}
this is known as the \textbf{induced map} of $\vphi$.
exercise: check this is well-defined, that is
$\gamma_{0}\simeq_{p}\gamma_{1}\ \implies \ \vphi\circ\gamma_{0}\simeq_{p}
\vphi\circ\gamma_{1}$.
idea: if $F:[0,1]^{2}\sto X$ is path homotopy from $\gamma_{0}$ to $\gamma_{1}$,
then $G=\vphi\circ F$ is a path homotopy from $\vphi\circ\gamma_{0}$ to
$\vphi\circ\gamma_{1}$. (check this!)

\begin{prop}
    let $\vphi:(X,x_{0})\sto(Y,y_{0})$.
    then the induced map is a homomorphism.
\end{prop}

\begin{pf}[source=Primary Source Material]
    we check $\vphi_{*}([\gamma_{0}]*[\gamma_{1}]) =
    \vphi_{*}([\gamma_{0}])*\vphi_{*}([\gamma_{1}])$.
    \begin{gather*}
        \vphi_{*}([\gamma_{0}]*[\gamma_{1}])
        \ = \ \vphi_{*}([\gamma_{0}*\gamma_{1}])
        \ = \ [\vphi\circ(\gamma_{0}*\gamma_{1})] \\
        \ \textcolor{red}{=} \ [(\vphi\circ\gamma_{0})*(\vphi\circ\gamma_{1})]
        \ = \ \vphi_{*}([\gamma_{0}])*\vphi_{*}([\gamma_{1}])
    \end{gather*}
    to see red equality, note that $\vphi\circ(\gamma_{0}*\gamma_{1})
    =(\vphi\circ\gamma_{0})*(\vphi\circ\gamma_{1})$. check this!
\end{pf}

some properties:
\begin{enumerate}[(i)]
    \item if $\vphi:(X,x_{0})\sto(Y,y_{0})$ and $\psi:(Y,y_{0})\sto(Z,z_{0})$,
        then $(\psi\circ\vphi)_{*}=\psi_{*}\circ\vphi_{*}$.
    \item if $\iota:(X,x_{0})\sto(X,x_{0})$ is id, then $\iota_{*}$ is id.
    \item if $\vphi:(X,x_{0})\sto(Y,y_{0})$ is homeo, then $\vphi_{*}$ is iso.
\end{enumerate}
proofs:
\begin{enumerate}[(i)]
    \item $(\vphi\circ\psi)_{*}([\gamma])=[\vphi\circ\psi\circ\gamma]
        =\vphi_{*}([\psi\circ\gamma])=\vphi_{*}(\psi_{*}([\gamma]))$
    \item $\iota_{*}([\gamma])=[\iota\circ\gamma]=[\gamma]$
    \item by $(i)$ and $(ii)$, $\vphi_{*}\circ(\vphi^{-1})_{*}=\iota_{*}$ and
        $(\vphi^{-1})_{*}\circ\vphi_{*}=\iota_{*}$.
        this also shows $(\vphi_{*})^{-1}=(\vphi^{-1})_{*}$.
\end{enumerate}

summary: given the following:
\begin{equation*}
    (X,x_{0}) \xrightarrow{\qquad \vphi \qquad} (Y,y_{0})
\end{equation*}
we can apply $\pi_{1}$ to transform this into:
\begin{equation*}
    \pi_{1}(X,x_{0}) \xrightarrow{\qquad \vphi_{*} \qquad} \pi_{1}(Y,y_{0})
\end{equation*}


