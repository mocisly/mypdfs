\subsection{Path Homotopy}
\lecdate{Lec 19 - Jul 23 (Week 11)}
ALGTOP RAAAHHHHHHHHHHHHH

\begin{defn}
    given $\gamma_{0},\gamma_{1}$ from $x$ to $y$, a \textbf{path homotopy}
    from $\gamma_{0}$ to $\gamma_{1}$ is a cts $F:[0,1]^{2}\sto X$ such that
    the following holds:
    \begin{equation*}
        F(s,0)=\gamma_{0}(s) \trm{ and } F(s,1)=\gamma_{1}(s) \qquad
        F(0,t)=x \trm{ and } F(1,t)=y
    \end{equation*}
    we say $\gamma_{0}$ is \textbf{path homotopic} to $\gamma_{1}$,
    or $\gamma_{0}\simeq_{p}\gamma_{1}$, if a pathtopy exists
\end{defn}
[pathtopy is certainly a. choice.]
note that its important for $F$ to be cts; this is strictly stronger than
requiring $F$ to be cts in each coordinate (beloved $xy/x^{2}+y^{2}$).

\begin{xmp}[source=Primary Source Material]
    $A\subseteq\bR^{n}$ cvx $\implies$ any two paths w/ same endpoints are
    homotopic: for fixed $s$, take $F(s,t)=(1-t)\gamma_{0}(s)+t\gamma_{1}(s)$.
    check that this is a pathtopy.
\end{xmp}

we can generalize pathtopy to deformations of general cts functions.
\begin{defn}
    sps $f,g:X\sto Y$ cts.
    a \textbf{homotopy} from $f$ to $g$ is a cts $F:X\times[0,1]\sto Y$ with
    \begin{equation*}
        F(x,0)=f(x) \qquad F(x,1) = g(x)
    \end{equation*}
    we write $f\simeq g$ in this case.
\end{defn}
clearly, both $\simeq, \simeq_{p}$ are equiv rels.
note transitivity uses pasting lemma [technically. its like a single pt tho].

\subsection{The Fundamental Group}
[yeah this should go here tbh]

\begin{defn}
    for a path $\gamma_{0}$ from $x$ to $y$ and $\gamma_{1}$ from $y$ to $z$,
    define:
    \begin{equation*}
        \gamma_{0}*\gamma_{1}(s) =
        \begin{cases}
            \gamma_{0}(2s) & s \in[0,1/2] \\
            \gamma_{1}(2s-1) & s \in[1/2,1]
        \end{cases}
    \end{equation*}
    this induces an operation $*$ on the set of equivalence classes.
\end{defn} \

\begin{prop}
    $*$ is well-defined on equivalence classes.
\end{prop} \

\begin{pf}[source=Primary Source Material]
    fix $\gamma_{0}\simeq\gamma_{0}'$ and $\gamma_{1}\simeq\gamma_{1}'$.
    let $F,G:[0,1]^{2}\sto X$ be pathtopies from $\gamma_{i}$ to $\gamma_{i}'$
    respectively. consider:
    \begin{equation*}
        H(s,t)=
        \begin{cases}
            F(2s,t) & (s,t)\in[0,1/2]\times[0,1] \\
            G(2s-1,t) & (s,t)\in[1/2,1]\times[0,1]
        \end{cases}
    \end{equation*}
    it is easy to see that $H$ is then a homotopy from $\gamma_{0}*\gamma_{1}$
    to $\gamma_{0}'*\gamma_{1}'$.
\end{pf} \

\begin{defn}
    a \textbf{loop} is a path $\gamma:[0,1]\sto X$ with $\gamma(0)=\gamma(1)$.
    we say that $\gamma$ is a loop at $x_{0}$ if $\gamma(0)=\gamma(1)=x_{0}$.

    given a fixed $x_{0}$, we denote by $\pi_{1}(X,x_{0})$ the set of all
    pathtopy equiv classes of loops at $x_{0}$.
\end{defn}
we define by $e_{x}:[0,1]\sto X$ and $\bar{\gamma}:[0,1]\sto X$ as:
\begin{equation*}
    e_{x}(s)=x \qquad \bar{\gamma}(s)=\gamma(1-s)
\end{equation*}
these are the ``constant" and ``inverse" paths respectively.

\begin{defn}
    given $\gamma$, let $\vphi:[0,1]\sto[0,1]$ be cts with $\vphi(0)=0$ and
    $\vphi(1)=1$.
    we call $\gamma\circ\vphi$ a \textbf{reparametrization} of $\gamma$.
\end{defn} \

\begin{lm}
    $\gamma\simeq_{p}\gamma\circ\vphi$
\end{lm}

\begin{pf}[source=Primary Source Material]
    $F(s,t)=\gamma((1-t)s + t\vphi(s))$ is a pathtopy.
\end{pf}

okay the rest is just proving that $\pi_{1}(X,x_{0})$ is a grp under $*$.
uhh the pfs look kinda annoying to write so im just        not gonna.
uses the reparametrization tho

\lecdate{Lec 20 - Jul 25 (Week 11)}

from last time: given $X$ and a basept $x_{0}\in X$,
we associated the group $\pi_{1}(X,x_{0})$ to $(X,x_{0})$
called the fundamental grp of $X$ w basept $x_{0}$.

$\pi_{1}$ is also known as a \textbf{functor}.
\begin{gather*}
    \trm{Topological space} \ \xrightarrow{\quad \pi_{1} \quad} \ \trm{Group} \\
    \trm{Continuous map} \ \xrightarrow{\quad \quad} \ \trm{Homomorphism} \\
    \trm{Homeomorphism} \ \xrightarrow{\quad \quad} \ \trm{Isomorphism}
\end{gather*}

today we will prove this! whatever that means.
in the meantime: did you know the torus has fundamental group $\bZ$?

Q: what happens to $\pi_{1}(X,x_{0})$ if we change the basept?

\begin{prop}
    fix $x_{0},x_{1}\in X$.
    let $\alpha$ be a path from $x_{0}$ to $x_{1}$.
    define $\hat{\alpha}:\pi_{1}(X,x_{0})\sto\pi_{1}(X,x_{1})$ as:
    \begin{equation*}
        \hat{\alpha}([\gamma])=[\bar{\alpha}*\gamma*\alpha]
    \end{equation*}
    then $\hat{\alpha}$ is well-defined, and
    $\hat{\alpha}$ is an isomorphism.
\end{prop} \

\begin{pf}[source=Primary Source Material]
    well-definedness is an exercise. show:
    \begin{equation*}
        \gamma_{0}\simeq_{p}\gamma_{1}\ \implies \
        \bar{\alpha}*\gamma_{0}*\alpha\simeq_{p}\bar{\alpha}*\gamma_{1}*\alpha
    \end{equation*}
    it is a homomorphism because:
    \begin{align*}
        \hat{\alpha}([\gamma_{0}]*[\gamma_{1}])
        = \hat{\alpha}([\gamma_{0}*\gamma_{1}])
        & = [\bar{\alpha}*\gamma_{0}*\gamma_{1}*\alpha] \\
        & = [\bar{\alpha}*\gamma_{0}*e_{x_{0}}*\gamma_{1}*\alpha] \\
        & = [\bar{\alpha}*\gamma_{0}*\alpha*\bar{\alpha}*\gamma_{1}*\alpha] \\
        & = [\bar{\alpha}*\gamma_{0}*\alpha]*[\bar{\alpha}*\gamma_{1}*\alpha] \\
        & = \hat{\alpha}([\gamma_{0}])*\hat{\alpha}([\gamma_{1}])
    \end{align*}
    it is bijective because:
    \begin{equation*}
        \hat{\alpha}\circ\hat{\bar{\alpha}}([\gamma])
        = \hat{\alpha}([\alpha*\gamma*\bar{\alpha}])
        = [\bar{\alpha}*\alpha*\gamma*\bar{\alpha}*\alpha]
        = [\gamma]
    \end{equation*}
    $\hat{\bar{\alpha}}\circ\hat{\alpha}$ is similarly id.
\end{pf} \

\begin{crll}
    $X$ pathconn then $\pi_{1}(X,x_{0})\simeq\pi_{1}(X,x_{1})$ for all
    $x_{0},x_{1}$.
    in this case, fundgrp does not depend on basept, and we can denote it
    $\pi_{1}(X)$.
\end{crll}

\begin{defn}
    $X$ is \textbf{simply connected} iff $X$ pathconn and fundgrp is trivial.
\end{defn}
for instance, any cvx subset of $\bR^{n}$ is simply conn.

notation: we write $\vphi:(X,x_{0})\sto(Y,y_{0})$ to mean that $\vphi$ cts,
$\vphi(x_{0})=x_{1}$.

any map $\vphi:(X,x_{0})\sto(X,x_{1})$ induces a homomorphism
$\vphi_{*}:\pi_{1}(X,x_{0})\sto\pi_{1}(X,x_{1})$ as:
\begin{equation*}
    \vphi_{*}([\gamma]) = [\vphi\circ\gamma]
\end{equation*}
this is known as the \textbf{induced map} of $\vphi$.
exercise: check this is well-defined, that is
$\gamma_{0}\simeq_{p}\gamma_{1}\ \implies \ \vphi\circ\gamma_{0}\simeq_{p}
\vphi\circ\gamma_{1}$.
idea: if $F:[0,1]^{2}\sto X$ is path homotopy from $\gamma_{0}$ to $\gamma_{1}$,
then $G=\vphi\circ F$ is a path homotopy from $\vphi\circ\gamma_{0}$ to
$\vphi\circ\gamma_{1}$. (check this!)

\begin{prop}
    let $\vphi:(X,x_{0})\sto(Y,y_{0})$.
    then the induced map is a homomorphism.
\end{prop}

\begin{pf}[source=Primary Source Material]
    we check $\vphi_{*}([\gamma_{0}]*[\gamma_{1}]) =
    \vphi_{*}([\gamma_{0}])*\vphi_{*}([\gamma_{1}])$.
    \begin{gather*}
        \vphi_{*}([\gamma_{0}]*[\gamma_{1}])
        \ = \ \vphi_{*}([\gamma_{0}*\gamma_{1}])
        \ = \ [\vphi\circ(\gamma_{0}*\gamma_{1})] \\
        \ \textcolor{red}{=} \ [(\vphi\circ\gamma_{0})*(\vphi\circ\gamma_{1})]
        \ = \ \vphi_{*}([\gamma_{0}])*\vphi_{*}([\gamma_{1}])
    \end{gather*}
    to see red equality, note that $\vphi\circ(\gamma_{0}*\gamma_{1})
    =(\vphi\circ\gamma_{0})*(\vphi\circ\gamma_{1})$. check this!
\end{pf}

some properties:
\begin{enumerate}[(i)]
    \item if $\vphi:(X,x_{0})\sto(Y,y_{0})$ and $\psi:(Y,y_{0})\sto(Z,z_{0})$,
        then $(\psi\circ\vphi)_{*}=\psi_{*}\circ\vphi_{*}$.
    \item if $\iota:(X,x_{0})\sto(X,x_{0})$ is id, then $\iota_{*}$ is id.
    \item if $\vphi:(X,x_{0})\sto(Y,y_{0})$ is homeo, then $\vphi_{*}$ is iso.
\end{enumerate}
proofs:
\begin{enumerate}[(i)]
    \item $(\vphi\circ\psi)_{*}([\gamma])=[\vphi\circ\psi\circ\gamma]
        =\vphi_{*}([\psi\circ\gamma])=\vphi_{*}(\psi_{*}([\gamma]))$
    \item $\iota_{*}([\gamma])=[\iota\circ\gamma]=[\gamma]$
    \item by $(i)$ and $(ii)$, $\vphi_{*}\circ(\vphi^{-1})_{*}=\iota_{*}$ and
        $(\vphi^{-1})_{*}\circ\vphi_{*}=\iota_{*}$.
        this also shows $(\vphi_{*})^{-1}=(\vphi^{-1})_{*}$.
\end{enumerate}

summary: given the following:
\begin{equation*}
    (X,x_{0}) \xrightarrow{\qquad \vphi \qquad} (Y,y_{0})
\end{equation*}
we can apply $\pi_{1}$ to transform this into:
\begin{equation*}
    \pi_{1}(X,x_{0}) \xrightarrow{\qquad \vphi_{*} \qquad} \pi_{1}(Y,y_{0})
\end{equation*}

\subsection{Covering Spaces}
\lecdate{Lec 21 - Jul 30 (Week 12)}

today's mission: prove $\pi_{1}(S^{1})=\bZ$.
we notate $I=[0,1]$.

proof sketch: let $\omega_{n}(s)=(\cos(2\pi sn),\sin(2\pi sn))$.
this is a loop in $S^{1}$ at $x_{0}=(1,0)$ that does $n$ revolutions.
moves ccw if $n>0$, cw if $n<0$.
we want to show any loop in $S^{1}$ is pathtopic to some $\omega_{n}$.

idea: show every $\gamma:I\sto S^{1}$ can be ``uniquely lifted" to a path $\tilde{\gamma}:I\sto\bR$
from $0$ to some $n$.
embed $\bR\ito\bR^{3}$ as the ``helix":
\begin{equation*}
    s\sto(\cos(2\pi s),\sin(2\pi s),s)
\end{equation*}
let $P:\bR\sto S^{1}$ be the projection of the helix onto the $xy$-plane.
we want to show two things:
\begin{enumerate}[(a)]
    \item for any loop $\gamma:I\sto S^{1}$, there is a unique $\tilde{\gamma}:I\sto\bR$ starting
        at $0$ s.t.:

        [commutative diagram]

    \item for every pathtopy $F:I^{2}\sto S^{1}$ s.t. $F(0,0)=x_{0}$, there is a unique
        pathtopy $\tilde{F}:I^{2}\sto\bR$ s.t. $\tilde{F}(0,0)=0$ and:

        [commutative diagram]
\end{enumerate}

\begin{defn}
    given spaces $X,E$, we say $P:E\sto X$ is a \textbf{covering map} if for every
    $x_{0}\in X$ there is an open $x_{0}\in U$ which is ``evenly covered by $P$", i.e.:
    \begin{equation*}
        P^{-1}(U)=\bigsqcup_{\alpha\in\Lambda}V_{\alpha}
    \end{equation*}
    is a union of pairwise disjoint open sets in $E$ such that the map
    $P\rvert_{V_{\alpha}}:V_{\alpha}\sto U$ is a homomorphism.

    in this context, we say $E$ is a \textbf{covering space} of $X$.
    each $V_{\alpha}$ is also known as a \textbf{sheet} or \textbf{slice}.
\end{defn}
some examples:
\begin{itemize}
    \item $\trm{id}:X\sto X$ is a cvring space (1-sheeted)
    \item $P:\bR\sto S^{1}$ given by $P(s)=(\cos(2\pi s),\sin(2\pi s))$ (countably many sheets)
    \item $P_{n}:S^{1}\sto S^{1}$ given by $P_{n}(z)=z^{n}$, where $S^{1}\subseteq\bC$
        ($n$-sheeted)
\end{itemize}

properties of covering maps:
\begin{itemize}
    \item $\forall \, x\in X$, $P^{-1}(x)$ is a discrete subspace of $E$.
        that is, every $e\in P^{-1}(x)$ has an open $V\subseteq E$ s.t. $V\cap P^{-1}(x)=\set{e}$
    \item cvring maps are open maps (exercise)
    \item cvring maps are \textbf{local homeos}. that is, for all
        $e\in E$, there is an open $e\in V\subseteq E$ s.t. $P\rvert_{V}:V\sto P(V)$ is a homeo
\end{itemize}

\begin{defn}
    let $P:E\sto X$ be a cvring map, $f:X\sto Y$ be cts.
    a \textbf{lifting of} $\mbf{f}$ is a cts $\tilde{f}:Y\sto E$ s.t. $f=P\circ\tilde{f}$
    
    [diagram]
\end{defn} \

\begin{lm}[title=Path-lifting property]
    let $P:E\sto X$ be a cvring map and $x=P(e)$.
    if $\gamma:I\sto X$ is a path starting at $x$, then there is a unique lifting to a path
    $\tilde{\gamma}:I\sto E$ starting at $e$.
\end{lm}

\begin{pf}[source=Primary Source Material]
    step 1: find a partition $0=t_{0}<t_{1}<\dots<t_{n}=1$ s.t. $\gamma([t_{i-1},t_{i}])$ is
    contained in some evenly covered open $U_{i}$.
    \begin{block}
        for all $t\in I$, $\gamma(t)\in X$.
        then $\exists$ open nbhd $U_{t}$ of $\gamma(t)$ s.t. each $U_{t}$ evenly cvred:
        \begin{equation*}
            I=\bigcup_{t\in I}\gamma^{-1}(U_{})
        \end{equation*}
        by lebesgue number lemma, let $\delta>0$ s.t. for any $\trm{diam}(A)<\delta$,
        there is some $t\in I$ s.t. $A\subseteq\gamma^{-1}(U_{t})$.
        choose a partition $P=\set{t_{0},\dots,t_{n}}$ s.t.
        $\norm{P}=\max\abs{t_{i}-t_{i-1}}<\delta$.
        then we have that $[t_{i-1},t_{i}]\subseteq\gamma^{-1}(U)$ for some open $U$ evenly cvred.
    \end{block}
    step 2: we prove existence of $\tilde{\gamma}$.
    we construct $\tilde{\gamma}$ inductively on each subinterval of the partition.
    \begin{block}
        $\tilde{\gamma}(0)=e_{0}$.
        sps $\tilde{\gamma}$ defined on $[0,t_{i-1}]$.
        we extend to $[0, t_{i}]$ by defining $\tilde{\gamma}$ on the next
        subinterval $[t_{i-1},t_{i}]$.

        by step 1, there is an open $U$ evenly covered s.t. $\gamma([t_{i-1},t_{i}])\subseteq U$.
        notice that $\tilde{\gamma}(t_{i-1})\in P^{-1}(U)$ since:
        \begin{equation*}
            P\circ\tilde{\gamma}(t_{i-1}) = \gamma(t_{i-1})\in U
        \end{equation*}
        then $\tilde{\gamma}(t_{i-1})\in V_{\alpha}$ where $V_{\alpha}$ is a sheet of $P^{-1}(U)$.

        recall $P\rvert_{V_{\alpha}}:V_{\alpha}\sto U$ is homeo. define:
        \begin{equation*}
            \tilde{\gamma}(s) = \left( P^{-1}\rvert_{V_{\alpha}} \right)(\gamma(s)) \qquad
            s \in [t_{i-1},t_{i}]
        \end{equation*}
        note $\tilde{\gamma}\rvert_{[0,t_{i}]}$ cts by pasting lemma, and
        $\gamma=P\circ\tilde{\gamma}$.
    \end{block}
    step 3: we prove uniqueness of $\tilde{\gamma}$.
    \begin{block}
        sps $\hat{\gamma}$ is another lifting of $\gamma$ with $\hat{\gamma}(0)=e_{0}$.
        we show $\hat{\gamma}(s)=\tilde{\gamma}(s)$ for all $s \in[t_{i-1},t_{i}]$ inductively.

        sps $\hat{\gamma}\rvert_{[0,t_{i-1}]}=\tilde{\gamma}\rvert_{[0,t_{i-1}]}$.
        note $\hat{\gamma}([t_{i-1},t_{i}])$ conn and $\hat{\gamma}(t_{i-1})\in V_{\alpha}$
        where $V_{\alpha}$ is the sheet we used to define $\tilde{\gamma}$ on $[t_{i-1},t_{i}]$ in
        step 2.

        it then follows that $\hat{\gamma}([t_{i-1},t_{i}])\subseteq V_{\alpha}$ by conn.
        since $\gamma=P\circ\hat{\gamma}$:
        \begin{equation*}
            \hat{\gamma}(s)=\left(P^{-1}\rvert_{V_{\alpha}}\right)(\gamma(s)) = \tilde{\gamma}(s)
        \end{equation*}
        for all $s \in[t_{i-1},t_{i}]$.
    \end{block}
\end{pf}

\newpage
\begin{lm}[title=Path-homotopy lifting property]
    sps $P:E\sto X$ cvring map, $x=P(e)$.
    if $F:I^{2}\sto X$ is a pathtopy with $F(0,0)=x$, then there is
    a unique lifting $\tilde{F}:I^{2}\sto E$ which is a \textbf{pathtopy} in $E$ s.t.
    $\tilde{F}(0,0)=e$.
\end{lm}
pf: very similar to the one above. not writing allat

\begin{crll}
    let $P:E\sto X$ be a cvring map and $x=P(e)$.
    if $\gamma_{0}\simeq_{p}\gamma_{1}$, then $\tilde{\gamma_{0}}\simeq_{p}\tilde{\gamma_{1}}$.
\end{crll}

\begin{thm}
    $\pi_{1}(S^{1})=\bZ$.
\end{thm}
dont rly wanna put this in a box

let $x_{0}=(1,0)\in S^{1}$ and $P:\bR\sto S^{1}$ be the cvring map given by:
\begin{equation*}
    P(s) = (\cos(2\pi s),\sin(2\pi s))
\end{equation*}
given $[\gamma]\in\pi_{1}(S^{1},x_{0})$, let $\tilde{\gamma}:I\sto\bR$ be the unique lifting of
$\gamma$ s.t. $\tilde{\gamma}(0)=0$.
define $\vphi:\pi_{1}(S^{1},x_{0})\sto\bZ$ given by:
\begin{equation*}
    \vphi([\gamma]) = \tilde{\gamma}(1)\in P^{-1}(x_{0})=\bZ
\end{equation*}
first, we show $\vphi$ is well-defined.
\begin{block}
    by prev crll, if $\gamma_{0}\simeq_{p}\gamma_{1}$, then
    $\tilde{\gamma}_{0}\simeq_{p}\tilde{\gamma}_{1}$.
    in particular, $\tilde{\gamma}_{0}(1)=\tilde{\gamma}_{1}(1)$.
\end{block}
next, $\vphi$ is surjective.
\begin{block}
    fix $n \in\bZ$. consider $\omega_{n}(s)=(\cos(2\pi sn),\sin(2\pi sn))$.
    then $\vphi([\omega_{n}])=\tilde{\omega_{n}}(1)=n$.
\end{block}
next, $\vphi$ is injective.
\begin{block}
    sps $[\gamma_{0}],[\gamma_{1}]\in\pi_{1}(S^{1},x_{0})$ s.t.
    $\vphi([\gamma_{0}])=\vphi([\gamma_{1}])$.
    note $\tilde{\gamma_{0}},\tilde{\gamma_{1}}:I\sto\bR$ are paths in $\bR$.
    since $\bR$ cvx, then $\tilde{\gamma_{0}}\simeq_{p}\tilde{\gamma_{1}}$ via pathtopy
    $\tilde{F}:I^{2}\sto\bR$.
    then, $P\circ F:I^{2}\sto S^{1}$ is a pathtopy from $P\circ\tilde{\gamma_{0}}=\gamma_{0}$ to
    $P\circ\tilde{\gamma_{1}}=\gamma_{1}$, so $[\gamma_{0}]=[\gamma_{1}]$.
\end{block}
finally, $\vphi$ is homo.
\begin{block}
    given $[\gamma_{0}],[\gamma_{1}]\in\pi_{1}(S^{1},x_{0})$, let:
    \begin{equation*}
        \tilde{\gamma_{0}}(1) = n \qquad \tilde{\gamma_{1}}(1) = m
    \end{equation*}
    then $\gamma_{0}\simeq_{p}\omega_{n}$ and $\gamma_{1}\simeq_{p}\omega_{m}$.
    we show $\vphi([\omega_{n}]*[\omega_{m}])=n+m$.

    let $\tilde{\omega_{n}}$ be the lifting starting at $0$ and ending at $n$.
    let $\tilde{\omega_{m}}$ be the lifting starting at $n$. then, we have that:
    \begin{equation*}
        \vphi([\omega_{n}]*[\omega_{m}])=(\tilde{\omega_{n}}*\tilde{\omega_{m}})(1)=n+m
    \end{equation*}
\end{block}
note: check that $\tilde{\omega_{n}}*\tilde{\omega_{m}}$ is indeed a lifting of
$\omega_{n}*\omega_{m}$.




